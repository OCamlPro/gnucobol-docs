\chapter{Key}

\begin{table}[!h]
  \centering
  \begin{tabular}[!h]{p{0.4\textwidth} p{0.5\textwidth}}
    \toprule
    Element & Notes \\ \midrule
    Braces, $\left\{\ {}\right\}$ & One element within the braces must be selected. \\
    Brackets, $\left[\ {}\right]$ & One or zero elements within the brackets must be selected. \\
    Vertical lines, $\left|\ {}\right|$ & Each element may be selected once and in any order; if within braces, at least one element must be selected. \\
    Ellipsis, \ldots & The preceding element may be repeated any number of times. \\
    OPTIONAL-RESERVED-WORD & \\
    \key{MANDATORY-RESERVED-WORD} & Mandatory reserved words in brackets are often used instead of optional reserved words to indicate an optional feature. \\
    \deleted{Deleted element} & These elements were previously in the COBOL standard but have since been deleted. Their use is strongly discouraged. \\
    \archaic{Archaic element} & These elements remain in the standard, but their use is considered poor style and is strongly discouraged. \\
    \obsolete{Obsolete element} & These elements are slated to be deleted from the standard. Their use is strongly discouraged. \\
    \xopen{X\slash{}Open extension} & \\
    \gnucobol{GnuCOBOL-only extension} & \\
    \miscext{Miscellaneous extension} & An extension which may have come from COBOL dialects by AcuCorp, CA, Fujitsu, HP, IBM, Micro Focus, Microsoft or Ryan-McFarland. \\
    \pending{Unimplemented element} & These elements are recognised by GnuCOBOL, but are non-functional. \\ \bottomrule
  \end{tabular}
\end{table}

%%% Local Variables:
%%% mode: latex
%%% TeX-master: "grammar.tex"
%%% End:
