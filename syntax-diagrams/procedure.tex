\chapter{Procedure division}

\key{PROCEDURE} \key{DIVISION}
\begin{0-1}
using-chaining-clause
\end{0-1}
\begin{0-1}
  \key{RETURNING}
  \begin{1=}
    \identifier \\
    \key{OMITTED}
  \end{1=}
\end{0-1}.\newline
\begin{0-1}
  declaratives
\end{0-1}\newline
\begin{0-1}
  section-name-2 \key{SECTION}. \\
  paragraph-name-2. \\
  \imperativestatement .
\end{0-1} \ldots

where using-chaining-clause is

\begin{1=}
  \key{USING} \\
  \miscext{\key{CHAINING}}
\end{1=}

\begin{1=}
  BY
  \begin{1=}
    \key{REFERENCE} \\
    \pending{\key{VALUE}}
  \end{1=}
  \miscext{
    \begin{0-1}
      \begin{0-1}
        \key{UNSIGNED}
      \end{0-1}
      \key{SIZE} IS
      \begin{1=}
        \key{AUTO} \\
        \integer
      \end{1=} \\

      \key{SIZE} IS \key{DEFAULT}
    \end{0-1}
  }

  \begin{0-1}
    \key{OPTIONAL}
  \end{0-1}
  \identifier
\end{1=}\ldots

where declaratives is

\key{DECLARATIVES}.

\begin{0-1}
  section-name-1 \key{SECTION}.
  use-statement
  \begin{0-1}
    \begin{0-1}
      paragraph-name-2.
    \end{0-1}
    \imperativestatement .
  \end{0-1} \ldots
\end{0-1}\ldots

\key{END} \key{DECLARATIVES}.

\section{Common phrases}

\subsection{ROUNDED phrase}

\key{ROUNDED}
\begin{0-1}
  \key{MODE} IS
  \begin{1=}
    \key{AWAY-FROM-ZERO} \\
    \key{NEAREST-AWAY-FROM-ZERO} \\
    \key{NEAREST-EVEN} \\
    \key{NEAREST-TOWARD-ZERO} \\
    \key{PROHIBITED} \\
    \key{TOWARD-GREATER} \\
    \key{TOWARD-LESSER} \\
    \key{TRUNCATION}
  \end{1=}
\end{0-1}

\subsection{SIZE phrase}

\gnucobol{
  \begin{1=}
    \key{SIZE} IS \key{AUTO} \\
    \key{SIZE} IS \key{DEFAULT} \\
    \key{SIZE} IS \integer \\
    \key{UNSIGNED} \key{SIZE} IS \key{AUTO} \\
    \key{UNSIGNED} \key{SIZE} IS \integer
  \end{1=}
}
\section{ACCEPT statement}

\format{device}
\key{ACCEPT}
\begin{1=}
  \identifier \\
  \miscext{\key{OMITTED}}
\end{1=}
\begin{0-1}
  \key{FROM} \mnemonicname
\end{0-1}
\begin{0-1}
  \key{END-ACCEPT}
\end{0-1}

\format{screen}
\key{ACCEPT}
\begin{1=}
  \identifier \\
  \miscext{\key{OMITTED}}
\end{1=}

\begin{0+}
  \begin{1=}
    \begin{1+}
      AT \key{LINE} NUMBER
      \begin{1=}
        \identifier \\
        \integer
      \end{1=} \\

      AT
      \begin{1=}
        \key{COLUMN} \\
        \key{COL} \\
        \miscext{\key{POSITION}}
      \end{1=}
      NUMBER
      \begin{1=}
        \identifier \\
        \integer
      \end{1=}
    \end{1+} \\

    \miscext{
      \key{AT}
      \begin{1=}
        \identifier \\
        \integer
      \end{1=}
    }
  \end{1=} \\

  \miscext{\key{FROM} \key{CRT}} \\
  \miscext{\key{MODE} IS \key{BLOCK}} \\
  \miscext{screen-attribute-clauses}
\end{0+}

where screen-attribute-clauses is

\begin{0-1}
WITH \key{AUTO}
\end{0-1}

\begin{0-1}
  WITH \key{TAB}
\end{0-1}

\begin{0-1}
  WITH
  \begin{1=}
    \key{BELL} \\
    \key{BEEP}
  \end{1=}
\end{0-1}

\begin{0-1}
WITH \key{BLINK}
\end{0-1}

\begin{0-1}
  WITH \key{CONVERSION}
\end{0-1}

\begin{0-1}
  WITH
  \begin{1=}
    \key{FULL} \\
    \key{LENGTH-CHECK} \\
  \end{1=}
\end{0-1}

\begin{0-1}
  WITH
  \begin{1=}
    \key{HIGHLIGHT} \\
    \key{LOWLIGHT}
  \end{1=}
\end{0-1}

\begin{0-1}
  WITH \key{LEFTLINE}
\end{0-1}

\begin{0-1}
  WITH \key{LOWER}
\end{0-1}

\begin{0-1}
  WITH \key{NO-ECHO}
\end{0-1}
  
\begin{0-1}
  WITH \key{OVERLINE}
\end{0-1}

\begin{0-1}
  WITH \key{PROMPT}
  \begin{0-1}
    \key{CHARACTER} IS
    \begin{1=}
      \identifier \\
      \literal
    \end{1=}
  \end{0-1}
\end{0-1}

\begin{0-1}
  WITH
  \begin{1=}
    \key{REQUIRED} \\
    \key{EMPTY-CHECK}
  \end{1=}
\end{0-1}

\begin{0-1}
WITH \key{REVERSE-VIDEO}
\end{0-1}

\begin{0-1}
WITH \key{SECURE}
\end{0-1}

\begin{0-1}
  WITH PROTECTED \key{SIZE} IS
  \begin{1=}
    \identifier \\
    \integer
  \end{1=}
\end{0-1}

\begin{0-1}
WITH \key{UNDERLINE}
\end{0-1}

\begin{0-1}
  WITH
  \begin{0-1}
    \key{NO}
  \end{0-1}
  \begin{1=}
    \key{DEFAULT} \\
    \key{UPDATE}
  \end{1=}
\end{0-1}
  
\begin{0-1}
  WITH \key{UPPER}
\end{0-1}

\begin{0-1}
  WITH
  \begin{1=}
    \key{FOREGROUND-COLOR} \\
    \key{FOREGROUND-COLOUR}
  \end{1=}
  IS
  \begin{1=}
    \identifier \\
    \integer
  \end{1=}
\end{0-1}

\begin{0-1}
  WITH
  \begin{1=}
    \key{BACKGROUND-COLOR} \\
    \key{BACKGROUND-COLOUR}
  \end{1=}
  IS
  \begin{1=}
    \identifier \\
    \integer
  \end{1=}
\end{0-1}

\begin{0-1}
  WITH \key{SCROLL} \key{UP}
  \begin{0-1}
    \identifier \\
    \integer
  \end{0-1}
  \begin{1=}
    \key{LINE} \\
    \key{LINES}
  \end{1=}
\end{0-1}

\begin{0-1}
  WITH \key{SCROLL} \key{DOWN}
  \begin{0-1}
    \identifier \\
    \integer
  \end{0-1}
  \begin{1=}
    \key{LINE} \\
    \key{LINES}
  \end{1=}
\end{0-1}

\begin{0-1}
  WITH
  \begin{1=}
    \key{TIMEOUT} \\
    \key{TIME-OUT} \\
  \end{1=}
  AFTER
  \begin{0-1}
    \identifier \\
    \integer
  \end{0-1}
\end{0-1}

% Shouldn't all the clauses except AT LINE/COLUMN be miscext?
  
\format{temporal}
\key{ACCEPT} \identifier \key{FROM}
\begin{1=}
  \key{DATE}
  \begin{0-1}
    \key{YYYYMMDD}
  \end{0-1} \\

  \key{DAY}
  \begin{0-1}
    \key{YYYYDDD}
  \end{0-1} \\

  \key{DAY-OF-WEEK} \\
  \key{TIME} \\
\end{1=}

\format{environment}
\miscext{
  \begin{minipage}[!h]{1.0\linewidth}
    \key{ACCEPT} \identifier \key{FROM}
    \begin{1=}
      \key{ARGUMENT-NUMBER} \\

      \begin{1=}
        \key{COLUMNS} \\
        \key{COLS}
      \end{1=} \\

      \key{COMMAND-LINE} \\
      \key{ESCAPE} \key{KEY} \\
      \key{EXCEPTION} \key{STATUS} \\

      \begin{1=}
        \key{LINES} \\
        \key{LINE} \key{NUMBER}
      \end{1=} \\

      \key{USER} \key{NAME} \\
      \key{WORD}
    \end{1=}
  \end{minipage}
}


\format{environment-exception}
\miscext{
  \begin{minipage}[!h]{1.0\linewidth}
    \key{ACCEPT} \identifier \key{FROM}
    \begin{1=}
      \key{ARGUMENT-VALUE} \\
      \key{ENVIRONMENT}
      \begin{1=}
        \identifier \\
        \literal
      \end{1=} \\
      \key{ENVIRONMENT-VALUE} \\
    \end{1=}

    \begin{0+}
      ON
      \begin{1=}
        \key{EXCEPTION} \\
        \key{ESCAPE}
      \end{1=}
      \imperativestatement \\

      \key{NOT} ON
      \begin{1=}
        \key{EXCEPTION} \\
        \key{ESCAPE}
      \end{1=}
      \imperativestatement \\
    \end{0+}
  \end{minipage}
}
\section{ADD statement}

\format{simple}
\key{ADD}
\begin{1=}
  \identifier \\
  \literal \\
\end{1=} \ldots
\key{TO}
\begin{1=}
  \identifier
\end{1=} \ldots

\begin{0+}
  ON \key{SIZE} \key{ERROR} \imperativestatement \\
  \key{NOT} ON \key{SIZE} \key{ERROR} \imperativestatement
\end{0+}


\begin{0-1}
  \key{END-ADD}
\end{0-1}

\format{giving}
\key{ADD}
\begin{1=}
  \identifier \\
  \literal \\
\end{1=} \ldots
\begin{0-1}
  \key{TO}
  \begin{0-1}
    \identifier
  \end{0-1} \ldots
\end{0-1}

\key{GIVING}
\begin{1=}
  \identifier
  \begin{0-1}
    rounded-phrase
  \end{0-1}
\end{1=} \ldots

\begin{0+}
  ON \key{SIZE} \key{ERROR} \imperativestatement \\
  \key{NOT} ON \key{SIZE} \key{ERROR} \imperativestatement
\end{0+}

\begin{0-1}
  \key{END-ADD}
\end{0-1}

\format{corresponding}
\key{ADD}
\begin{1=}
  \key{CORRESPONDING} \\
  \key{CORR}
\end{1=}
\identifier \key{TO} \identifier
\begin{0-1}
  rounded-phrase
\end{0-1}

\begin{0+}
  ON \key{SIZE} \key{ERROR} \imperativestatement \\
  \key{NOT} ON \key{SIZE} \key{ERROR} \imperativestatement
\end{0+}

\begin{0-1}
  \key{END-ADD}
\end{0-1}


\section{ALLOCATE statement}

\key{ALLOCATE}
\begin{1=}
  \identifier
  \begin{0-1}
    \key{INITIALIZED}
  \end{0-1} \\
  \arithmeticexpression
  \begin{0-1}
    \key{INITIALIZED}
    \gnucobol{
      \begin{0-1}
        \key{TO}
        \begin{1=}
          \identifier \\
          \literal
        \end{1=}
      \end{0-1}
    }
  \end{0-1}
\end{1=}

\begin{0-1}
  \key{RETURNING} \identifier
\end{0-1} \\

\section{ALTER statement}

\deleted{
  \key{ALTER}
  \begin{1=}
    \procedurename TO PROCEED \key{TO} \procedurename
  \end{1=} \ldots
}

\section{CALL statement}

\key{CALL}
\miscext{
  \begin{0-1}
    \mnemonicname \\
    \key{STATIC} \\
    \key{STDCALL}
  \end{0-1}
}
\begin{1=}
  \identifier \\
  \literal \\
  \functionname
\end{1=}


\begin{0-1}
  \key{USING}
  \begin{1=}
    \begin{0-1}
      BY
      \begin{1=}
        \key{REFERENCE} \\
        \key{CONTENT} \\
        \key{VALUE}
      \end{1=}
    \end{0-1}
    \begin{1=}
      \key{OMITTED} \\

      \gnucobol{
        \begin{0-1}
          size-phrase
        \end{0-1}
      }
      \begin{1=}
        \identifier \\
        \literal
      \end{1=}
    \end{1=}
  \end{1=}\ldots
\end{0-1}

\begin{0-1}
  \begin{1=}
    \key{RETURNING} \\
    \miscext{\key{GIVING}}
  \end{1=}
  \begin{1=}
    INTO \identifier \\
    \key{ADDRESS} OF \identifier \\
    \gnucobol{\key{NOTHING}} \\
    \key{NULL} \\
    \key{OMITTED} \\
  \end{1=}
\end{0-1}

\begin{0+}
  ON
  \begin{1=}
    \key{EXCEPTION} \\
    \archaic{\key{OVERFLOW}}
  \end{1=}
  \imperativestatement \\
  \key{NOT} ON \key{EXCEPTION} \imperativestatement
\end{0+}

\begin{0-1}
  \key{END-CALL}
\end{0-1}

\section{CANCEL statement}

\key{CANCEL}
\begin{1=}
  \identifier \\
  \literal
\end{1=} \ldots

\section{CLOSE statement}

\key{CLOSE}
\begin{1=}
  \filename
  \begin{0-1}
    \begin{1=}
      \key{REEL} \\
      \key{UNIT}
    \end{1=}
    \begin{0-1}
      FOR \key{REMOVAL}
    \end{0-1} \\

    WITH \key{NO} \key{REWIND} \\
    WITH \key{LOCK}
  \end{0-1}
\end{1=} \ldots

\section{COMMIT statement}

\miscext{\key{COMMIT}}

\section{COMPUTE statement}

\key{COMPUTE}
\begin{1=}
  \identifier
  \begin{0-1}
    rounded-phrase
  \end{0-1}
\end{1=} \ldots
\begin{1=}
  = \\
  \miscext{\key{EQUAL}} \\
  \miscext{\key{EQUALS}}
\end{1=}
\arithmeticexpression

\begin{0+}
  ON \key{SIZE} \key{ERROR} \imperativestatement \\
  \key{NOT} ON \key{SIZE} \key{ERROR} \imperativestatement
\end{0+}

\begin{0-1}
  \key{END-COMPUTE}
\end{0-1}

\section{CONTINUE statement}

\key{CONTINUE}

\section{DELETE statement}

\format{record}
\key{DELETE} \filename RECORD

\begin{0+}
  \key{INVALID} KEY \imperativestatement \\
  \key{NOT} \key{INVALID} KEY \imperativestatement
\end{0+}

\begin{0-1}
  \key{END-DELETE}
\end{0-1}

\format{file}
\miscext{
  \key{DELETE} \key{FILE}
  \begin{1=}
    \filename
  \end{1=} \ldots
}
\gnucobol{\begin{0-1}
  \key{END-DELETE}
\end{0-1}}

\section{DISPLAY statement}

\format{device}
\key{DISPLAY}
\begin{1=}
  \identifier \\
  \literal
\end{1=} \ldots
\begin{0+}
  \key{UPON} \mnemonicname \\
  WITH \key{NO} \key{ADVANCING}
\end{0+}

\begin{0+}
  ON \key{EXCEPTION} \imperativestatement \\
  \key{NOT} ON \key{EXCEPTION} \imperativestatement \\
\end{0+}

\begin{0-1}
  \key{END-DISPLAY}
\end{0-1}

\format{environment}
\miscext{
  \begin{minipage}[!h]{1.0\linewidth}
    \key{DISPLAY}
    \begin{1=}
      \identifier \\
      \literal
    \end{1=}
    \key{UPON}
    \begin{1=}
      \key{ARGUMENT-NUMBER} \\
      \key{COMMAND-LINE} \\
      \key{ENVIRONMENT-NAME} \\
      \key{ENVIRONMENT-VALUE} \\
    \end{1=}

    \begin{0+}
      ON \key{EXCEPTION} \imperativestatement \\
      \key{NOT} ON \key{EXCEPTION} \imperativestatement \\
    \end{0+}

    \begin{0-1}
      \key{END-DISPLAY}
    \end{0-1}
  \end{minipage}
}

\format{screen}
\key{DISPLAY}
\begin{1=}
  \begin{1=}
    \identifier \\
    \miscext{\literal} \\
    \miscext{\key{OMITTED}}
  \end{1=}

  \miscext{
    \begin{0-1}
      \standard{position-clauses} \\

      \key{UPON}
      \begin{1=}
        \key{CRT} \\
        \key{CRT-UNDER}
      \end{1=} \\

      \key{MODE} IS \key{BLOCK} \\

      WITH
      \begin{1=}
        \key{BELL} \\
        \key{BEEP} \\
      \end{1=} \\

      WITH \key{BLANK}
      \begin{1=}
        \key{LINE} \\
        \key{SCREEN}
      \end{1=} \\

      WITH \key{BLINK} \\

      WITH \key{CONVERSION} \\

      WITH \key{ERASE}
      \begin{1=}
        \key{EOL} \\
        \key{EOS} \\

        \begin{0-1}
          \key{END} OF
        \end{0-1}
        \begin{1=}
          \key{LINE} \\
          \key{SCREEN}
        \end{1=}
      \end{1=} \\

      WITH
      \begin{1=}
        \key{HIGHLIGHT} \\
        \key{LOWLIGHT}
      \end{1=} \\

      WITH \key{OVERLINE} \\

      WITH \key{REVERSE-VIDEO} \\

      WITH \key{SIZE} IS
      \begin{1=}
        \identifier \\
        \literal
      \end{1=} \\

      WITH \key{UNDERLINE} \\

      WITH
      \begin{1=}
        \key{FOREGROUND-COLOR} \\
        \key{FOREGROUND-COLOUR}
      \end{1=}
      IS
      \begin{1=}
        \identifier \\
        \integer
      \end{1=} \\

      WITH
      \begin{1=}
        \key{BACKGROUND-COLOR} \\
        \key{BACKGROUND-COLOUR}
      \end{1=}
      IS
      \begin{1=}
        \identifier \\
        \integer
      \end{1=} \\

      WITH \key{SCROLL} \key{UP}
      \begin{0-1}
        \identifier \\
        \integer
      \end{0-1}
      \begin{1=}
        \key{LINE} \\
        \key{LINES}
      \end{1=} \\

      WITH \key{SCROLL} \key{DOWN}
      \begin{0-1}
        \identifier \\
        \integer
      \end{0-1}
      \begin{1=}
        \key{LINE} \\
        \key{LINES}
      \end{1=} \\
    \end{0-1}
  }
\end{1=} \miscext{\ldots}

\begin{0+}
  ON \key{EXCEPTION} \imperativestatement \\
  \key{NOT} ON \key{EXCEPTION} \imperativestatement \\
\end{0+}

\begin{0-1}
  \key{END-DISPLAY}
\end{0-1}

where position-clauses is

\begin{1=}
  \begin{1+}
    \key{LINE} NUMBER
    \begin{1=}
      \identifier \\
      \literal
    \end{1=} \\

    \key{AT}
    \begin{1=}
      \key{COLUMN} \\
      \key{COL} \\
      \miscext{\key{POSITION}}
    \end{1=}

    \begin{1=}
      \identifier \\
      \literal
    \end{1=}
  \end{1+} \\


  \miscext{
    \key{AT}
    \begin{1=}
      \identifier \\
      \literal
    \end{1=}
  }
\end{1=}

\section{DIVIDE statement}

\format{into}
\key{DIVIDE}
\begin{1=}
  \identifier \\
  \literal
\end{1=}
\key{INTO}
\begin{1=}
  \begin{1=}
    \identifier \\
    \literal
  \end{1=}
  \begin{0-1}
    rounded-phrase
  \end{0-1}
\end{1=} \ldots

\begin{0+}
  ON \key{SIZE} \key{ERROR} \imperativestatement \\
  \key{NOT} ON \key{SIZE} \key{ERROR} \imperativestatement
\end{0+}

\begin{0-1}
  \key{END-DIVIDE}
\end{0-1}

\format{giving}
\key{DIVIDE}
\begin{1=}
  \identifier \\
  \literal
\end{1=}
\begin{1=}
  \key{BY} \\
  \key{INTO}
\end{1=}
\begin{1=}
  \identifier \\
  \literal
\end{1=}

\key{GIVING}
\begin{1=}
  \begin{1=}
    \identifier \\
    \literal
  \end{1=}
  \begin{0-1}
    rounded-phrase
  \end{0-1}
\end{1=}
\ldots

\begin{0-1}
  \key{REMAINDER}
  \begin{1=}
    \identifier \\
    \literal
  \end{1=}
\end{0-1}

\begin{0+}
  ON \key{SIZE} \key{ERROR} \imperativestatement \\
  \key{NOT} ON \key{SIZE} \key{ERROR} \imperativestatement
\end{0+}

\begin{0-1}
  \key{END-DIVIDE}
\end{0-1}

\section{ENTRY statement}

\miscext{
  \begin{minipage}[!h]{1.0\linewidth}
    \key{ENTRY} \literal

    \begin{0-1}
      \key{USING}

      \begin{1=}
        \begin{0-1}
          BY
          \begin{1=}
            \key{REFERENCE} \\
            \gnucobol{\key{CONTENT}} \\
            \key{VALUE}
          \end{1=}
        \end{0-1}

        \begin{1=}
          \gnucobol{\key{OMITTED}} \\

          \begin{1=}
            \gnucobol{
              \begin{0-1}
                size-phrase
              \end{0-1}
            }
            \begin{1=}
              \identifier \\
              \literal
            \end{1=}
          \end{1=}
        \end{1=}\ldots
      \end{1=}
    \end{0-1}
  \end{minipage}
}

\section{EVALUATE statement}

\key{EVALUATE}
\begin{1=}
  \expression \\
  \key{TRUE} \\
  \key{FALSE}
\end{1=}
\begin{0-1}
  \key{ALSO}
  \begin{1=}
    \expression \\
    \key{TRUE} \\
    \key{FALSE}
  \end{1=}
\end{0-1} \ldots

\begin{1=}
  \key{WHEN}
  selection-object
  \begin{0-1}
    \key{ALSO} selection-object
  \end{0-1}\ldots\ {}
  \imperativestatement
\end{1=} \ldots

\begin{0-1}
  \key{WHEN} \key{OTHER} \imperativestatement
\end{0-1}

\begin{0-1}
  \key{END-EVALUATE}
\end{0-1}

where selection-object is

\begin{1=}
  partial-\expression
  \begin{0-1}
    \begin{1=}
      \key{THROUGH} \\
      \key{THRU}
    \end{1=}
    \expression
  \end{0-1} \\

  \key{ANY} \\
  \key{TRUE} \\
  \key{FALSE}
\end{1=}

\section{EXIT statement}

\key{EXIT}
\begin{0-1}
  \key{FUNCTION} \\
  \key{PARAGRAPH} \\

  \key{PERFORM}
  \begin{0-1}
    \key{CYCLE}
  \end{0-1} \\

  \key{PROGRAM}
  \miscext{
    \begin{0-1}
      \begin{1=}
        \key{RETURNING} \\
        \key{GIVING}
      \end{1=}
    \end{0-1}
    \begin{1=}
      \identifier \\
      \literal
    \end{1=}
  } \\

  \key{SECTION} \\
\end{0-1}

\section{FREE statement}

\key{FREE}
\begin{1=}
  \identifier
\end{1=} \ldots

\section{GENERATE statement}

\pending{\key{GENERATE} \reportname}

\section{GO TO statement}

\key{GO} TO
\begin{1=}
  \procedurename
\end{1=} \ldots
\begin{0-1}
  \key{DEPENDING} ON \identifier
\end{0-1}

\section{GOBACK statement}

\key{GOBACK}
\begin{0-1}
  \begin{1=}
    \key{RETURNING} \\
    \miscext{\key{GIVING}}
  \end{1=}
  \begin{1=}
    \identifier \\
    \literal
  \end{1=}
\end{0-1}

\section{IF statement}

\key{IF} condition THEN
\begin{1=}
  \imperativestatement \\
  \key{ELSE} \imperativestatement
\end{1=} \ldots

\begin{0-1}
  \key{END-IF}
\end{0-1}

\section{INITIALIZE statement}

\begin{1=}
  \key{INITIALIZE} \\
  \miscext{\key{INITIALISE}}
\end{1=}
\begin{1=}
  \identifier \\
  basic-\literal
\end{1=} \ldots
\begin{0-1}
  WITH \key{FILLER}
\end{0-1}

\begin{0-1}
  \begin{1=}
    \key{ALL} \\
    \key{ALPHABETIC} \\
    \key{ALPHANUMERIC} \\
    \key{ALPHANUMERIC-EDITED} \\
    \key{NATIONAL} \\
    \key{NATIONAL-EDITED} \\
    \key{NUMERIC} \\
    \key{NUMERIC-EDITED}
  \end{1=}
  TO \key{VALUE}
\end{0-1}

\begin{0-1}
  \key{REPLACING}
  \begin{1=}
    \begin{1=}
      \key{ALPHABETIC} \\
      \key{ALPHANUMERIC} \\
      \key{ALPHANUMERIC-EDITED} \\
      \key{NATIONAL} \\
      \key{NATIONAL-EDITED} \\
      \key{NUMERIC} \\
      \key{NUMERIC-EDITED}
    \end{1=}
    DATA \key{BY}
    \begin{1=}
      \identifier \\
      \literal
    \end{1=}
  \end{1=} \ldots
\end{0-1}

\begin{0-1}
  THEN TO \key{DEFAULT}
\end{0-1}

\section{INITIATE statement}
\pending{
  \key{INITIATE}
  \begin{1=}
    \reportname
  \end{1=} \ldots
}

\section{INSPECT statement}

\key{INSPECT}
\begin{1=}
  \identifier \\
  \literal \\
  \functionname
\end{1=}
\begin{1=}
  tallying-phrase
  \begin{0-1}
    replacing-phrase
  \end{0-1} \\

  replacing-phrase \\
  converting-phrase
\end{1=}

where tallying-phrase is

\key{TALLYING}
\begin{1=}
  \begin{1=}
    \begin{1=}
      \identifier \\
      \literal
    \end{1=}
    \key{FOR}
    \begin{1=}
      \key{CHARACTERS} \\

      \begin{1=}
        \key{ALL} \\
        \key{LEADING} \\
        \key{TRAILING}
      \end{1=}
      \begin{1=}
        \identifier \\
        \literal
      \end{1=}
    \end{1=}
  \end{1=}\ldots
  \begin{0-1}
    before-after-phrase
  \end{0-1}
\end{1=} \ldots

where replacing-phrase is

\key{REPLACING}
\begin{1=}
  \begin{1=}
    \key{CHARACTERS} \\

    \begin{0-1}
      \key{ALL} \\
      \key{LEADING} \\
      \key{FIRST} \\
      \key{TRAILING}
    \end{0-1}
    \begin{1=}
      \identifier \\
      \literal
    \end{1=}
  \end{1=}
  \key{BY}
  \begin{1=}
    \identifier \\
    \literal \\
  \end{1=}
  \begin{0-1}
    before-after-phrase
  \end{0-1} \\
\end{1=} \ldots

where converting-phrase is

\key{CONVERTING}
\begin{1=}
  \identifier \\
  \literal
\end{1=}
\key{TO}
\begin{1=}
  \identifier \\
  \literal
\end{1=}
\begin{0-1}
  before-after-phrase
\end{0-1}

where before-after-phrase is

\begin{0+}
  \key{BEFORE} INITIAL
  \begin{1=}
    \identifier \\
    \literal
  \end{1=} \\

  \key{AFTER} INITIAL
  \begin{1=}
    \identifier \\
    \literal
  \end{1=}
\end{0+}

\section{MERGE statement}

\key{MERGE} \identifier
\begin{0-1}
  ON
  \begin{1=}
    \key{ASCENDING} \\
    \key{DESCENDING}
  \end{1=}
  KEY
  \begin{0-1}
    \identifier
  \end{0-1}\ldots
\end{0-1} \ldots

\begin{0-1}
  WITH \key{DUPLICATES}
  \begin{0-1}
    IN \key{ORDER}
  \end{0-1}
\end{0-1}

\begin{0-1}
  COLLATING \key{SEQUENCE} IS \identifier
\end{0-1}

\begin{0-1}
  \key{USING}
  \begin{1=}
    \filename
  \end{1=}\ldots
\end{0-1}

\begin{0-1}
  \key{GIVING}
  \begin{1=}
    \filename
  \end{1=}\ldots \\

  \key{OUTPUT} \key{PROCEDURE} IS
  \procedurename
  \begin{0-1}
    \begin{1=}
      \key{THROUGH} \\
      \key{THRU}
    \end{1=}
    \procedurename
  \end{0-1}
\end{0-1}

\section{MOVE statement}

\key{MOVE}
\begin{0-1}
  \key{CORRESPONDING} \\
  \key{CORR}
\end{0-1}
\begin{1=}
  \identifier \\
  \literal
\end{1=}
\key{TO}
\begin{1=}
  \identifier
\end{1=} \ldots

\section{MULTIPLY statement}

\format{simple}
\key{MULTIPLY}
\begin{1=}
  \identifier \\
  \literal
\end{1=}
\key{BY}
\begin{1=}
  \begin{1=}
    \identifier \\
    \literal
  \end{1=}
  \begin{0-1}
    rounded-phrase
  \end{0-1}
\end{1=} \ldots

\begin{0+}
  ON \key{SIZE} \key{ERROR} \imperativestatement \\
  \key{NOT} ON \key{SIZE} \key{ERROR} \imperativestatement
\end{0+}

\begin{0-1}
  \key{END-MULTIPLY}
\end{0-1}

\format{giving}
\key{MULTIPLY}
\begin{1=}
  \identifier \\
  \literal
\end{1=}
\key{BY}
\begin{1=}
  \identifier \\
  \literal
\end{1=}

\key{GIVING}
\begin{1=}
  \begin{1=}
    \identifier \\
    \literal
  \end{1=}
  \begin{0-1}
    rounded-phrase
  \end{0-1}
\end{1=} \ldots

\begin{0+}
  ON \key{SIZE} \key{ERROR} \imperativestatement \\
  \key{NOT} ON \key{SIZE} \key{ERROR} \imperativestatement
\end{0+}

\begin{0-1}
  \key{END-MULTIPLY}
\end{0-1}


\section{NEXT SENTENCE statement}

\archaic{
  \key{NEXT} \key{SENTENCE}
}

\section{OPEN statement}

\key{OPEN}
\begin{1=}
  \begin{1=}
    \key{INPUT} \\
    \key{OUTPUT} \\
    \key{I-O} \\
    \key{EXTEND}
  \end{1=}
  \begin{0-1}
    \key{SHARING} WITH
    \begin{1=}
      \key{ALL} OTHER \\
      \key{NO} OTHER \\
      \key{READ} \key{ONLY}
    \end{1=}
  \end{0-1}
  \begin{1=}
    \filename
  \end{1=} \ldots
  \begin{0-1}
    WITH \key{NO} \key{REWIND} \\
    WITH \key{LOCK} \\
    \deleted{\key{REVERSED}}
  \end{0-1}
\end{1=}\ldots


\section{PERFORM statement}

\format{procedure}
\key{PERFORM} \procedurename
\begin{0-1}
  \begin{1=}
    \key{THROUGH} \\
    \key{THRU}
  \end{1=}
  \procedurename
\end{0-1}
\begin{0-1}
  \gnucobol{\key{FOREVER}} \\
  times-phrase \\
  until-phrase \\
  varying-phrase
\end{0-1}

\format{inline}
\key{PERFORM}
\begin{0-1}
  \gnucobol{\key{FOREVER}} \\
  times-phrase \\
  until-phrase \\
  varying-phrase
\end{0-1}
\imperativestatement
\begin{0-1}
  \key{END-PERFORM}
\end{0-1}

where times-phrase is

\begin{1=}
  \identifier \\
  \literal \\
  \functionname
\end{1=}
\key{TIMES} \\

where until-phrase is

\begin{0-1}
  WITH \key{TEST}
  \begin{1=}
    \key{BEFORE} \\
    \key{AFTER} \\
  \end{1=}
\end{0-1}
\key{UNTIL}
\begin{1=}
  \condition \\
  \gnucobol{\key{EXIT}}
\end{1=} \\

and where varying-phrase is

\begin{0-1}
  WITH \key{TEST}
  \begin{1=}
    \key{BEFORE} \\
    \key{AFTER} \\
  \end{1=}
\end{0-1}

\key{VARYING} \identifier \key{FROM}
\begin{1=}
  \identifier \\
  \literal
\end{1=}
\key{BY}
\begin{1=}
  \identifier \\
  \literal
\end{1=}
\key{UNTIL}
\condition

\begin{0-1}
  \key{AFTER} \identifier \key{FROM}
  \begin{1=}
    \identifier \\
    \literal
  \end{1=}
  \key{BY}
  \begin{1=}
    \identifier \\
    \literal
  \end{1=}

  \key{UNTIL}
  \condition
\end{0-1} \ldots


% \begin{0-1}
%   \key{\begin{0-1}
%   \key{END-PERFORM}
% \end{0-1}} \\
%   .
% \end{0-1}

% TO-DO: Improve

\section{READ statement}

\key{READ} \filename
\begin{0-1}
  \key{NEXT} \\
  \key{PREVIOUS}
\end{0-1}
RECORD
\begin{0-1}
  \key{INTO} \identifier
\end{0-1}

\begin{0-1}
  \key{IGNORING} \key{LOCK} \\

  WITH
  \begin{1=}
    \key{KEPT} \\
    \key{NO} \\
    \key{IGNORE} \\
  \end{1=}
  \key{LOCK} \\

  WITH \key{WAIT}
\end{0-1}

\begin{0-1}
  \key{KEY} IS \identifier
\end{0-1}

\begin{0-1}
  \begin{1+}
    \key{INVALID} \key{KEY} \imperativestatement \\
    \key{NOT} \key{INVALID} \key{KEY} \imperativestatement
  \end{1+} \\

  \begin{1+}
    AT \key{END} \imperativestatement \\
    \key{NOT} AT \key{END} \imperativestatement
  \end{1+}
\end{0-1}

\begin{0-1}
  \key{END-READ}
\end{0-1}

\section{READY statement}

\miscext{\key{READY} \key{TRACE}}

\section{RELEASE statement}

\key{RELEASE} \identifier
\begin{0-1}
  \key{FROM}
  \begin{1=}
    \identifier \\
    \literal \\
    function-call-1
  \end{1=}
\end{0-1}

\section{RESET statement}

\miscext{\key{RESET} \key{TRACE}}

\section{RETURN statement}

\key{RETURN} \filename RECORD
\begin{0-1}
  \key{INTO} \identifier
\end{0-1}

AT \key{END} \imperativestatement

\begin{0-1}
  \key{NOT} AT \key{END} \imperativestatement
\end{0-1}

\begin{0-1}
  \key{END-RETURN}
\end{0-1}

\section{REWRITE statement}

\key{REWRITE}
\recordname
\begin{0-1}
  \key{FROM}
  \begin{1=}
    \identifier \\
    \literal \\
    \functionname
  \end{1=}
\end{0-1}
\begin{0-1}
  WITH
  \begin{0-1}
    \key{NO}
  \end{0-1}
  \key{LOCK}
\end{0-1}

\begin{0+}
  \key{INVALID} \key{KEY} \imperativestatement \\
  \key{NOT} \key{INVALID} \key{KEY} \imperativestatement
\end{0+} \\

\begin{0-1}
  \key{END-REWRITE}
\end{0-1}

\section{ROLLBACK statement}

\miscext{\key{ROLLBACK}}

\section{SEARCH statement}

\format{simple}
\key{SEARCH} \identifier
\begin{0-1}
  \key{VARYING} \identifier
\end{0-1}

\begin{0-1}
  AT \key{END} \imperativestatement
\end{0-1}

\begin{1=}
  \key{WHEN} \condition \imperativestatement
\end{1=} \ldots

\begin{0-1}
  \key{END-SEARCH}
\end{0-1}

\format{all}
\key{SEARCH} \key{ALL} \identifier

\begin{0-1}
  AT \key{END} \imperativestatement
\end{0-1}

\key{WHEN} \expression \imperativestatement

\begin{0-1}
  \key{END-SEARCH}
\end{0-1}


\section{SET statement}

\format{simple}
\key{SET} \identifier \key{TO}
\begin{1=}
  \identifier \\
  \literal \\
  \arithmeticexpression
\end{1=}

\format{entry}
\gnucobol{
  \key{SET} \identifier \key{TO} \key{ENTRY}
  \begin{1=}
    \identifier \\
    \literal
  \end{1=}
}

\format{environment}
\miscext{
  \key{SET} \key{ENVIRONMENT}
  \begin{1=}
    \identifier \\
    \literal
  \end{1=}
  \key{TO}
  \begin{1=}
    \identifier \\
    \literal
  \end{1=}
}

\format{attribute}
\key{SET} \identifier \key{ATTRIBUTE}
\begin{1=}
  \begin{1=}
    \begin{1=}
      \key{BELL} \\
      \key{BEEP}
    \end{1=} \\

    \key{BLINK} \\
    \key{HIGHLIGHT} \\
    \key{LOWLIGHT} \\
    \key{REVERSE-VIDEO} \\
    \key{UNDERLINE} \\
    \key{LEFTLINE} \\
    \key{OVERLINE}
  \end{1=}
  \begin{1=}
    \key{ON} \\
    \key{OFF}
  \end{1=}
\end{1=}\ldots

\format{arithmetic}
\key{SET}
\begin{1=}
  \cobolindexname
\end{1=}\ldots
\begin{1=}
  \key{UP} \\
  \key{DOWN}
\end{1=}
\key{BY}
\arithmeticexpression

\format{on\slash{}off}
\key{SET}
\begin{1=}
  \begin{1=}
    \mnemonicname
  \end{1=}\ldots
  \key{TO}
  \begin{1=}
    \key{ON} \\
    \key{OFF}
  \end{1=}
\end{1=} \ldots

\format{true\slash{}false}
\key{SET}
\begin{1=}
  \begin{1=}
    \conditionname
  \end{1=}\ldots
  \key{TO}
  \begin{1=}
    \key{TRUE} \\
    \key{FALSE}
  \end{1=}
\end{1=} \ldots

\format{exception}
\key{SET} \key{LAST} \key{EXCEPTION} \key{TO} \key{OFF}

\section{SORT statement}

\key{SORT} \identifier
\begin{0-1}
  ON
  \begin{1=}
    \key{ASCENDING} \\
    \key{DESCENDING}
  \end{1=}
  KEY
  \begin{0-1}
    \identifier
  \end{0-1}\ldots
\end{0-1} \ldots

\begin{0-1}
  WITH \key{DUPLICATES}
  \begin{0-1}
    IN \key{ORDER}
  \end{0-1}
\end{0-1}

\begin{0-1}
  COLLATING \key{SEQUENCE} IS \identifier
\end{0-1}

\begin{0-1}
  \key{USING}
  \begin{1=}
    \filename
  \end{1=}\ldots \\

  \key{INPUT} \key{PROCEDURE} IS
  \procedurename
  \begin{0-1}
    \begin{1=}
      \key{THROUGH} \\
      \key{THRU}
    \end{1=}
    \procedurename
  \end{0-1}
\end{0-1}

\begin{0-1}
  \key{GIVING}
  \begin{1=}
    \filename
  \end{1=}\ldots \\

  \key{OUTPUT} \key{PROCEDURE} IS
  \procedurename
  \begin{0-1}
    \begin{1=}
      \key{THROUGH} \\
      \key{THRU}
    \end{1=}
    \procedurename
  \end{0-1}
\end{0-1}

\section{START statement}

\key{START} \filename
\begin{0-1}
  \key{FIRST} \\

  \key{KEY} IS
  relational-operator
  \identifier\\

  \key{LAST}
\end{0-1}

\begin{0-1}
  WITH
  \begin{1=}
    \key{SIZE} \\
    \gnucobol{\key{LENGTH}}
  \end{1=}
  \arithmeticexpression
\end{0-1}

\begin{0+}
  \key{INVALID} KEY \imperativestatement \\
  \key{NOT} \key{INVALID} KEY \imperativestatement
\end{0+}

\begin{0-1}
  \key{END-START}
\end{0-1}

\section{STOP statement}

\format{standard}
\ {}\newline
\key{STOP} \key{RUN}
\begin{0-1}
  \begin{1=}
    \key{RETURNING} \\
    \miscext{\key{GIVING}}
  \end{1=}
  \begin{1=}
    \identifier \\
    \literal
  \end{1=} \\

  WITH
  \begin{1=}
    \key{ERROR} \\
    \key{NORMAL}
  \end{1=}
  STATUS
  \begin{0-1}
    \identifier \\
    \literal
  \end{0-1}
\end{0-1}

\format{literal}
\ {}\newline
\deleted{\key{STOP} \literal}

\format{ACUCOBOL}
\ {}\newline
\miscext{
  \key{STOP} \key{RUN}
  \begin{1=}
    \identifier \\
    \literal
  \end{1=}
}

\section{STRING statement}

\key{STRING}
\begin{1=}
  \begin{1=}
    \identifier \\
    \literal
  \end{1=}

  \begin{0-1}
    \key{DELIMITED} BY
    \begin{1=}
      \key{SIZE} \\
      \identifier \\
      \literal
    \end{1=}
  \end{0-1}
\end{1=} \ldots\ {}
\key{INTO} \identifier

\begin{0-1}
  WITH \key{POINTER} IS \identifier
\end{0-1}

\begin{0+}
  ON \key{OVERFLOW} \imperativestatement \\
  \key{NOT} ON \key{OVERFLOW} \imperativestatement
\end{0+}

\section{SUBTRACT statement}

\format{simple}
\key{SUBTRACT}
\begin{1=}
  \identifier \\
  \literal
\end{1=} \ldots
\key{FROM}
\begin{1=}
  \begin{1=}
    \identifier \\
    \literal
  \end{1=}
  \begin{0-1}
    rounded-phrase
  \end{0-1}
\end{1=} \ldots

\begin{0+}
  ON \key{SIZE} \key{ERROR} \imperativestatement \\
  \key{NOT} ON \key{SIZE} \key{ERROR} \imperativestatement
\end{0+}

\begin{0-1}
  \key{END-SUBTRACT}
\end{0-1}

\format{giving}
\key{SUBTRACT}
\begin{1=}
  \identifier \\
  \literal
\end{1=} \ldots
\key{FROM}
\begin{1=}
  \identifier \\
  \literal
\end{1=}

\key{GIVING}
\begin{1=}
  \begin{1=}
    \identifier \\
    \literal
  \end{1=}
  \begin{0-1}
    rounded-phrase
  \end{0-1}
\end{1=} \ldots

\begin{0+}
  ON \key{SIZE} \key{ERROR} \imperativestatement \\
  \key{NOT} ON \key{SIZE} \key{ERROR} \imperativestatement
\end{0+}

\begin{0-1}
  \key{END-SUBTRACT}
\end{0-1}

\format{corresponding}
\key{SUBTRACT}
\begin{1=}
  \key{CORR} \\
  \key{CORRESPONDING}
\end{1=}
\identifier{} \key{FROM} \identifier
\begin{0-1}
  rounded-phrase
\end{0-1}

\begin{0+}
  ON \key{SIZE} \key{ERROR} \imperativestatement \\
  \key{NOT} ON \key{SIZE} \key{ERROR} \imperativestatement
\end{0+}

\begin{0-1}
  \key{END-SUBTRACT}
\end{0-1}

\section{SUPPRESS statement}

\pending{
  \key{SUPPRESS} PRINTING
}

\section{TERMINATE statement}

\pending{
  \key{TERMINATE}
  \begin{1=}
    \reportname
  \end{1=} \ldots
}

\section{TRANSFORM statement}

\deleted{
  \key{TRANSFORM} \identifier \key{FROM}
  \begin{1=}
    \identifier \\
    \literal
  \end{1=}
  \key{TO}
  \begin{1=}
    \identifier \\
    \literal
  \end{1=}
}

\section{UNLOCK statement}

\key{UNLOCK} \filename
\begin{0-1}
  \key{RECORD} \\
  \key{RECORDS}
\end{0-1}

\section{UNSTRING statement}

\key{UNSTRING} \identifier

\begin{0-1}
  \key{DELIMITED} BY
  \begin{0-1}
    \key{ALL}
  \end{0-1}
  \begin{1=}
    \identifier \\
    \literal
  \end{1=}
  \begin{1=}
    \key{OR}
    \begin{0-1}
      \key{ALL}
    \end{0-1}
    \begin{1=}
      \identifier \\
      \literal
    \end{1=}
  \end{1=} \ldots
\end{0-1}

\key{INTO}
\begin{1=}
  \identifier
  \begin{0-1}
    \key{DELIMITER} IN \identifier
  \end{0-1}
  \begin{0-1}
    \key{COUNT} IN \identifier
  \end{0-1}
\end{1=} \ldots

\begin{0-1}
  WITH \key{POINTER} IS \identifier
\end{0-1}

\begin{0-1}
  \key{TALLYING} IN \identifier
\end{0-1}

\begin{0+}
  ON \key{OVERFLOW} \imperativestatement \\
  \key{NOT} ON \key{OVERFLOW} \imperativestatement
\end{0+}

\begin{0-1}
  \key{END-OVERFLOW}
\end{0-1}

\section{USE statement}

\format{file exception}
\key{USE}
\begin{0-1}
  \key{GLOBAL}
\end{0-1}
AFTER STANDARD
\begin{1=}
  \key{EXCEPTION} \\
  \key{ERROR}
\end{1=}
PROCEDURE ON

\begin{1=}
  \begin{1=}
    \filename
  \end{1=} \ldots
  \begin{0+}
    \key{INPUT} \\
    \key{OUTPUT} \\
    \key{I-O} \\
    \key{EXTEND}
  \end{0+} \ldots
\end{1=}

\format{debugging}
\deleted{
  \key{USE} FOR \key{DEBUGGING} ON
  \begin{1=}
    \procedurename \\
    \key{ALL} \key{PROCEDURES} \\
    \key{ALL} REFERENCES OF \identifier
  \end{1=} \ldots
}

\format{start\slash{}end}
\miscext{\pending{
  \key{USE} AT \key{PROGRAM}
  \begin{1=}
    \key{START} \\
    \key{END}
  \end{1=}
}}

\format{reporting}
\key{USE}
\begin{0-1}
  \key{GLOBAL}
\end{0-1}
\key{BEFORE} \key{REPORTING} \identifier

\format{exception}
\pending{
  \key{USE}
  \begin{1=}
    \key{EXCEPTION-CONDITION} \\
    \key{EC}
  \end{1=}
}

\section{WRITE statement}

\format{sequential}
\key{WRITE} \recordname
\begin{0-1}
  \key{FROM}
  \begin{1=}
    \identifier \\
    \literal \\
    \functionname
  \end{1=}
\end{0-1}

\begin{0-1}
  \begin{1=}
    \key{BEFORE} \\
    \key{AFTER}
  \end{1=}
  ADVANCING
  \begin{1=}
    \begin{1=}
      \identifier \\
      \literal
    \end{1=}
    \begin{0-1}
      \key{LINE} \\
      \key{LINES}
    \end{0-1} \\

    \mnemonicname \\

    \key{PAGE}
  \end{1=}
\end{0-1}

\begin{0-1}
  WITH
  \begin{0-1}
    \key{NO}
  \end{0-1}
  \key{LOCK}
\end{0-1}

\begin{0+}
  AT
  \begin{1=}
    \key{END-OF-PAGE} \\
    \key{EOP}
  \end{1=}
  \imperativestatement \\

  \key{NOT} AT
  \begin{1=}
    \key{END-OF-PAGE} \\
    \key{EOP}
  \end{1=}
  \imperativestatement
\end{0+}

\begin{0-1}
  \key{END-WRITE}
\end{0-1}

\format{random}
\key{WRITE} \recordname
\begin{0-1}
  \key{FROM}
  \begin{1=}
    \identifier \\
    \literal \\
    \functionname
  \end{1=}
\end{0-1}

\begin{0-1}
  \begin{1=}
    \key{BEFORE} \\
    \key{AFTER}
  \end{1=}
  ADVANCING
  \begin{1=}
    \begin{1=}
      \identifier \\
      \literal
    \end{1=}
    \begin{0-1}
      \key{LINE} \\
      \key{LINES}
    \end{0-1} \\

    \mnemonicname \\

    \key{PAGE}
  \end{1=}
\end{0-1}

\begin{0-1}
  WITH
  \begin{0-1}
    \key{NO}
  \end{0-1}
  \key{LOCK}
\end{0-1}

\begin{0+}
  \key{INVALID} \key{KEY} \imperativestatement \\
  \key{NOT} \key{INVALID} \key{KEY} \imperativestatement
\end{0+} \\

\begin{0-1}
  \key{END-WRITE}
\end{0-1}

%%% Local Variables:
%%% mode: latex
%%% TeX-master: "grammar.tex"
%%% End:
