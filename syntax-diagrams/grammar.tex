\documentclass[a4paper,oneside,svgnames]{scrbook}

\usepackage{microtype}
\usepackage[toc,page]{appendix}
\usepackage{booktabs}
\usepackage{textcomp}

% for arithmetic expression table
\usepackage{pifont}
\newcommand{\tick}{\ding{51}}
\usepackage{multirow}

% Erewhon is for math mode.
\usepackage[proportional,scaled=1.064]{erewhon}
\usepackage[erewhon,vvarbb,bigdelims]{newtxmath}
\usepackage{roboto}
\usepackage[T1]{fontenc}
\renewcommand*\oldstylenums[1]{\textosf{#1}}

\usepackage[dvipsnames]{xcolor}
\usepackage{afterpage}
\usepackage{tikz}
\usepackage[printwatermark]{xwatermark}
\usepackage[tocindentauto]{tocstyle}
\usetocstyle{standard}
\usepackage[most]{tcolorbox} % for coloured boxes and warning boxes
\usepackage[pdfauthor={Edward Hart},
            pdftitle={The GnuCOBOL 2.2 Grammar},
            pdfkeywords={GnuCOBOL,COBOL,manual,grammar,guide},
            colorlinks]{hyperref}

\newcommand{\clearpageifnotfirst}[1]{%
  \ifnum\value{#1}>0 \clearpage{} \fi%
}

\newtcolorbox{streak}[1][]{frame code={}
  halign title=flush center,
  left=40mm,
  right=40mm,
  top=10pt,
  bottom=10pt,
  colback=Maroon,
  colupper=white,
  left skip=40mm,
  grow to left by=40mm,
  right skip=40mm,
  grow to right by=40mm,
  width=\paperwidth,
  text width=\textwidth,
  arc=0pt,outer arc=0pt,
  #1
}

% \newtcolorbox{syntaxbox}[1][]{enhanced,
%   before skip=2mm,after skip=3mm,
%   boxrule=0.4pt,left=5mm,right=2mm,top=1mm,bottom=1mm,
%   colback=yellow!50,
%   colframe=yellow!20!black,
%   sharp corners,rounded corners=southeast,arc is angular,arc=3mm,
%   underlay={%
%     \path[fill=tcbcol@back!80!black] ([yshift=3mm]interior.south east)--++(-0.4,-0.1)--++(0.1,-0.2);
%     \path[draw=tcbcol@frame,shorten <=-0.05mm,shorten >=-0.05mm] ([yshift=3mm]interior.south east)--++(-0.4,-0.1)--++(0.1,-0.2);
%     \path[fill=yellow!50!black,draw=none] (interior.south west) rectangle node[white]{\Huge\bfseries !} ([xshift=4mm]interior.north west);
%     },
%   drop fuzzy shadow}

\newtcolorbox{syntaxbox}[1][]{enhanced,
  boxrule=0.4pt,left=1mm,right=1mm,top=1mm,bottom=1mm,%
  sharp corners,% Rectangular shape
  halign=flush left,% To prevent spread out lines
  parbox=false,% To give normal paragraph indentation.
  colback=#1,% To allow different background colours.
  before upper={\parskip3pt\parindent15pt}%
}

\renewcommand{\arraystretch}{1.2}
\def\defaultparskip{\parskip}

\newenvironment{syntax}[1][black!5!white]{% black!5!white is the tcolorbox default
  \begin{syntaxbox}[#1]\small\selectfont\setlength{\parskip}{\baselineskip}}{%
  \setlength{\parskip}{\defaultparskip}\normalsize\selectfont\end{syntaxbox}}

\renewcommand{\familydefault}{\sfdefault}

\newcommand{\metaelement}[1]{%
  \textit{#1}%
}

\newcommand{\newlexicalelement}[1]{%
  \newcounter{#1}[subsection]%
  \expandafter\newcommand\csname #1\endcsname{%
    \stepcounter{#1}%
    \metaelement{#1-\arabic{#1}}\ {}%
  }%
}

\newcommand{\newlexicalelementwithname}[2]{%
  \newcounter{#1}[subsection]%
  \expandafter\newcommand\csname #1\endcsname{%
    \stepcounter{#1}%
    \metaelement{#2-\arabic{#1}}\ {}%
  }%
}

\newcounter{format}[subsection]
\newcommand{\format}[1]{%
  \stepcounter{format}%
  \paragraph{Format \arabic{format}\vspace{1em}\ {}(#1)}\ {}\newline%
}

\newcommand{\key}[1]{\underline{#1}}
\newcommand{\directiveindicator}[0]{>{}>}

\newcommand{\deletedcolour}{red!75}
\newcommand{\deleted}[1]{%
  \colorbox{\deletedcolour}{#1}}

\newcommand{\archaiccolour}{pink}
\newcommand{\archaic}[1]{%
  \colorbox{\archaiccolour}{#1}}

\newcommand{\obsoletecolour}{red!60}
\newcommand{\obsolete}[1]{%
  \colorbox{\obsoletecolour}{#1}}

\newcommand{\xopencolour}{green!75}
\newcommand{\xopen}[1]{%
  \colorbox{\xopencolour}{#1}}

\newcommand{\gnucobolcolour}{orange!75}
\newcommand{\gnucobol}[1]{%
  \colorbox{\gnucobolcolour}{#1}}

\newcommand{\miscextcolour}{blue!50}
\newcommand{\miscext}[1]{%
  \colorbox{\miscextcolour}{#1}}

\newcommand{\pending}[1]{%
  \textcolor{gray!75}{#1}}

\newcommand{\standard}[1]{%
  \colorbox{white}{#1}}

\newenvironment{0-1}{$\left[ \begin{tabular}{@{}l@{}}}{\end{tabular} \right]$}
\newenvironment{0+}{$\left[\left| \begin{tabular}{@{}l@{}}}{\end{tabular} \right|\right]$}
\newenvironment{1=}{$\left\{ \begin{tabular}{@{}l@{}}}{\end{tabular} \right\}$}
\newenvironment{1+}{$\left\{\left| \begin{tabular}{@{}l@{}}}{\end{tabular} \right|\right\}$}

\begin{document}

\newsavebox\mybox
\savebox\mybox{\tikz[color=gray,opacity=0.3]\node{DRAFT};}
% \newwatermark*[
%   allpages,
%   angle=45,
%   scale=12,
%   xpos=-42,
%   ypos=39
% ]{\usebox\mybox}

\newlexicalelementwithname{arithmeticexpression}{arithmetic-expression}
\newlexicalelementwithname{conditionname}{condition-name}
\newlexicalelementwithname{filename}{file-name}
\newlexicalelementwithname{functionname}{function-name}
\newlexicalelementwithname{imperativestatement}{imperative-statement}
\newlexicalelementwithname{cobolindexname}{index-name}
\newlexicalelementwithname{libraryname}{library-name}
\newlexicalelementwithname{mnemonicname}{mnemonic-name}
\newlexicalelementwithname{procedurename}{procedure-name}
\newlexicalelementwithname{pseudotext}{pseudo-text}
\newlexicalelementwithname{recordname}{record-name}
\newlexicalelementwithname{reportname}{report-name}
\newlexicalelementwithname{sourcetext}{source-text}
\newlexicalelementwithname{textname}{text-name}
\newlexicalelementwithname{cdname}{cd-name}
\newlexicalelement{argument}
\newlexicalelement{condition}
\newlexicalelement{expression}
\newlexicalelement{identifier}
\newlexicalelement{integer}
\newlexicalelement{literal}
\newlexicalelementwithname{switchstatusname}{switch-status-name}
\newlexicalelementwithname{switchname}{switch-name}
\newlexicalelementwithname{systemname}{system-name}
\newlexicalelementwithname{computername}{computer-name}

\frontmatter

\begin{titlepage}
  \pagecolor{Maroon!75}\afterpage{\nopagecolor}

  \centering

  \vfill

  \begin{streak}[bottom=7.5pt]
    {\centering\fontsize{32pt}{0cm}\bfseries
      The GnuCOBOL 2.2 Grammar \par
    }
    \vspace{1em}
    {\centering\scshape\LARGE for r1899 \par}
  \end{streak}

  \vspace{3cm}
  \begin{streak}
    {\LARGE\itshape Edward Hart\par}
    \vspace{5pt}
    {\large\href{mailto:edward.dan.hart@gmail.com}{\color{white}{edward.dan.hart@gmail.com}}}
  \end{streak}

  \vfill

  % Bottom of the page
  \begin{streak}[top=3pt,bottom=3pt]
    {\Large \today\par}
  \end{streak}
\end{titlepage}

\topskip0pt
\vspace*{\fill}

COBOL is an industry language and is not the property of any company or group of companies, or of any organisation or group of organisations.

No warranty, expressed or implied, is made by any contributor, or by the CODASYL COBOL Committee,\footnote{The CODASYL COBOL committee was dissolved in 1992. Its work was continued by ANSI X3J4 and then INCITS PL22.4, which was itself dissolved in 2015.} as to the accuracy and functioning of the programming system and language. Moreover, no responsibility is assumed by any contributor, or by the committee, in connection therewith.

The authors and copyright holders of the copyrighted materials used herein are:
\begin{itemize}
\item FLOW-MATIC (trademark of Sperry Rand Corporation), Programming for the UNIVAC \textregistered{} I and II, Data Automation Systems, copyrighted 1958, 1959 by Sperry Rand Corporation,\footnote{Sperry Rand's computer business is now part of Unisys.}
\item IBM Commercial Translator, Form No. F28-8013, copyrighted 1959 by IBM, and
\item FACT, DSI 27A5260-2760, copyrighted 1960 by Minneapolis-Honeywell.
\end{itemize}

They have specially authorised the use of this material, in whole or in part, in the COBOL specifications. Such authorisation extends to the reproduction and use of COBOL specifications in programming manuals or similar publications.

\vfill

\begin{center}
  This work is typeset in Roboto.
\end{center}

\vfill

\begin{center}
  Copyright \textcopyright{} \the\year{} Edward Hart

  \vspace{5pt}

  Permission is granted to copy, distribute and\slash{}or modify this document under the terms of the GNU Free Documentation License, Version 1.3 or any later version published by the Free Software Foundation; with no Invariant Sections, no Front-Cover Texts, and no Back-Cover Texts. Your attention is drawn to the copy of the license in Appendix \ref{label_fdl}.
  \vspace{5pt}

  The moral rights of the author have been asserted.
\end{center}

\tableofcontents

\chapter{Foreword}

This document describes the syntax of COBOL as supported by GnuCOBOL. It is hoped it will complement Gary Cutler and Vincent Coen's \textit{GnuCOBOL Programmer's Guide} which (currently) does not document recent features added to GnuCOBOL. This document is also formatted in \LaTeX, so that everything looks a bit prettier.

The syntax diagrams were transcribed from GnuCOBOL's parsers. It thus replicates some unusual syntax rules and misses some syntax rules implemented outside the parser. For example, the obsolete identification division comment paragraphs are allowed in any order and the syntax of \hyperref[file-control-entry]{a file-control entry} does not distinguish between SEQUENTIAL, INDEXED and RELATIVE organisations.

This is a draft and so has many flaws. If people find this document useful, I
will try to fix these shortcomings.

%%% Local Variables:
%%% mode: latex
%%% TeX-master: "grammar.tex"
%%% End:

\chapter{Changelog}

\section{GnuCOBOL 3.0-dev}

\paragraph{General}
\begin{itemize}
\item \textbf{C API (configuration)}: added cob\_set\_runtime\_option and cob\_get\_runtime\_option for specifying the file handle which printer output or trace output should go to.
\item \textbf{cobc command-line options}: added -fdump to dump data of all modules on abend of program.
\item \textbf{cobc command-line options}: disabled -fif-cutoff pending removal.
\item \textbf{cobc command-line options}: -debug no longer implies -ftrace (\bug{449}).
\item \textbf{cobc command-line options}: added -O0 for disabling optimisations (\fr{255}).
\item \textbf{cobc command-line options}: added -Wno-dialect to suppress dialect-specific warnings.
\item \textbf{cobc command-line options}: added -Wother to enable miscellaneous warnings.
\item \textbf{cobcrun command-line options}: renamed -runtime-conf to -runtime-config.
\item \textbf{Copyfiles}: fixed segfault when GnuCOBOL tried and failed to open a copyfile (\bug{458}).
\item \textbf{Copyfiles}: GnuCOBOL now checks for recursive copyfiles (\bug{467}).
\item \textbf{Debugging}: Added debug logging for compiler developers (--enable-debug-log ./configure option).
\item \textbf{DJGGP support}: added minimal support for DJGGP.
\item \textbf{Expressions}: fixed \bug{431}, where decimal constants were not initialised in INITIAL programs.
\item \textbf{IBM i support}: fixed segfault involving filepaths.
\item \textbf{Listings}: fixed the substitution of tokens in quotes in copyfiles (\bug{494}).
\item \textbf{Modules}: enabled running of modules generated with older versions of GnuCOBOL (\fr{239}).
\item \textbf{Nested programs}: fixed \bug{435}, where identifiers in the containing program where not propogated to the contained program, causing a segfault.
\item \textbf{Recursion}: fixed \bug{222}, where the return value was lost when returning from a recursive call.
\item \textbf{Reserved words}: allow users to make default reserved words aliases for other words.
\item \textbf{Signals}: added error message for SIGFPE (see \bug{434}).
\item \textbf{Solaris support}: fixed errors and warnings when compiling in Solaris 10.
\item \textbf{Tracing}: added new options for controlling tracing (see \fr{242}).
\end{itemize}

\paragraph{Configuration options}
\begin{itemize}
\item \textbf{IBM dialect}: reserved words updated to Enterprise COBOL V6.2.
\item \textbf{incorrect-conf-sec-order option}: changed to ``ok'' from ``error'' in mf and default dialects.
\item \textbf{New compiler configuration options}: binary-comp-1, display-special-fig-consts, free-redefines-position (\fr{211}), line-col-zero-default, move-figurative-space-to-numeric, move-non-numeric-lit-to-numeric-is-zero, missing-statement, perform-without-varying-by, record-delimiter, record-delim-with-fixed-recs, screen-section-rules, sequential-delimiters
\item \textbf{New runtime configuration options}: col\_just\_lrc (for enabling\slash{}disabling the LEFT\slash{}RIGHT\slash{}CENTER phrases of the report COLUMN clause), printer and display\_print\_pipe (for specifying what command should be executed before DISPLAY UPON PRINTER; similar to Micro Focus' COBPRINTER option); display\_print\_file (name of file which DISPLAY UPON PRINTER will append its output to), trace\_format, dump\_file and dump\_width.
\item \textbf{Registers}: disabling a register also removes the reserved word with the same name, if it exists (\fr{278}).
\end{itemize}

\paragraph{Compiler directives}
\begin{itemize}
\item \textbf{New directives}: ADDRSV, ADDSYN, MAKESYN, OVERRIDE and REMOVE (\fr{210}); COMP1.
\end{itemize}

\paragraph{Identification division}
\begin{itemize}
\item \textbf{OPTIONS paragraph}: added support for ARITHMETIC clause.
\end{itemize}

\paragraph{Environment division}
\begin{itemize}
\item \textbf{ALTERNATE RECORD KEY clause}: implemented split keys (SOURCE IS; \fr{23}) and sparse keys (SUPPRESS WHEN; \fr{281}).
\item \textbf{ASSIGN clause}: re-enabled use of linkage section or BASED items (\bug{421}).
\item \textbf{CLASS phrase}: added recognition of ALPHANUMERIC\slash{}NATIONAL and alphabet phrases.
\item \textbf{CURRENCY phrase}: changed to emit error when CURRENCY SIGN other than ``\$'' is entered.
\item \textbf{CURRENCY phrase}: improved error messages for invalid currency signs.
\item \textbf{EXTERN clause}: improved error messages (\bug{446}).
\item \textbf{FILE STATUS clause}: added detection of VSAM secondary status identifier (\fr{51}).
\item \textbf{OCCURS clause}: correctly implemented nested OCCURS DEPENDING tables (ODOSLIDE).
\item \textbf{PASSWORD clause}: added detection of.
\item \textbf{RECORD DELIMITER phrase}:  added support for BINARY-SEQUENTIAL and LINE-SEQUENTIAL phrases.
\item \textbf{RECORD DELIMITER phrase}: improved syntax checks (see \bug{442}).
\item \textbf{RECORD KEY clause}: implemented sparse keys (SOURCE IS; \fr{23}).
\end{itemize}

\paragraph{Data division}
\begin{itemize}
\item \textbf{ANY LENGTH clause}: fixed \bug{487}, where literals moved to ANY LENGTH items were incorrectly truncated.
\item \textbf{Screen section}: made some COBOL words context-sensitive to screen section.
\item \textbf{Screen description}: the rules on which clauses must be specified when now depend on the dialect (see \bug{382}).
\item \textbf{SYNCHRONIZED clause}: deactivated RIGHT phrase, pending correct implementation (previously SYNCHRONIZED RIGHT was the same as SYNCHRONIZED LEFT).
\item \textbf{USAGE clause}: added option to make COMP-1 mean a 16-bit signed integer (\fr{272}).
\end{itemize}

\paragraph{Procedure division}
\begin{itemize}
\item \textbf{ACCEPT statement (screen)}: fixed ACCEPT WITH UPDATE and not working (\bug{423}).
\item \textbf{ACCEPT statement (screen)}: fixed \bug{426}, where backspacing at the start of a field moved the cursor to the second-to-last character of the preceding field.
\item \textbf{ACCEPT statement (screen)}: added detection of CONTROL KEY clause.
\item \textbf{ACCEPT statement (screen)}: allow numeric-edited fields to contain spaces (see \bug{491}).
\item \textbf{ACCEPT statement (screen)}: fixed error when filling in a one character field with insert mode on (\bug{498}).
\item \textbf{ACCEPT statement (temporal)}: fixed \bug{469}, where ACCEPT FROM DAY was off by $-1$.
\item \textbf{CALL statement}: fixed ON EXCEPTION not working properly with -fstatic (\bug{462}).
\item \textbf{DISPLAY statement (printer)}: allow for redirecting DISPLAY UPON PRINTER(-1) to files based on the runtime configuration.
\item \textbf{DISPLAY statement (screen)}: fixed DISPLAY LOW-VALUE not setting position of cursor for next DISPLAY statement (\bug{423}).
\item \textbf{DISPLAY statement (screen)}: added configuration option to disable Micro Focus' special behaviour with some figurative constants (see \bug{423}).
\item \textbf{DISPLAY statement (screen)}: fixed \bug{428}, where DISPLAY ALL ``x'' WITH SIZE only displayed ``x'' once.
\item \textbf{Exceptions}: the exception status is now only reset by SET LAST EXCEPTION TO OFF.
\item \textbf{Exceptions}: fixed EC-SIZE-OVERFLOW being raised when EC-SIZE-ZERO-DIVIDE is active (\bug{223}).
\item \textbf{File I-O}: fixed \bug{457}, where file status 30 was set instead of the correct permanent error status (34, 35 or 37).
\item \textbf{File I-O}: added support for VBISAM 2.1.1.
\item \textbf{Floating-point arithmetic}: fixed incorrect SIZE ERROR exceptions (see \bug{470}).
\item \textbf{Floating-point arithmetic}: fixed \bug{478}, where calculations with the SIZE ERROR phrase have different results to calculations without it.
\item \textbf{Floating-point arithmetic}: now activate the SIZE ERROR handler when a floating-point variable is set to a non-finite value (see \bug{122}).
\item \textbf{INQUIRE statement}: added detection of.
\item \textbf{LENGTH OF phrase}: fixed error message when used on group fields (\bug{175}).
\item \textbf{PERFORM statement}: added check for non-zero item in BY phrase (\fr{268}).
\item \textbf{PERFORM statement}: allow the BY phrase to be omitted, which is the same as specifying BY 1 (\fr{158}).
\item \textbf{MODIFY statement}: added detection of.
\item \textbf{MOVE statement}: added options to interpret moving non-numeric values to numeric items as moving zero to numeric items.
\item \textbf{Reference modification}: improved out-of-bounds error message (\bug{445}).
\item \textbf{OPEN statement}: fail with file status 39 when the first record of an indexed file is larger than specified in the FD.
\item \textbf{READ statement (indexed)}: fail with file status 43 when the read record is larger than specified in the FD.
\item \textbf{Reference modification}: added warnings for invalid reference modifications involving variables.
\item \textbf{Report writer}: implemented, including features from COBOL 2002 and the IBM Report Writer.
\item \textbf{SET statement}: fixed \bug{225}, where an invalid SET statement caused an error saying ``invalid MOVE statement''.
\item \textbf{STOP statement}: fixed the NORMAL phrase causing a compilation error (\bug{433}).
\item \textbf{VALIDATE statement}: added detection of.
\end{itemize}

\paragraph{Intrinsic functions}
\begin{itemize}
\item \textbf{MOD function}: fixed EC-SIZE-ZERO-DIVIDE being raised instead of EC-ARGUMENT-FUNCTION when zero was provided as an argument.
\item \textbf{REM function}: fixed EC-SIZE-ZERO-DIVIDE being raised instead of EC-ARGUMENT-FUNCTION when zero was provided as an argument.
\item \textbf{WHEN-COMPILED function}: fixed timezone, which was missing its sign and contained nonsense when negative (\bug{436}).
\end{itemize}

\paragraph{Built-in subprograms}
\begin{itemize}
\item \textbf{C\$GETPID}: fixed wrong process ID being returned after forking (\bug{451}).
\item \textbf{C\$SLEEP}: if the requested time is too large, sleep for the maximum possible time instead of not at all.
\item \textbf{CBL\_READ\_KBD\_CHAR}: this now works (\bug{500}).
\item \textbf{SYSTEM}: added workaround of buggy system() implementation on Windows, which removes leading and trailing quotes.
\end{itemize}

\section{GnuCOBOL 2.2}

This list tracks changes made from 23 November 2013. This excludes many changes made in 2009--2013 which would be pertinent to those upgrading from a 2009 build of OpenCOBOL 1.1 found in many package repositories.

\paragraph{General}
\begin{itemize}
\item \textbf{64-bit numbers}: fixed bugs in handling of 64-bit numbers (e.g. \bug{229}).
\item \textbf{ACUCOBOL windows}: added detection of ACUCOBOL's window\slash{}message box GUI syntax.
\item \textbf{C API (data)}: added several functions for getting and setting cob\_field items.
\item \textbf{C API (files)}: added cob\_file\_external\_addr, cob\_file\_malloc and cob\_file\_free.
\item \textbf{C API (screen)}: added several functions from Micro Focus' C to COBOL API: cob\_display\_text, cob\_sys\_get\_char, cob\_get\_char, cob\_get\_text, cob\_display\_formatted\_text, cob\_sys\_get\_csr\_pos, cob\_sys\_set\_csr\_pos, cobmove, cobaddstrc, cobprintf and cobgetch (feature requests \frnoname{148} and~\frnoname{187}).
\item \textbf{C API (signals)}: added cob\_raise to send signal to signal handlers.
\item \textbf{C compiler support}: fixed errors in compilers without designated initializers.
\item \textbf{cobc command-line options}: added -O3 to enable more optimisations.
\item \textbf{cobc command-line options}: added -Wfatal-error to make the compiler abort on the first error.
\item \textbf{cobc command-line options}: added -Wpossible-overlap to warn items that \emph{may} overlap (-Woverlap only warns if items definitely overlap).
\item \textbf{cobc command-line options}: added -fmax-errors to set number of errors at which the compiler aborts.
\item \textbf{cobc command-line options}: added -fwinmain to output WinMain instead of main (\fr{194}).
\item \textbf{cobc command-line options}: added -t and -T for complete listing support (-t for 80-characters wide listings and -T for 132-characters wide) which includes cross-references (thanks to Dave Pitts).
\item \textbf{cobc command-line options}: added -vvv (like -vv but passes verbose option to the linker as well) and -\#\#\# (like -v but commands are not executed).
\item \textbf{cobc command-line options}: allow -, i.e. stdin, as a source file.
\item \textbf{COBOL-85 NIST testsuite}: tests now refer to \$COBC, \$COBCRUN and \$COBCRUN\_DIRECT environment variables instead of directly calling cobc and cobcrun, allowing the testsuite to run in conjunction with tools like valgrind.
\item \textbf{COBOL-85 NIST testsuite}: tests for obsolete feature flagging are now executed, if possible.
\item \textbf{Comments}: added ACUCOBOL comments: \$ as synonym for * in indicator area and $\vert$ as synonym for floating comment indicator *>.
\item \textbf{Communication facility}: added detection of communication facility syntax.
\item \textbf{configure}: added useful error message when help2man, bison and flex are missing when they are needed.
\item \textbf{curses}: fixed compilation errors when configured without curses (\bug{90}).
\item \textbf{Error messages}: error messages are now lowercase, in line with the GNU Coding Standards (\bug{198}).
\item \textbf{Error messages}: segfaults in the compiler now cause an error message to be displayed.
\item \textbf{Error messages}: replaced instances of ``ODO'' by the clearer ``OCCURS DEPENDING ON''.
\item \textbf{Expressions}: resolve constant expressions and optimise constant decimals at compile time.
\item \textbf{Expressions}: added support for IBM OS/VS COBOL's arithmetic.
\item \textbf{Expressions}: improved error messages for malformed expressions.
\item \textbf{Indicators}: invalid indicators no longer cause compilation to immediately terminate (\fr{126}).
\item \textbf{Information}: output compiler version used to build GnuCOBOL and any mathematical libraries used (\fr{169}).
\item \textbf{Information}: output what a reserved word is an alias for in the --list-reserved output (\fr{214}).
\item \textbf{Manpage}: added manpage generation and installation.
\item \textbf{Nested programs}: Nested programs no longer need to have END PROGRAM.
\item \textbf{National literals}: added basic support for national literals.
\item \textbf{Numeric literals}: added ACUCOBOL numeric literals: B\#\ldots\, for binary, O\#\ldots\, for octal, and X\#\ldots\ and H\#\ldots\, for hexadecimal.
\item \textbf{Literals}: fixed heap corruptions caused by uncommon literals (\bug{195}).
\item \textbf{Literals}: allow concatenation of literal and Boolean literals.
\item \textbf{Memory management}: all memory belonging to the parsers and lexers is freed upon a compiler abend.
\item \textbf{Memory management}: fixed memory leaks due to recursive CALLs.
\item \textbf{Microsoft Visual C++}: output when compiling with cl.exe is now filtered and temporary files are deleted.
\item \textbf{MinGW}: fixed use of wrong directory separator.
\item \textbf{Signals}: removed error message on SIGPIPE.
\item \textbf{Signals}: added error message for SIGBUS.
\item \textbf{Translations}: updated, with new support for German and Italian.
\item \textbf{User-defined functions}: function definitions must now end with END FUNCTION.
\item \textbf{User-defined functions}: function definitions may no longer be nested in programs (\bug{255}).
\item \textbf{Windows support}: allow linking with asm files.
\item \textbf{Windows support}: added support for DISAM in the batch file which creates distributables.
\item \textbf{Windows support}: fixed environment-setting batch files not working with Microsoft Visual Studio 2017.
\item \textbf{Windows support}: fixed 64-bit environment-setting batch files not checking the correct directories for binaries and libraries.
\end{itemize}

\paragraph{Configuration options}
\begin{itemize}
\item \textbf{Deleted compiler configuration options}: eject-statement, cobol85-reserved.
\item \textbf{New compiler configurations}: all dialects have been split into standard and strict dialects, with strict dialects maintaining source compatibility with the dialect's compiler(s).
\item \textbf{New compiler configurations}: acu for ACUCOBOL, cobol2014 for COBOL 2014, rm for RM-COBOL, xopen for X\slash{}Open.
\item \textbf{New compiler configuration options}: accept-display-extensions, accept-update, accept-auto, acu-literals, arithmetic-osvs, call-overflow, console-is-crt, constant-01, constant-78, constant-folding, define-constant-directive, hexadecimal-boolean, hexadecimal-national-literals, incorrect-conf-sec-order, intrinsic-function, listing-statements, literal-length, move-figurative-constant-to-numeric, move-figurative-quote-to-numeric, move-ibm, national-literals, no-echo-means-secure, not-exception-before-exception, numeric-boolean, numeric-literal-length, numeric-value-for-edited-item, pic-length, program-name-redefinition, program-prototypes, reference-out-of-declaratives (\fr{179}), register, renames-uncommon-levels, reserved, reserved-words, stop-identifier, system-name, title-statement, use-for-debugging, word-length (\fr{43}). % TO-DO: expand
\item \textbf{Registers}: compiler configurations can now specify all the registers to generate.
\item \textbf{Registers}: added registers not yet implemented by GnuCOBOL as reserved words.
\item \textbf{Renamed compiler configuration options}: debugging-line to debugging-mode, relaxed-syntax-check to relax-syntax-checks.
\item \textbf{Reserved words}: compiler configurations can now specify all the reserved words and context-sensitive words permitted.
\item \textbf{Reserved words}: compiler configurations can now specify whether a reserved word is an alias for another reserved word. % TO-DO: Improve
\item \textbf{Runtime configuration}: added ability to configure some libcob features at runtime.
\item \textbf{Support options}: options which specify if a feature is supported can now take a ``+'' before their argument to indicate it takes effect only if the current level of support is less strict than ``ok''.
\end{itemize}

\paragraph{Compiler directives}
\begin{itemize}
\item \textbf{\$ indicator character}: added \$ as an indicator for compiler directive lines.
\item \textbf{\directiveindicator{}IF directive}: fixed \bug{263}, where nested \directiveindicator{}IF directives were not handled correctly.
\item \textbf{New constants}: GCCOMP, GNUCOBOL.
\item \textbf{New directives}: \directiveindicator{}CALL-CONVENTION, \directiveindicator{}LISTING, \directiveindicator{}PAGE. % TO-DO: Expand
\item \textbf{New directives (detection only)}: *CBL, *CONTROL, TITLE.
\item \textbf{New \directiveindicator{}SET phrase}: SOURCEFORMAT.
\end{itemize}

\paragraph{Identification division}
\begin{itemize}
\item \textbf{Comment paragraphs}: fixed invalid parsing of quote characters inside comment paragraphs (\bug{297}).
\item \textbf{FUNCTION-ID}: added checks for redefinition of function-names.
\item \textbf{INITIAL phrase}: fixed premature deallocation of INITIAL programs (\bug{52}).
\item \textbf{OPTIONS paragraph}: added with implementation of DEFAULT ROUNDED MODE and ENTRY-CONVENTION phrases and recognition of INTERMEDIATE ROUNDING phrase.
\item \textbf{PROGRAM-ID}: added checks for redefinition of program-names.
\item \textbf{PROGRAM-ID phrases}: permit INITIAL or RECURSIVE before COMMON (\bug{244}).
\item \textbf{Program\slash{}function-names}: warn if program\slash{}function-names contain spaces.
\end{itemize}

\paragraph{Environment division}
\begin{itemize}
\item \textbf{ASSIGN clause}: missing ASSIGN clauses are now detected at compile-time.
\item \textbf{ASSIGN clause}: added PRINTER and PRINTER-1 device-names for writing to a printer.
\item \textbf{ASSIGN clause}: added CARD-PUNCH, CARD-READER, CASSETTE, INPUT, INPUT-OUTPUT, MAGNETIC-TAPE and OUTPUT device-names for line sequential devices.
\item \textbf{ASSIGN clause}: temporarily prohibit BASED and linkage items in ASSIGN USING due to \bug{421}.
\item \textbf{CALL-CONVENTION phrase}: statically calling functions with CALL-CONVENTION 74 no longer causes linker errors (\bug{316}).
\item \textbf{CURRENCY phrase}: fixed \bug{182}, where a preceding SWITCH phrase caused an incorrect duplicate CURRENCY clause error.
\item \textbf{File-control entry}: fixed \bug{71}, where referring to a global constant caused an internal error.
\item \textbf{File-control entry}: fixed \bug{331}, where using an identifier in a file record qualified with the file's name caused an error.
\item \textbf{FUNCTION phrase}: added checks for redefinition of function-(prototype-)names.
\item \textbf{FUNCTION phrase}: compiler will no longer stop when it encounters a syntax error.
\item \textbf{LOCK MODE clause}: fixed combination of LOCK MODE IS AUTOMATIC/MANUAL with LOCK ON MULTIPLE.
\item \textbf{PROGRAM phrase}: added support for program-prototype-names.
\item \textbf{SIGN clause}: improved syntax checks.
\item \textbf{SWITCH phrase}: added check for duplicate on\slash{}off clauses (\bug{136}).
\item \textbf{SWITCH phrase}: added new switch names: SWITCH-16 through to SWITCH-36 (\fr{65}), ``SWITCH 1'' to ``SWITCH 26'' (and their aliases  ``SWITCH A'' to ``SWITCH Z''), UPSI-0 to UPSI-8 (equivalent to ``SWITCH 0'' to ``SWITCH 8'') and USW-0 to USW-31 (equivalent to ``SWITCH 0'' to ``SWITCH 31'').
\end{itemize}

\paragraph{Data division}
\begin{itemize}
\item \textbf{78-level items}: strengthened syntax checks.
\item \textbf{88-level items}: strengthened syntax checks.
\item \textbf{ANY NUMERIC clause}: ANY NUMERIC items must now have PIC 9.
\item \textbf{ANY LENGTH clause}: ANY LENGTH items may no longer be BY VALUE parameters (see \bug{219}).
\item \textbf{ANY LENGTH clause}: ANY LENGTH items must now have PIC X or PIC N.
\item \textbf{BLANK clause}: fixed \bug{143}, where BLANK LINE\slash{}SCREEN did not colour line\slash{}screen.
\item \textbf{BLANK WHEN ZERO clause}: added checks that BLANK WHEN ZERO is not specified with PICTURE clauses containing S.
\item \textbf{Constant items}: expressions in VALUE clauses now permitted.
\item \textbf{Data description}: added a maximum record length.
\item \textbf{Data description}: increased maximum size of non-indexed file record to 64 MiB (maximum size of an indexed file record is 65535 bytes).
\item \textbf{ERASE clause}: fixed \bug{186}, where ERASE EOL and ERASE EOS could be specified simultaneously.
\item \textbf{FULL clause}: added warning for useless FULL clauses on numeric items (\fr{209}).
\item \textbf{HIGHLIGHT and LOWLIGHT clauses}: added checks that HIGHLIGHT and LOWLIGHT are not specified simultaneously.
\item \textbf{Local-storage section}: fixed \bug{78}, where local-storage items where initialised after file section items.
\item \textbf{LOWLIGHT clause}: implemented.
\item \textbf{OCCURS clause}: fixed internal compiler when used with SYNC (\bug{155}).
\item \textbf{OCCURS clause}: allow KEY phrase and INDEXED phrase in any order.
\item \textbf{OCCURS clause}: fixed \bug{167}, where overly large numeric literals where accepted in the OCCURS clause.
\item \textbf{OCCURS clause (depending)}: require the minimum length to be less than the maximum length (\fr{99}).
\item \textbf{OCCURS clause (depending)}: disabled nested OCCURS DEPENDING tables due to bugs.
\item \textbf{OCCURS clause (screen-section)}: require relative LINE\slash{}COLUMN clauses in OCCURS entries (\bug{83}).
\item \textbf{OCCURS clause (unbounded)}: added by Frank Swarbrick (\patch{50}).
\item \textbf{PICTURE clause}: restricted number of permitted PICTURE strings (\bug{232}).
\item \textbf{PICTURE clause}: improved checks of constant-names referenced in PICTURE strings.
\item \textbf{RENAMES items}: strengthened syntax checks.
\item \textbf{RESERVE clause}: allow the optional word AREAS.
\item \textbf{Screen description}: permit figurative constants in screen items (\bug{108}).
\item \textbf{TALLY special register}: added.
\item \textbf{USAGE clause}: added ACUCOBOL's HANDLE phrases (see \fr{77}).
\item \textbf{VALUE clause}: VALUE clauses in REDEFINES entries now cause warnings, not errors, for compatibility.
\item \textbf{Variable records}: added checks that the minimum size of a variable record is large enough to contain the record key.
\end{itemize}

\paragraph{Procedure division}
\begin{itemize}
\item \textbf{ACCEPT statement}: added ESCAPE as synonym for EXCEPTION.
\item \textbf{ACCEPT statement}: permit clauses in any order.
\item \textbf{ACCEPT statement}: allow WITH before every screen attribute clause.
\item \textbf{ACCEPT statement}: entering control-C now terminates the program.
\item \textbf{ACCEPT statement (screen)}: fixed failed ACCEPTs caused by a buffer overflow.
\item \textbf{ACCEPT statement (screen)}: enhanced support for special keys (insert, tab, delete, alt-delete, etc.).
\item \textbf{ACCEPT statement (screen)}: fixed \bug{161} where screens terminated after entering a few characters in a field.
\item \textbf{ACCEPT statement (screen)}: added DEFAULT as synonym for UPDATE.
\item \textbf{ACCEPT statement (screen)}: ERASE and BLANK clauses in screens are now ignored (\bug{192}).
\item \textbf{ACCEPT statement (screen)}: fixed \bug{160} where ACCEPT statement LINE\slash{}COLUMN clauses did not work.
\item \textbf{ACCEPT statement (screen)}: fixed segfault on ACCEPT OMITTED (\bug{300}).
\item \textbf{ACCEPT statement (screen)}: added checks that screen attributes are not specified multiple times or after conflicting attributes.
\item \textbf{ACCEPT statement (screen)}: fixed some phrases not being recognised without being preceded by WITH (\bug{402}).
\item \textbf{ACCEPT statement (screen)}: fixed the backspace and delete keys not working and the insert key not toggling between insertion and overwriting.
\item \textbf{ACCEPT statement (screen)}: cursor now changes with insertion\slash{}overwrite mode (if supported by the terminal).
\item \textbf{ACCEPT statement (screen)}: a beep is emitted on attempts to \emph{insert} data into a full field.
\item \textbf{ADD statement (corresponding)}: restricted to numeric items (\bug{235}).
\item \textbf{ADD statement (table)}: added detection of ADD TABLE.
\item \textbf{Addition of COMP-3 numbers}: fixed bug where COMP-3 addition failed.
\item \textbf{Addition of floating-point numbers}: fixed incorrect addition of floating-point numbers.
\item \textbf{CALL statement}: implemented \fr{101}, allowing more arguments to be provided.
\item \textbf{CALL statement}: fixed behaviour when calling cancelled modules.
\item \textbf{CALL statement}: added RETURNING NOTHING. % TO-DO: What does this do?
\item \textbf{CALL statement}: the generation of C function declarations for static CALLs can now be disabled.
\item \textbf{CALL statement}: added checks for static CALLs referring to C macros.
\item \textbf{CALL statement}: warn if a literal containing the program-name contains spaces.
\item \textbf{CALL statement}: added detection of NESTED phrase.
\item \textbf{CANCEL statement}: fixed crash caused by cancelling a cancelled module.
\item \textbf{Conditions}: restricted use of IS (\bug{321}).
\item \textbf{Conditions}: added warnings for always true\slash{}false conditions (including the reason why it is always true\slash{}false).
\item \textbf{DESTROY statement}: added detection of DESTROY.
\item \textbf{DISPLAY statement}: permit clauses in any order.
\item \textbf{DISPLAY statement}: allow WITH before every screen attribute clause.
\item \textbf{DISPLAY statement (screen)}: fixed bug where EC-SCREEN exceptions did not trigger ON EXCEPTION handler (\bug{243}).
\item \textbf{DISPLAY statement (screen)}: fixed bugs in DISPLAY SPACES\slash{}ALL X``02''\slash{}ALL X``07''.
\item \textbf{DISPLAY statement (screen)}: added checks that screen attributes are not specified multiple times or after conflicting attributes.
\item \textbf{DISPLAY statement (screen)}: DISPLAY OMITTED marked as unfinished; currently equivalent to DISPLAY LOW-VALUE.
\item \textbf{DISPLAY statement (screen)}: fixed some phrases not being recognised without being preceded by WITH (\bug{402}).
\item \textbf{END DECLARATIVES phrase}: fixed \bug{88}, where an erroneous unreachable code warning was emitted for code without a main procedure.
\item \textbf{ENTRY statement}: suppress incorrect unreachable code warnings.
\item \textbf{Exception handlers}: permit NOT ON EXCEPTION\slash{}END-OF-PAGE\slash{}etc. before ON EXCEPTION\slash{}END-OF-PAGE\slash{}etc.
\item \textbf{EXIT statement}: added extension RETURNING\slash{}GIVING clause for PROGRAM phrase.
\item \textbf{File I-O}: added detection of and handling for error when no disc space is available for output files.
\item \textbf{File I-O}: added RETRY and ADVANCING ON LOCK as pending features.
\item \textbf{File I-O}: fixed detection of DISAM file handler.
\item \textbf{FREE statement}: NULL addresses no longer cause an exception.
\item \textbf{GOBACK statement}: added extension RETURNING\slash{}GIVING clause.
\item \textbf{INITIALIZE statement}: fixed \bug{84}, where literals could be passed to INITIALIZE.
\item \textbf{INITIALIZE statement}: fixed \bug{287}, where reference-modified group items were not treated like elementary items.
\item \textbf{INSPECT statement}: fixed \bug{47}, where clauses were permitted in invalid orders.
\item \textbf{LENGTH OF phrase}: fixed \bug{89}, where the length of REDEFINES item where calculated incorrectly.
\item \textbf{LENGTH OF phrase}: fixed \bug{110}, where LENGTH OF was not allowed in the UNTIL phrase of a PERFORM statement or in a VALUE clause.
\item \textbf{MOVE statement}: added more checks for overlapping MOVE statements.
\item \textbf{MOVE statement}: fixed truncation of COMP numbers not conforming to the binary-truncate setting (\bug{69}).
\item \textbf{MOVE statement}: fixed \bug{344}, where trying to MOVE to a procedure-name caused a segfault.
\item \textbf{MOVE statement}: added support for IBM's character-by-character MOVE.
\item \textbf{PERFORM statement}: fixed \bug{368}, where the compiler segfaulted when there was a PERFORM statement with an empty body and DEBUGGING MODE was specified.
\item \textbf{Procedure division header}: fixed \bug{55}, where a user-defined function without parameters failed to compile.
\item \textbf{Procedure division header}: disabled the BY VALUE phrase, pending a working implementation.
\item \textbf{Procedure division header}: fixed \bug{349}, where BY VALUE pointer parameters lead to code that couldn't be compiled by older versions of Microsoft Visual C++ (patched by Mario Matos).
\item \textbf{Procedure division header}: RETURNING items must now be declared in the linkage section.
\item \textbf{Procedure division header}: added RETURNING OMITTED.
\item \textbf{Procedure division header}: added entry-convention specifiers.
\item \textbf{Procedure division header}: now mandatory in function definitions (see \bug{271}).
\item \textbf{Procedure division header}: CHAINING programs may no longer be called by other programs (\bug{354}), per the ACUCOBOL implementation.
\item \textbf{Reference modification}: fixed \bug{146}, where the length of reference-modified item in an OCCURS DEPENDING table was too long because it was assumed the OCCURS DEPENDING table was at its maximum size.
\item \textbf{READ statement}: a failed second READ of a missing OPTIONAL file now results in a file status of 46, not 23.
\item \textbf{REWRITE statement}: added REWRITE FILE (\fr{170}).
\item \textbf{Screen I-O}: added detection of situations which raise EC-SCREEN-LINE-NUMBER, EC-SCREEN-STARTING-COLUMN and EC-SCREEN-ITEM-TRUNCATED.
\item \textbf{Screen I-O}: added support for the LINE 0 and COL 0 extensions.
\item \textbf{Screen I-O}: added some ACUCOBOL synonyms (NO ECHO, OFF, REVERSED, REVERSE, etc.).
\item \textbf{Screen I-O}: added detection of ACUCOBOL's non-standard clauses like TAB, NO-ECHO, STANDARD, BACKGROUND-HIGH, BACKGROUND-LOW, BACKGROUND-STANDARD and SIZE.
\item \textbf{SEARCH statement (ALL)}: fixed \bug{314}, where SEARCH ALL with an empty OCCURS DEPENDING table did not exit as soon as possible.
\item \textbf{Segment numbers}: added syntax checks.
\item \textbf{SET statement (address)}: disallowed changing address of non-01\slash{}77-level item (\bug{366}).
\item \textbf{SET statement (attribute)}: made HIGHLIGHT ON imply LOWLIGHT OFF and vice versa.
\item \textbf{SET statement (exception)}: added.
\item \textbf{SET statement (thread)}: added detection of ACUCOBOL extension.
\item \textbf{STOP statement (identifier)}: added (see \bug{320}).
\item \textbf{STOP statement (literal)}: fixed segfault.
\item \textbf{STOP statement (thread)}: added detection of ACUCOBOL extension.
\item \textbf{STRING statement}: strengthened syntax checks (\bug{259}).
\item \textbf{SUBTRACT statement (corresponding)}: restricted to numeric items (\bug{235}).
\item \textbf{SUBTRACT statement (table)}: added detection of SUBTRACT TABLE.
\item \textbf{Tracing}: fixed \bug{216}, where a segfault occurred with a program made from modules some of which had been compiled with tracing and physical CANCEL % TO-DO: Define "physical CANCEL"
enabled and some of which hadn't.
\item \textbf{UNSTRING statement}: fixed \bug{54}, where the POINTER value was calculated incorrectly when the delimiter was longer than one character.
\item \textbf{UNSTRING statement}: allow a literal to be the subject of an UNSTRING.
\item \textbf{WRITE statement}: added WRITE FILE (\fr{170}).
\end{itemize}

\paragraph{Intrinsic functions}
\begin{itemize}
\item \textbf{New functions (ACUCOBOL)}: ABSOLUTE-VALUE (synonym for ABS).
\item \textbf{New functions (COBOL 2014)}: FORMATTED-CURRENT-DATE, FORMATTED-DATE, FORMATTED-DATETIME, FORMATTED-TIME, INTEGER-OF-FORMATTED-DATE, TEST-FORMATTED-DATETIME.
\item \textbf{ISO-8601-date-handling functions}: added extension SYSTEM-OFFSET as replacement for last optional argument.
\item \textbf{ISO-8601-date-handling functions}: added EC-IMP-UTC-UNKNOWN if a time format ending in Z is provided but the timezone cannot be found.
\item \textbf{LENGTH function}: added detection of PHYSICAL phrase.
\item \textbf{RANDOM function}: fixed non-random number generation.
\end{itemize}

\paragraph{Built-in subprograms}
\begin{itemize}
\item \textbf{CBL\_GC\_FORK}: added.
\item \textbf{CBL\_GC\_PRINTABLE}: renamed from C\$PRINTABLE.
\item \textbf{CBL\_GC\_WAITPID}: added.
\item \textbf{CBL\_SET\_CSR\_POS}: added (feature requests \frnoname{148} and~\frnoname{187}).
\item \textbf{CBL\_READ\_KBD\_CHAR}: added (feature requests \frnoname{148} and~\frnoname{187}).
\end{itemize}

%%% Local Variables:
%%% mode: latex
%%% TeX-master: "grammar.tex"
%%% End:


\mainmatter

\chapter{Key}

\begin{table}[!h]
  \centering
  \begin{tabular}[!h]{p{0.4\textwidth} p{0.5\textwidth}}
    \toprule
    Element & Notes \\ \midrule
    Braces, $\left\{\ {}\right\}$ & One element within the braces must be selected. \\
    Brackets, $\left[\ {}\right]$ & One or zero elements within the brackets must be selected. \\
    Vertical lines, $\left|\ {}\right|$ & Each element may be selected once and in any order; if within braces, at least one element must be selected. \\
    Ellipsis, \ldots & The preceding element may be repeated any number of times. \\
    OPTIONAL-RESERVED-WORD & \\
    \key{MANDATORY-RESERVED-WORD} & Mandatory reserved words in brackets are often used instead of optional reserved words to indicate an optional feature. \\
    \deleted{Deleted element} & These elements were previously in the COBOL standard but have since been deleted. Their use is strongly discouraged. \\
    \archaic{Archaic element} & These elements remain in the standard, but their use is considered poor style and is strongly discouraged. \\
    \obsolete{Obsolete element} & These elements are slated to be deleted from the standard. Their use is strongly discouraged. \\
    \xopen{X\slash{}Open extension} & \\
    \gnucobol{GnuCOBOL-only extension} & \\
    \miscext{Miscellaneous extension} & An extension which may have come from COBOL dialects by AcuCorp, CA, Fujitsu, HP, IBM, Micro Focus, Microsoft or Ryan-McFarland. \\
    \pending{Unimplemented element} & These elements are recognised by GnuCOBOL, but are non-functional. \\ \bottomrule
  \end{tabular}
\end{table}

%%% Local Variables:
%%% mode: latex
%%% TeX-master: "grammar.tex"
%%% End:


\preto\section{\clearpageifnotfirst{section}}

\chapter{Language fundamentals}

\section{Lexical elements}

\subsection{COBOL words}

\subsection{User-defined words}

\subsection{Reserved words}

\subsection{Literals}

\section{References}

\section{Expressions}

\subsection{Arithmetic expressions}

Arithmetic expressions may contain the following operators:

\begin{table}[!h]
  \begin{tabular}[!h]{l l l}
    \toprule
    \textbf{Binary operators} & \textbf{Purpose} & \textbf{Precedence} \\
    + & addition & 1 \\
    -- & subtraction & 1\\
    * & multiplication & 2\\
    / & division & 2 \\
    ** & exponentiation & 3 \\
    \gnucobol{\^{}} & \gnucobol{exponentiation} & 3 \\ \midrule
    \textbf{Unary operators} \\
    + & no effect & 4 \\
    -- & multiplication by $-1$ & 4 \\ \bottomrule
  \end{tabular}
\end{table}

Binary operators must have a numeric item or expression to both their left and right. Unary operators must have a numeric item or expression to their right only.

Operators with greatest precedence are evaluated first. If an expression contains multiple operators of equal precedence, they are evaluated from left to right.

Arithmetic expressions may contain arithmetic expressions surrounded by parentheses. These nested expressions are evaluated first, before any of the operators of the outer expression.

\begin{table}[!h]
  \centering
  \begin{tabular}[!h]{c c c c c c}
    \toprule
    \multirow{2}{*}{\textbf{First symbol}} & \multicolumn{5}{c}{\textbf{Second symbol}} \\
    \cmidrule(lr){2-6}
                                           & Identifier or literal & Binary operator & Unary operator & (     & ) \\ \midrule
    Identifier or literal &                       & \tick           &                &       & \tick \\
    Binary operator       & \tick                 &                 & \tick          & \tick & \\
    Unary operator        & \tick                 &                 &                & \tick & \\
    (                     & \tick                 &                 & \tick          & \tick & \\
    )                     &                       & \tick           &                &       & \tick \\
    \bottomrule
  \end{tabular}
\end{table}

\subsection{Concatenation expressions}

\begin{syntax}
  \begin{1=}
    \literal \\
    \metaelement{concatenation-expression-1}
  \end{1=}
  \& \literal
\end{syntax}

\subsection{Conditional expressions}

\begin{table}[!h]
  \begin{tabular}[!h]{l l l}
    \toprule
    \textbf{Binary operators} & \textbf{Purpose} & \textbf{Precedence} \\
    AND & logical and & 1 \\
    OR & logical or & 2 \\ \midrule
    \textbf{Unary operator} \\
    NOT & logical not & 3 \\ \bottomrule
  \end{tabular}
\end{table}

%%% Local Variables:
%%% mode: latex
%%% TeX-master: "grammar.tex"
%%% End:

\section{Concepts}

\subsection{Files}

\subsubsection{I-O status}

\paragraph{Successful completion of an operation}
\begin{enumerate}
\item[00] Success
\item[02] Success -- duplicate
\item[04] Success -- incomplete
\item[05] Success -- optional
\item[07] Success -- no unit
\end{enumerate}

\paragraph{Implementor-defined successful completion}
\paragraph{End of file}
\begin{enumerate}
\item[10] End of file
\item[14] Out of key range
\end{enumerate}

\begin{enumerate}
\item[21] Key invalid
\item[22] Key exists
\item[23] Key does not exist
\item[24] Key boundary
\end{enumerate}


\begin{enumerate}
\item[30] Permanent I-O error
\item[31] Inconsistent filename
\item[34] Boundary violation
\item[35] File does not exist
\item[37] Permission denied
\item[38] Closed with lock
\item[39] Conflict attribute
\end{enumerate}

\begin{enumerate}
\item[41] File already open
\item[42] File not open
\item[43] Read not done
\item[44] Record overflow
\item[46] Read error
\item[47] Input denied
\item[48] Output denied
\item[49] I-O denied
\end{enumerate}

\begin{enumerate}
\item[51] Record locked
\item[57] I-O linage
\end{enumerate}

\begin{enumerate}
\item[61] File sharing
\end{enumerate}

\begin{enumerate}
\item[91] Not available
\end{enumerate}

\subsubsection{Organizations}

\subsubsection{Locking}

\subsection{Locales}

\subsection{Screens}

\subsection{User-defined functions}

%%% Local Variables:
%%% mode: latex
%%% TeX-master: "grammar.tex"
%%% End:


\preto\subsection{\clearpageifnotfirst{subsection}}

\chapter{Compiler directives}

\section{CALL-CONVENTION directive}

\begin{syntax}
  \directiveindicator\key{CALL-CONVENTION}
  \begin{1=}
    \key{COBOL} \\
    \key{EXTERN} \\
    \key{STDCALL} \\
    \key{STATIC}
  \end{1=}
\end{syntax}

\subsubsection{Syntax rules}

\subsubsection{General rules}

\section{*CONTROL statement}

\begin{syntax}[\miscextcolour]
  \pending{
    \begin{1=}
      *\key{CBL} \\
      *\key{CONTROL}
    \end{1=}
    \begin{1=}
      \key{SOURCE} \\
      \key{NOSOURCE} \\
      \key{LIST} \\
      \key{NOLIST} \\
      \key{MAP} \\
      \key{NOMAP}
    \end{1=}
  }
\end{syntax}

\subsubsection{Syntax rules}

\subsubsection{General rules}

\section{COPY statement}

\begin{syntax}
  \begin{1=}
    \key{COPY} \\
    \deleted{\key{INCLUDE}}
  \end{1=}
  \begin{1=}
    \literal \\
    \textname \\
  \end{1=}
  \begin{0-1}
    \begin{1=}
      \key{IN} \\
      \key{OF}
    \end{1=}
    \begin{1=}
      \literal \\
      \libraryname
    \end{1=}
  \end{0-1}

  \begin{0-1}
    \key{SUPPRESS} PRINTING
  \end{0-1}

  \begin{0-1}
    \key{REPLACING}
    \begin{1=}
      \begin{1=}
        == \pseudotext == \\
        \identifier \\
        \literal
      \end{1=}
      \key{BY}
      \begin{1=}
        == \pseudotext == \\
        \identifier \\
        \literal
      \end{1=} \\

      \begin{1=}
        \key{LEADING} \\
        \key{TRAILING}
      \end{1=}
      == \metaelement{partial-word-1} ==
      \key{BY}
      == \metaelement{partial-word-2} ==
    \end{1=}\ldots
  \end{0-1}
  .
\end{syntax}


\subsubsection{Syntax rules}

\subsubsection{General rules}

\section{D directive}

\begin{syntax}[\miscextcolour]
  \directiveindicator\key{D} \sourcetext
\end{syntax}

\subsubsection{Syntax rules}

\subsubsection{General rules}

\section{DEFINE directive}

\begin{syntax}
  \begin{1=}
    \directiveindicator \\
    \miscext{\textdollar}
  \end{1=}
  \key{DEFINE}
  \gnucobol{
    \begin{0-1}
      \key{CONSTANT}
    \end{0-1}
  }
  \metaelement{compilation-variable-1} AS
  \begin{1=}
    \begin{1=}
      \literal \\
      \key{PARAMETER}
    \end{1=}
    \begin{0-1}
      \key{OVERRIDE}
    \end{0-1} \\
    \key{OFF}
  \end{1=}
\end{syntax}

\subsubsection{Syntax rules}

\subsubsection{General rules}

\section{DISPLAY directive}

\format{general}
\begin{syntax}[\miscextcolour]
  \begin{1=}
    \gnucobol{\directiveindicator} \\
    \textdollar
  \end{1=}
  \key{DISPLAY} \sourcetext
\end{syntax}

\format{vcs}
\begin{syntax}[\miscextcolour]
  \pending{
    \textdollar\key{DISPLAY} \key{VCS} = \metaelement{version-string}
  }
\end{syntax}

\subsubsection{Syntax rules}

\subsubsection{General rules}

\section{EJECT statement}

\begin{syntax}[\miscextcolour]
  \pending{\key{EJECT}}
\end{syntax}

\subsubsection{Syntax rules}

\subsubsection{General rules}

\section{IF directive}

\begin{syntax}
  \begin{1=}
    \directiveindicator \\
    \miscext{\textdollar}
  \end{1=}
  \key{IF} \metaelement{compilation-variable-1} IS NOT
  \begin{1=}
    \key{DEFINED} \\
    \key{SET} \\
    relation \\
    \metaelement{compilation-variable-2}
  \end{1=}

  \sourcetext

  \miscext{
    \begin{0-1}
      \begin{1=}
        \gnucobol{\directiveindicator} \\
        \textdollar
      \end{1=}
      \begin{1=}
        \key{ELIF} \\
        \key{ELSE-IF}
      \end{1=}
      \condition
      \sourcetext
    \end{0-1} \ldots
  }

  \begin{0-1}
    \begin{1=}
      \directiveindicator \\
      \miscext{\textdollar}
    \end{1=}
    \key{ELSE} \sourcetext
  \end{0-1}

  \begin{0-1}
    \directiveindicator\key{END-IF} \\
    \miscext{\textdollar\key{END}}
  \end{0-1}
\end{syntax}

\subsubsection{Syntax rules}

\subsubsection{General rules}

\section{LEAP-SECOND directive}

\begin{syntax}
  \pending{\directiveindicator\key{LEAP-SECOND}}
\end{syntax}

\subsubsection{Syntax rules}

\subsubsection{General rules}

\section{LISTING directive}

\begin{syntax}
  \directiveindicator\key{LISTING}
  \begin{1=}
    \key{ON} \\
    \key{OFF}
  \end{1=}
\end{syntax}

\subsubsection{Syntax rules}

\subsubsection{General rules}

\section{@OPTIONS directive}

\begin{syntax}[\miscextcolour]
  \pending{
    \key{@OPTIONS}
    \begin{0-1}
      \metaelement{options-text}
    \end{0-1}
  }
\end{syntax}

\subsubsection{Syntax rules}

\subsubsection{General rules}

\section{PAGE directive}

\begin{syntax}
  \directiveindicator\key{PAGE}
  \begin{0-1}
    \metaelement{comment-text}
  \end{0-1}
\end{syntax}

\subsubsection{Syntax rules}

\subsubsection{General rules}

\section{PROCESS statement}

\begin{syntax}[\miscextcolour]
  \pending{\key{PROCESS}}
\end{syntax}

\subsubsection{Syntax rules}

\subsubsection{General rules}

\section{REPLACE statement}

\format{on}
\begin{syntax}
  \key{REPLACE}
  \begin{0-1}
    \key{ALSO}
  \end{0-1}
  \begin{1=}
    \begin{1=}
      == \pseudotext == \\
      \identifier
    \end{1=}
    \key{BY}
    \begin{1=}
      == \pseudotext == \\
      \identifier
    \end{1=} \\

    \begin{1=}
      \key{LEADING} \\
      \key{TRAILING}
    \end{1=}
    == \metaelement{partial-word-1} ==
    \key{BY}
    == \metaelement{partial-word-2} ==
  \end{1=}\ldots .
\end{syntax}

\format{off}
\begin{syntax}
  \key{REPLACE}
  \begin{0-1}
    \key{LAST}
  \end{0-1}
  \key{OFF}.
\end{syntax}

\subsubsection{Syntax rules}

\subsubsection{General rules}

\section{SET directive}

\begin{syntax}[\miscextcolour]
  \begin{1=}
    \gnucobol{\directiveindicator} \\
    \textdollar
  \end{1=}
  \key{SET}
  \begin{1=}
    \gnucobol{
      \key{CONSTANT} \metaelement{compilation-variable-1} \literal
    } \\

    \gnucobol{
      \key{XFD} \literal
    } \\

    \gnucobol{
      \metaelement{compilation-variable-2}
      \begin{0-1}
        \literal
      \end{0-1}
    } \\

    \metaelement{micro-focus-directive}
  \end{1=}
\end{syntax}

\subsubsection{Syntax rules}

\subsubsection{General rules}

\section{SKIP1 statement}

\begin{syntax}[\miscextcolour]
  \pending{\key{SKIP1}}
\end{syntax}

\subsubsection{Syntax rules}

\subsubsection{General rules}

\section{SKIP2 statement}

\begin{syntax}[\miscextcolour]
  \pending{\key{SKIP2}}
\end{syntax}

\subsubsection{Syntax rules}

\subsubsection{General rules}

\section{SKIP3 statement}

\begin{syntax}[\miscextcolour]
  \pending{\key{SKIP3}}
\end{syntax}

\subsubsection{Syntax rules}

\subsubsection{General rules}

\section{SOURCE directive}

\begin{syntax}
  \directiveindicator\key{SOURCE} FORMAT IS
  \begin{1=}
    \key{FIXED} \\
    \key{FREE} \\
    \gnucobol{\key{VARIABLE}}
  \end{1=}
\end{syntax}

\subsubsection{Syntax rules}

\subsubsection{General rules}

\section{TITLE statement}

\begin{syntax}[\miscextcolour]
  \key{TITLE} \literal
\end{syntax}

\subsubsection{Syntax rules}

\subsubsection{General rules}

\section{TURN directive}

\begin{syntax}
  \directiveindicator\key{TURN}
  \begin{1=}
    exception-name-1
  \end{1=} \ldots
  \key{CHECKING}
  \begin{1=}
    \key{ON}
    \begin{0-1}
      WITH \key{LOCATION}
    \end{0-1} \\

    \key{OFF}
  \end{1=}
\end{syntax}

\subsubsection{Syntax rules}

\subsubsection{General rules}

\section{Micro Focus directives}

\subsection{ADDRSV directive}

\begin{syntax}[\miscextcolour]
  \begin{1=}
    \key{ADDRSV} \\
    \key{ADD-RSV}
  \end{1=}
  \literal \dots % TO-DO: Replace with MF-directive-literal
\end{syntax}

\subsubsection{Syntax rules}

\subsubsection{General rules}

\subsection{ADDSYN directive}

\begin{syntax}[\miscextcolour]
  \begin{1=}
    \key{ADDSYN} \\
    \key{ADD-SYN}
  \end{1=}
  \literal = \literal
\end{syntax}

\subsubsection{Syntax rules}

\subsubsection{General rules}

\subsection{ASSIGN directive}

\begin{syntax}[\miscextcolour]
  \key{ASSIGN}
  \begin{1=}
    "EXTERNAL" \\
    "DYNAMIC"
  \end{1=}
\end{syntax}

\subsubsection{Syntax rules}

\subsubsection{General rules}

\subsection{BOUND directive}

\format{enable}
\begin{syntax}[\miscextcolour]
  \key{BOUND} \\
\end{syntax}

\format{disable}
\begin{syntax}[\miscextcolour]
  \begin{1=}
    \key{NOBOUND} \\
    \key{NO-BOUND}
  \end{1=}
\end{syntax}

\subsubsection{Syntax rules}

\subsubsection{General rules}

\subsection{CALLFH directive}

\begin{syntax}[\miscextcolour]
  \key{CALLFH}
  \begin{0-1}
    \literal
  \end{0-1}
\end{syntax}

\subsubsection{Syntax rules}

\subsubsection{General rules}

\subsection{COMP1 directive}

\begin{syntax}[\miscextcolour]
  \begin{1=}
    \key{COMP1} \\
    \key{COMP-1}
  \end{1=}
  \begin{1=}
    "BINARY" \\
    "FLOAT"
  \end{1=}
\end{syntax}

\subsubsection{Syntax rules}

\subsubsection{General rules}

\subsection{FOLDCOPYNAME directive}

\format{enable}
\begin{syntax}[\miscextcolour]
  \begin{1=}
    \key{FOLDCOPYNAME} \\
    \key{FOLD-COPY-NAME}
  \end{1=}
  \gnucobol{AS}
  \begin{1=}
    "UPPER" \\
    "LOWER"
  \end{1=} 
\end{syntax}

\format{disable}
\begin{syntax}[\miscextcolour]
  \begin{1=}
    \key{NOFOLDCOPYNAME} \\
    \key{NOFOLD-COPY-NAME} \\
    \key{NO-FOLD-COPY-NAME}
  \end{1=}
\end{syntax}

\subsubsection{Syntax rules}

\subsubsection{General rules}

\subsection{MAKESYN directive}

\begin{syntax}[\miscextcolour]
  \begin{1=}
    \key{MAKESYN} \\
    \key{MAKE-SYN}
  \end{1=}
  \literal = \literal
\end{syntax}

\subsubsection{Syntax rules}

\subsubsection{General rules}

\subsection{OVERRIDE directive}

\begin{syntax}[\miscextcolour]
  \key{OVERRIDE}
  \begin{1=}
    \literal = \literal
  \end{1=} \dots
\end{syntax}

\subsubsection{Syntax rules}

\subsubsection{General rules}

\subsection{REMOVE directive}

\begin{syntax}[\miscextcolour]
  \key{REMOVE} \literal \dots
\end{syntax}

\subsubsection{Syntax rules}

\subsubsection{General rules}

\subsection{SOURCEFORMAT directive}

\begin{syntax}[\miscextcolour]
  \begin{1=}
    \key{SOURCEFORMAT} \\
    \key{SOURCE-FORMAT}
  \end{1=}
  \begin{1=}
    "FIXED" \\
    "FREE" \\
    "VARIABLE"
  \end{1=}
\end{syntax}

\subsubsection{Syntax rules}

\subsubsection{General rules}

\subsection{SSRANGE directive}

\format{enable}
\begin{syntax}[\miscextcolour]
  \key{SSRANGE}
  \begin{0-1}
    \pending{"1"} \\
    "2" \\
    \pending{"3"}
  \end{0-1}
\end{syntax}

\format{disable}
\begin{syntax}[\miscextcolour]
  \begin{1=}
    \key{NOSSRANGE} \\
    \key{NO-SSRANGE}
  \end{1=}
\end{syntax}

\subsubsection{Syntax rules}

\subsubsection{General rules}

\section{Predefined compilation variables}

GnuCOBOL defines compilation variables when certain conditions are true. If the condition associated with a variable is false, the variable is not defined.

\begin{centering}
  \begin{longtable}[!h]{p{0.2\textwidth} p{0.7\textwidth}}
    \toprule
    \textbf{Name} & \textbf{Condition} \\

    \midrule
    DEBUG & The -d debug flag is specified. \\

    EXECUTABLE & The module being compiled contains the main program. \\

    GCCOMP & The size of a COMP item is determined according to the GnuCOBOL scheme, where for a PICTURE of length:
    \begin{itemize}
    \item 1--2, the item has 1 byte
    \item 3--4, the item has 2 bytes
    \item 5--9, the item has 4 bytes
    \item 10--18, the item has 8 bytes.
    \end{itemize} \\

    GNUCOBOL & GnuCOBOL is compiling the source unit. \\

    HOSTSIGNS & A \emph{signed} packed-decimal item's value may be considered NUMERIC if the sign has value X"F". \\

    IBMCOMP & The size of a COMP item is determined according to the IBM scheme, where for a PICTURE of length:
    \begin{itemize}
    \item 1--4, the item has 2 bytes
    \item 5--9, the item has 4 bytes
    \item 10--18, the item has 8 bytes.
    \end{itemize} \\

    MODULE & The module being compiled does not contain the main program. \\

    NOHOSTSIGNS & A \emph{signed} packed-decimal item's value may not be considered NUMERIC if the sign has value X"F". \\

    NOIBMCOMP & The size of a COMP item is not determined according to the IBM scheme. \\

    NOSTICKY-LINKAGE & Sticky-linkage (linkage-section items remaining allocated between invocations) is not enabled. \\

    NOTRUNC & Numeric data items are truncated according to their internal representation. \\

    OCCOMP & The size of a COMP item is determined according to the GnuCOBOL scheme, where for a PICTURE of length:
    \begin{itemize}
    \item 1--2, the item has 1 byte
    \item 3--4, the item has 2 bytes
    \item 5--9, the item has 4 bytes
    \item 10--18, the item has 8 bytes.
    \end{itemize} \\

    OPENCOBOL & GnuCOBOL is compiling the source unit. \\

    P64 & Pointers are greater than 32 bits long. \\

    STICKY-LINKAGE & Sticky-linkage (linkage-section items remaining allocated between invocations) is enabled. \\

    TRUNC & Numeric data items are truncated according to their PICTURE clauses. \\
    \bottomrule
  \end{longtable}
\end{centering}


%%% Local Variables:
%%% mode: latex
%%% TeX-master: "grammar.tex"
%%% End:

\chapter{Compilation group}

\begin{0-1}
  program-definition \\
  function-definition
\end{0-1} \ldots

where program-definition is\vspace{1em}\newline
\begin{0-1}
  \begin{1=}
    \key{IDENTIFICATION} \\
    \miscext{\key{ID}}
  \end{1=}
  \key{DIVISION}.
\end{0-1} \newline
\key{PROGRAM-ID}.
\begin{1=}
  program-name-1 \\
  \miscext{\literal}
\end{1=}
\begin{0-1} \key{AS} \literal \end{0-1}
\begin{0-1} IS
  \begin{1=}
    \begin{1+}
      \key{COMMON} \\

      \begin{1=}
        \key{INITIAL} \\
        \key{RECURSIVE}
      \end{1=}
    \end{1+} \\

    \miscext{\pending{\key{EXTERNAL}}}
  \end{1=}
  PROGRAM
\end{0-1}. \newline
\deleted{
  \begin{0-1}
    comment-paragraphs
  \end{0-1}
} \newline
\begin{0-1}
  environment-division
\end{0-1} \newline
\begin{0-1}
  data-division
\end{0-1} \newline
\begin{0-1}
  procedure-division
  \begin{0-1}
    program-definition
  \end{0-1} \ldots
\end{0-1} \newline
\begin{0-1}
  \key{END} \key{PROGRAM}
  \begin{1=}
    program-name-1 \\
    \miscext{literal-1}
  \end{1=}.
\end{0-1}

where function-definition is\vspace{1em}\newline
\begin{0-1}
  \begin{1=}
    \key{IDENTIFICATION} \\
    \miscext{\key{ID}}
  \end{1=}
  \key{DIVISION}.
\end{0-1} \newline
\key{FUNCTION-ID}.
\begin{1=}
  program-name-1 \\
  \miscext{\literal}
\end{1=}
\begin{0-1} \key{AS} \literal \end{0-1}.\newline
\deleted{
  \begin{0-1}
    comment-paragraphs
  \end{0-1}
} \newline
\begin{0-1}
  environment-division
\end{0-1} \newline
\begin{0-1}
  data-division
\end{0-1} \newline
\begin{0-1}
  procedure-division
\end{0-1} \newline
\key{END} \key{FUNCTION}
\begin{1=}
  program-name-1 \\
  \miscext{literal-1}
\end{1=}.

%%% Local Variables:
%%% mode: latex
%%% TeX-master: "grammar.tex"
%%% End:
\chapter{Identification division}

\newcommand{\commenttext}{\metaelement{comment-text}}

\begin{syntax}
  \begin{0-1}
    \begin{1=}
      \key{IDENTIFICATION} \\
      \miscext{\key{ID}}
    \end{1=}
    DIVISION.
  \end{0-1}
  \newline
  \begin{1=}
    \metaelement{function-id-paragraph} \\
    \metaelement{program-id-paragraph}
  \end{1=}
  \newline
  \deleted{
    \begin{0+}
      \key{AUTHOR}. \commenttext. \\
      \key{DATE-WRITTEN}. \commenttext. \\
      \key{DATE-MODIFIED}. \commenttext. \\
      \key{DATE-COMPILED}. \commenttext. \\
      \key{INSTALLATION}. \commenttext. \\
      \key{REMARKS}. \commenttext. \\
      \key{SECURITY}. \commenttext. \\
    \end{0+} \gnucobol{\ldots}
  }
\end{syntax}

\subsubsection{Syntax rules}

\subsubsection{General rules}

\section{PROGRAM-ID paragraph}

\begin{syntax}
  \key{PROGRAM-ID}.
  \begin{1=}
    \metaelement{program-name-1} \\
    \literal
  \end{1=}
  \begin{0-1} \key{AS} \literal \end{0-1}

  \begin{0-1} IS
    \begin{1=}
      \begin{1+}
        \key{COMMON} \\

        \begin{1=}
          \key{INITIAL} \\
          \key{RECURSIVE}
        \end{1=}
      \end{1+} \\

      \miscext{\pending{\key{EXTERNAL}}}
    \end{1=}
    PROGRAM
  \end{0-1}.
\end{syntax}

\subsubsection{Syntax rules}

\subsubsection{General rules}

\section{FUNCTION-ID paragraph}

\begin{syntax}
  \key{FUNCTION-ID}.
  \begin{1=}
    \functionname \\
    \literal
  \end{1=}
  \begin{0-1} \key{AS} \literal \end{0-1}.
\end{syntax}

\subsubsection{Syntax rules}

\subsubsection{General rules}


%%% Local Variables:
%%% mode: latex
%%% TeX-master: "grammar.tex"
%%% End:

\chapter{Environment division}
\begin{syntax}
  \begin{0-1}
    \key{ENVIRONMENT} \key{DIVISION}.
  \end{0-1}
  \newline
  \begin{0-1}
    \metaelement{configuration-section}
  \end{0-1}
  \newline
  \begin{0-1}
    \metaelement{input-output-section}
  \end{0-1}
\end{syntax}

\subsubsection{Syntax rules}

\subsubsection{General rules}

\section{Configuration section}

\format{standard}

\begin{syntax}
  \key{CONFIGURATION} \key{SECTION}.
  \newline
  \begin{0-1}
    \metaelement{source-computer-paragraph}
  \end{0-1}
  \newline
  \begin{0-1}
    \metaelement{object-computer-paragraph}
  \end{0-1}
  \newline
  \begin{0-1}
    \metaelement{special-names-header}
    \begin{0-1}
      \metaelement{special-names-entry}
    \end{0-1} \ldots
  \end{0-1}
  \newline
  \begin{0-1}
    \metaelement{repository-paragraph}
  \end{0-1}
\end{syntax}

\format{micro-focus-and-gnucobol}

\begin{syntax}[\miscextcolour]
  \begin{0-1}
    \key{CONFIGURATION} \key{SECTION}.
  \end{0-1}  
  \newline
  \begin{0+}
    \metaelement{source-computer-paragraph} \\
    \metaelement{object-computer-paragraph} \\
    \metaelement{special-names-header} \\
    \metaelement{special-names-entry} \\
    \metaelement{repository-paragraph}
  \end{0+}
\end{syntax}

\subsubsection{Syntax rules}

\subsubsection{General rules}

\subsection{SOURCE-COMPUTER paragraph}

The SOURCE-COMPUTER paragraph identifies the computer on which the compilation unit should be compiled.

\begin{syntax}
  \key{SOURCE-COMPUTER}.
  \begin{0-1}
    \begin{1=}
      \computername
    \end{1=}\gnucobol{\ldots}
    \deleted{
      \begin{0-1}
        WITH \key{DEBUGGING} \key{MODE}
      \end{0-1}
    }
    .
  \end{0-1}
\end{syntax}

\subsubsection{Syntax rules}

\subsubsection{General rules}

\subsection{OBJECT-COMPUTER paragraph}

The OBJECT-COMPUTER paragraph identifies the computer on which the runtime module should be run.

\begin{syntax}
  \key{OBJECT-COMPUTER}.

  \begin{0-1}
    \begin{0-1}
      \begin{1=}
        \computername
      \end{1=}\gnucobol{\ldots}
    \end{0-1}\\\quad
    \begin{0-1}
      \deleted{
        \key{MEMORY} SIZE IS \integer
        \begin{1=}
          \key{CHARACTERS} \\
          \key{WORDS} \\
          \key{MODULES}
        \end{1=}
      } \\
      PROGRAM COLLATING \key{SEQUENCE} IS \metaelement{collating-sequence-1} \\
      \deleted{\key{SEGMENT-LIMIT} IS \integer} \\
      CHARACTER \key{CLASSIFICATION} IS
      \begin{1=}
        \metaelement{locale-name-1} \\
        \key{LOCALE} \\
        \key{SYSTEM-DEFAULT} \\
        \key{USER-DEFAULT}
      \end{1=}
    \end{0-1}\ldots\ {}.
  \end{0-1}
\end{syntax}

\subsubsection{Syntax rules}

\subsubsection{General rules}

\subsection{SPECIAL-NAMES paragraph}

\begin{syntax}
  \begin{0-1}
    \key{SPECIAL-NAMES}.
  \end{0-1}

  \begin{0-1}
    \begin{1=}
      \defdmetaelement{mnemonic-name-clause} \\
      \defdmetaelement{alphabet-name-clause} \\
      \defdmetaelement{symbolic-characters-clause} \\
      \key{LOCALE} \metaelement{locale-name-1} IS \literal \\
      \defdmetaelement{class-clause} \\

      \key{CURRENCY} SIGN IS \literal
      \begin{0-1}
        \pending{WITH \key{PICTURE-SYMBOL} \literal}
      \end{0-1} \\

      \key{DECIMAL-POINT} IS \key{COMMA} \\
      \miscext{\key{NUMERIC} \key{SIGN} IS \key{TRAILING} \key{SEPARATE}} \\
      \key{CURSOR} IS \identifier \\
      \key{CRT} \key{STATUS} IS \identifier \\
      \miscext{\pending{\key{SCREEN-CONTROL} IS \identifier}} \\
      \miscext{\pending{\key{EVENT-STATUS} IS \identifier}}
    \end{1=}\ldots\ {}.
  \end{0-1}\ldots
\end{syntax}

where \defnmetaelement{mnemonic-name-clause} is

\begin{syntax}
  \mnemonicname
  \begin{1=}
    IS \key{CRT} \\
    \integer IS \systemname \\
    \begin{0-1}
      IS \switchname
    \end{0-1}
    \begin{1+}
      \key{ON} STATUS IS \switchstatusname \\
      \key{OFF} STATUS IS \switchstatusname
    \end{1+}
  \end{1=}
\end{syntax}

where \defnmetaelement{alphabet-name-clause} is

\begin{syntax}
  \key{ALPHABET} \metaelement{alphabet-name-1} IS
  \begin{1=}
    \key{ASCII} \\
    \key{EBCDIC} \\
    \key{NATIVE} \\
    \key{STANDARD-1} \\
    \key{STANDARD-2} \\
    \begin{1=}
      \literal
      \begin{0-1}
        \begin{1=}
          \key{THROUGH} \\
          \key{THRU}
        \end{1=}
        \literal \\
        \begin{1=}
          \key{ALSO} \literal
        \end{1=}\ldots
      \end{0-1}
    \end{1=}\ldots
  \end{1=}
\end{syntax}

where \defnmetaelement{symbolic-characters-clause} is

\begin{syntax}
  \key{SYMBOLIC} CHARACTERS
  \begin{1=}
    \begin{1=}
      \metaelement{symbolic-character-name-1}
    \end{1=}\ldots
    \begin{1=}
      \key{IS} \\
      \key{ARE}
    \end{1=}
    \begin{1=}
      \integer
    \end{1=}\ldots
  \end{1=}\ldots
  \begin{0-1}
    \key{IN} \key{WORD}
  \end{0-1}
\end{syntax}

where \defnmetaelement{class-clause} is

\begin{syntax}
  \key{CLASS}
  \begin{0-1}
    FOR \key{ALPHANUMERIC} \\
    \pending{FOR \key{NATIONAL}} \\
  \end{0-1}
  \metaelement{class-name-1} IS
  \begin{1=}
    \literal
    \begin{0-1}
      \begin{1=}
        \key{THRU} \\
        \key{THROUGH}
      \end{1=}
      \literal
    \end{0-1}
  \end{1=}\ldots
  
  \pending{
    \begin{0-1}
      \key{IN} \metaelement{alphabet-name-1}
    \end{0-1}
  }
\end{syntax}

\subsubsection{Syntax rules}

\subsubsection{General rules}

\subsection{REPOSITORY paragraph}

\begin{syntax}
  \key{REPOSITORY}.

  \begin{0-1}
    \begin{1=}
      \key{FUNCTION}
      \begin{1=}
        \begin{1=}
          \functionname
        \end{1=}\ldots \\

        \key{ALL}
      \end{1=}
      \key{INTRINSIC} \\

      \key{FUNCTION} \functionname
      \begin{0-1}
        \key{AS} \literal
      \end{0-1} \\

      \key{PROGRAM} \programname
      \begin{0-1}
        \key{AS} \literal
      \end{0-1}
    \end{1=}\ldots\ {}.
  \end{0-1}
\end{syntax}

\subsubsection{Syntax rules}

\subsubsection{General rules}

\section{Input-output section}

\begin{syntax}
  \begin{0-1}
    \key{INPUT-OUTPUT} \key{SECTION}.
  \end{0-1}\newline
  \begin{0-1}
    \metaelement{file-control-paragraph}
  \end{0-1}\newline
  \begin{0-1}
    \metaelement{i-o-control-paragraph}
  \end{0-1}
\end{syntax}

\subsubsection{Syntax rules}

\subsubsection{General rules}

\subsection{FILE-CONTROL paragraph}

\begin{syntax}
  \begin{0-1}
    \key{FILE-CONTROL}.
  \end{0-1}\newline
  \begin{0-1}
    \defdmetaelement{file-control-entry}
  \end{0-1} \ldots
\end{syntax}

where \defnmetaelement{file-control-entry} is

\begin{syntax}
  \key{SELECT}
  \begin{0-1}
    \key{OPTIONAL} \\
    \miscext{\key{NOT} \key{OPTIONAL}}
  \end{0-1}
  \filename
  \begin{0-1}
    \defdmetaelement{assign-clause} \\
    \defdmetaelement{access-mode-clause} \\
    \defdmetaelement{alternative-record-key-clause} \\
    \defdmetaelement{collating-sequence-clause} \\
    \defdmetaelement{file-status-clause} \\
    \defdmetaelement{lock-mode-clause} \\
    \defdmetaelement{organization-clause} \\
    \defdmetaelement{padding-character-clause} \\
    \defdmetaelement{record-delimiter-clause} \\
    \defdmetaelement{record-key-clause} \\
    \defdmetaelement{relative-key-clause} \\
    \defdmetaelement{reserve-clause} \\
    \defdmetaelement{sharing-clause}
  \end{0-1}\ldots\ {}.
\end{syntax}

where \defnmetaelement{assign-clause} is

\ {}\newline
\begin{syntax}
  \key{ASSIGN}
  \begin{0-1}
    \key{TO} \\
    \key{USING}
  \end{0-1}
  \miscext{
    \begin{0-1}
      \key{DYNAMIC} \\
      \key{EXTERNAL}
    \end{0-1}
  }
  \begin{1=}
    \miscext{
      \begin{0-1}
        \key{LINE} \key{ADVANCING} FILE
      \end{0-1}
    }
    \begin{1=}
      \literal \\
      \identifier
    \end{1=}\\

    \begin{1=}
      \miscext{\key{CARD-PUNCH}} \\
      \miscext{\key{CARD-READER}} \\
      \miscext{\key{CASSETTE}} \\
      \key{DISC} \\
      \xopen{\key{DISK}} \\
      \miscext{\key{DISPLAY}} \\
      \miscext{\key{INPUT}} \\
      \miscext{\key{INPUT-OUTPUT}} \\
      \miscext{\key{KEYBOARD}} \\
      \miscext{\key{MAGNETIC-TAPE}} \\
      \miscext{\key{OUTPUT}} \\
      \miscext{\key{PRINTER-1}} \\
      \xopen{\key{PRINTER}} \\
      \key{PRINT} \\
      \key{RANDOM} \\
      \key{TAPE}
    \end{1=}
    \begin{0-1}
      \literal \\
      \identifier
    \end{0-1}
  \end{1=}
\end{syntax}

where \defnmetaelement{access-mode-clause} is

\begin{syntax}
  \key{ACCESS} MODE IS
  \begin{1=}
    \key{SEQUENTIAL} \\
    \key{DYNAMIC} \\
    \key{RANDOM}
  \end{1=}
\end{syntax}

where \defnmetaelement{alternative-record-key-clause} is

\begin{syntax}
  \key{ALTERNATE} RECORD KEY IS \identifier
  \pending{
    \begin{0-1}
      \begin{1=}
        = \\
        \key{SOURCE} IS \\
      \end{1=}
      \begin{1=}
        \identifier
      \end{1=}\ldots
    \end{0-1}
  }
  \begin{0-1}
    WITH \key{DUPLICATES}
  \end{0-1}
  \pending{
    \begin{0-1}
      \key{SUPPRESS} \key{WHEN}
      \begin{1=}
        \key{ALL} \literal \\
        \key{SPACE} \\
        \key{ZERO}
      \end{1=}
    \end{0-1}
  }
\end{syntax}

where \defnmetaelement{collating-sequence-clause} is

\begin{syntax}
  \pending{COLLATING \key{SEQUENCE} IS \metaelement{collating-sequence-name-1}}
\end{syntax}

where \defnmetaelement{file-status-clause} is

\begin{syntax}
  \begin{0-1}
    \key{FILE} \\
    \key{SORT}
  \end{0-1}
  \key{STATUS} IS \identifier
\end{syntax}

where \defnmetaelement{lock-mode-clause} is

\begin{syntax}
  \key{LOCK} MODE IS
  \begin{1=}
    \begin{1=}
      \begin{1=}
        \key{MANUAL} \\
        \key{AUTOMATIC}
      \end{1=}
      \begin{0-1}
        \key{WITH} \key{LOCK} \key{ON}
        \begin{0-1}
          \key{MULTIPLE}
        \end{0-1}
        \begin{1=}
          \key{RECORD} \\
          \key{RECORDS}
        \end{1=} \\
        \miscext{\pending{\key{WITH }\key{ROLLBACK}}}
      \end{0-1}
    \end{1=} \\
    \miscext{\key{EXCLUSIVE}}
  \end{1=}
\end{syntax}

where \defnmetaelement{organization-clause} is

\begin{syntax}
  \begin{0-1}
    \begin{1=}
      \key{ORGANIZATION} \\
      \miscext{\key{ORGANISATION}}
    \end{1=}
    IS
  \end{0-1}
  \begin{1=}
    \key{INDEXED} \\
    \xopen{\key{LINE} \key{SEQUENTIAL}} \\
    \miscext{RECORD BINARY} \key{SEQUENTIAL} \\
    \key{RELATIVE}
  \end{1=}
\end{syntax}

where \defnmetaelement{padding-character-clause} is

\begin{syntax}
  \pending{
    \key{PADDING} CHARACTER IS
    \begin{1=}
      \identifier \\
      \literal
    \end{1=}
  }
\end{syntax}

where \defnmetaelement{record-delimiter-clause} is

\begin{syntax}
  \key{RECORD} \key{DELIMITER} IS
  \begin{1=}
    \pending{\key{STANDARD-1}} \\
    \miscext{\key{LINE-SEQUENTIAL}} \\
    \miscext{\key{BINARY-SEQUENTIAL}}
  \end{1=}
\end{syntax}

where \defnmetaelement{record-key-clause} is

\begin{syntax}
  \key{RECORD} KEY IS \identifier
  \begin{0-1}
    \begin{1=}
      = \\
      \key{SOURCE} IS
    \end{1=}
    \begin{1=}
      \identifier
    \end{1=}\ldots
  \end{0-1}
\end{syntax}

where \defnmetaelement{relative-key-clause} is

\begin{syntax}
  \key{RELATIVE} KEY IS \identifier
\end{syntax}

where \defnmetaelement{reserve-clause} is

\begin{syntax}
  \pending{
    \key{RESERVE}
    \begin{1=}
      \key{NO} \\
      \integer
    \end{1=}
    \begin{0-1}
      AREA \\
      AREAS
    \end{0-1}
  }
\end{syntax}

where \defnmetaelement{sharing-clause} is

\begin{syntax}
  \key{SHARING} WITH
  \begin{1=}
    \key{ALL} OTHER \\
    \key{NO} OTHER \\
    \key{READ} \key{ONLY}
  \end{1=}
\end{syntax}

\subsubsection{Syntax rules}

\subsubsection{General rules}

\subsection{I-O-CONTROL paragraph}
\begin{syntax}
  \begin{0-1}
    \key{I-O-CONTROL}.
  \end{0-1}\newline
  \begin{1=}
    \key{SAME}
    \begin{0-1}
      \key{RECORD} \\
      \key{SORT} \\
      \key{SORT-MERGE}
    \end{0-1}
    AREA FOR
    \begin{1=}
      \filename
    \end{1=}\ldots \\

    \deleted{
      \key{MULTIPLE} FILE TAPE CONTAINS
      \begin{1=}
        \filename
        \begin{0-1}
          \key{POSITION} \integer
        \end{0-1}
      \end{1=}\ldots
    }
  \end{1=}\ldots\ {}.
\end{syntax}

\subsubsection{Syntax rules}

\subsubsection{General rules}

%%% Local Variables:
%%% mode: latex
%%% TeX-master: "grammar.tex"
%%% End:

\chapter{Data division}

\begin{syntax}
  \begin{0-1}
    \key{DATA} \key{DIVISION}.
  \end{0-1}\newline
  \begin{0-1}
    \metaelement{file-section}
  \end{0-1}\newline
  \begin{0-1}
    \metaelement{working-storage-section}
  \end{0-1}\newline
  \pending{
    \deleted{
      \begin{0-1}
        \metaelement{communication-section}
      \end{0-1}
    }
  }\newline
  \begin{0-1}
    \metaelement{local-storage-section}
  \end{0-1}\newline
  \begin{0-1}
    \metaelement{report-section}
  \end{0-1}\newline
  \begin{0-1}
    \metaelement{screen-section}
  \end{0-1}
\end{syntax}

\subsubsection{Syntax rules}

\subsubsection{General rules}

\section{File section}

\begin{syntax}
  \begin{0-1}
    \key{FILE} \key{SECTION}.
  \end{0-1}\newline
  \begin{1=}
    \metaelement{file-description-entry}
    \begin{1=}
      \metaelement{record-description} \\
      \metaelement{constant-definition}
    \end{1=}\ldots
  \end{1=}\ldots
\end{syntax}

\subsubsection{Syntax rules}

\subsubsection{General rules}

\subsection{File description entry}

\begin{syntax}
  \begin{1=}
    \key{FD} \\
    \key{SD}
  \end{1=}
  \filename
  \begin{0-1}
    \metaelement{block-clause} \\

    \pending{
      \key{CODE-SET} IS \metaelement{alphabet-name-1}
      \begin{0-1}
        \key{FOR}
        \begin{1=}
          \identifier
        \end{1=} \ldots
      \end{0-1}
    } \\

    \deleted{
      \key{DATA}
      \begin{1=}
        \key{RECORD} \\
        \key{RECORDS}
      \end{1=}
      \begin{0-1}
        IS \\
        ARE
      \end{0-1}
      \begin{1=}
        \identifier
      \end{1=}\ldots
    } \\

    IS \key{EXTERNAL} \\
    IS \key{GLOBAL} \\

    \deleted{
      \key{LABEL}
      \begin{1=}
        \key{RECORD} \\
        \key{RECORDS}
      \end{1=}
      \begin{0-1}
        IS \\
        ARE
      \end{0-1}
      \begin{1=}
        \key{STANDARD} \\
        \key{OMITTED}
      \end{1=}
    } \\

    \metaelement{linage-clause} \\

    \deleted{
      \key{RECORDING} MODE IS
      \begin{1=}
        \begin{1=}
          \key{F} \\
          \key{FIXED} \\
        \end{1=} \\

        \begin{1=}
          \key{V} \\
          \key{VARIABLE} \\
        \end{1=} \\

        \key{U} \\
        \key{S}
      \end{1=}
    } \\

    \pending{
      \begin{1=}
        \key{REPORT} IS \\
        \key{REPORTS} ARE
      \end{1=}
      \begin{1=}
        \identifier
      \end{1=}\ldots
    } \\

    \deleted{
      \key{VALUE} \key{OF}
      \begin{1=}
        \key{FILE-ID} \\
        \key{ID} \\
        \identifier
      \end{1=}
      IS
      \begin{1=}
        \literal \\
        \identifier
      \end{1=}
    } \\

    \metaelement{record-clause}
  \end{0-1}\ldots\ {}.
\end{syntax}

\subsubsection{Syntax rules}

\subsubsection{General rules}

\section{Working-storage section}
\begin{syntax}
  \key{WORKING-STORAGE} \key{SECTION}.\newline
  \begin{0-1}
    \metaelement{constant-definition} \\
    \metaelement{record-description}
  \end{0-1}\ldots
\end{syntax}

\subsubsection{Syntax rules}

\subsubsection{General rules}

\section{Communication section}
\begin{syntax}[\deletedcolour]
  \pending{
    \key{COMMUNICATION} \key{SECTION}.\newline
    \begin{0-1}
      \metaelement{communication-description-entry}
      \begin{0-1}
        \metaelement{record-description} \\
        \metaelement{constant-definition}
      \end{0-1}\ldots
    \end{0-1}\ldots
  }
\end{syntax}

\subsection{Communication description entry}

% TO-DO: Format more neatly.
\format{input}
\begin{syntax}[\deletedcolour]
  \key{CD} \metaelement{entry-name} FOR
  \begin{0-1}
    \key{INITIAL}
  \end{0-1}
  \key{INPUT}
  \begin{0-1}
    \begin{1+}
      SYMBOLIC \key{QUEUE} IS \identifier \\
      SYMBOLIC \key{SUB-QUEUE-1} IS \identifier \\
      SYMBOLIC \key{SUB-QUEUE-2} IS \identifier \\
      SYMBOLIC \key{SUB-QUEUE-3} IS \identifier \\
      \key{MESSAGE} \key{DATE} IS \identifier \\
      \key{MESSAGE} \key{TIME} IS \identifier \\
      SYMBOLIC \key{SOURCE} IS \identifier \\
      \key{TEXT} \key{LENGTH} IS \identifier \\
      \key{END} \key{KEY} IS \identifier \\
      \key{STATUS} \key{KEY} IS \identifier \\
      MESSAGE \key{COUNT} IS \identifier
    \end{1+} \\

    \hspace{1em}
    \begin{minipage}[!h]{0.4\textwidth}
      \hspace{-2.15em} \identifier \identifier \identifier \identifier \identifier \identifier \identifier \identifier \identifier \identifier \identifier
    \end{minipage}
  \end{0-1}
  .
\end{syntax}

\format{output}
\begin{syntax}[\deletedcolour]
  \key{CD} \metaelement{entry-name} FOR \key{OUTPUT}
  \begin{0+}
    \key{DESTINATION} \key{COUNT} IS \identifier \\
    \key{TEXT} \key{LENGTH} IS \identifier \\
    \key{STATUS} \key{KEY} IS \identifier \\
    \key{DESTINATION} \key{TABLE} \key{OCCURS} \integer TIMES
    \begin{0-1}
      \key{INDEXED} BY
      \begin{1=}
        \cobolindexname
      \end{1=}\ldots
    \end{0-1} \\
    \key{ERROR} \key{KEY} IS \identifier \\
    \key{DESTINATION} IS \identifier \\
    \key{SYMBOLIC} \key{DESTINATION} IS \identifier
  \end{0+}
  .
\end{syntax}

\format{I-O}
\begin{syntax}[\deletedcolour]
  \key{CD} \metaelement{entry-name} FOR INITIAL \key{I-O}
  \begin{0-1}
    \begin{1+}
      \key{MESSAGE} \key{DATE} IS \identifier \\
      \key{MESSAGE} \key{TIME} IS \identifier \\
      SYMBOLIC \key{TERMINAL} IS \identifier \\
      \key{TEXT} \key{LENGTH} IS \identifier \\
      \key{END} \key{KEY} IS \identifier \\
      \key{STATUS} \key{KEY} IS \identifier
    \end{1+} \\

    \identifier \identifier \identifier \identifier \identifier \identifier
  \end{0-1}
  .
\end{syntax}

\section{Local-storage section}
\begin{syntax}
  \key{LOCAL-STORAGE} \key{SECTION}.\newline
  \begin{0-1}
    \metaelement{constant-definition} \\
    \metaelement{record-description}
  \end{0-1}\ldots
\end{syntax}

\subsubsection{Syntax rules}

\subsubsection{General rules}

\section{Linkage section}
\begin{syntax}
  \key{LINKAGE} \key{SECTION}.\newline
  \begin{0-1}
    \metaelement{constant-definition} \\
    \metaelement{record-description}
  \end{0-1}\ldots
\end{syntax}

\subsubsection{Syntax rules}

\subsubsection{General rules}

\section{Report section}
\begin{syntax}
  \key{REPORT} \key{SECTION}.\newline
  \begin{0-1}
    \metaelement{constant-definition} \\
    \metaelement{report-description}
  \end{0-1}\ldots
\end{syntax}

\subsubsection{Syntax rules}

\subsubsection{General rules}

\subsection{Report description}

\begin{syntax}
  \key{RD} \reportname

  \begin{0-1}
    IS \key{GLOBAL} \\

    % TO-DO: Check where WITH came from.
    \gnucobol{WITH} \key{CODE} IS
    \begin{1=}
      \identifier \\
      \literal
    \end{1=} \\

    \begin{1=}
      \key{CONTROL} \\
      \key{CONTROLS}
    \end{1=}
    \begin{0-1}
      IS \\
      ARE
    \end{0-1}
    \begin{1=}
      \begin{1=}
        \identifier
      \end{1=}\ldots \\

      \key{FINAL}
      \begin{0-1}
        \identifier
      \end{0-1}\ldots \\
    \end{1=} \\

    \defdmetaelement{page-limits-clause}
  \end{0-1}\ldots\ {}.\newline

  \begin{1=}
    \metaelement{report-group-description-1}
  \end{1=}\ldots
\end{syntax}

where \defnmetaelement{page-limits-clause} is

\begin{syntax}
  \key{PAGE}
  \begin{0-1}
    \key{LIMIT} \\
    \key{LIMITS}
  \end{0-1}
  \begin{0-1}
    IS \\
    ARE
  \end{0-1}
  \begin{1=}
    \integer \\
    \identifier
  \end{1=}
  \begin{1=}
    \key{LINE} \\
    \key{LINES}
  \end{1=}

  \begin{0-1}
    \begin{1=}
      \integer \\
      \identifier
    \end{1=}
    \begin{1=}
      \key{COLUMNS} \\
      \key{COLS}
    \end{1=}
  \end{0-1}

  \begin{0-1}
    \key{HEADING} IS
    \begin{1=}
      \integer \\
      \identifier
    \end{1=} \\

    \key{FIRST}
    \begin{1=}
      \key{DETAIL} \\
      \key{DE}
    \end{1=}
    IS
    \begin{1=}
      \integer \\
      \identifier
    \end{1=} \\

    \key{LAST}
    \begin{1=}
      \key{CONTROL} \key{HEADING} \\
      \key{CH}
    \end{1=}
    IS
    \begin{1=}
      \integer \\
      \identifier
    \end{1=} \\

    \key{LAST}
    \begin{1=}
      \key{DETAIL} \\
      \key{DE}
    \end{1=}
    IS
    \begin{1=}
      \integer \\
      \identifier
    \end{1=} \\

    \key{FOOTING} IS
    \begin{1=}
      \integer \\
      \identifier
    \end{1=} \\

    \miscext{
      \key{LINE} \key{LIMIT} IS
      \begin{1=}
        \integer \\
        \identifier
      \end{1=}
    }
  \end{0-1} \ldots
\end{syntax}

where \defnmetaelement{report-group-description} is

\begin{syntax}
  \metaelement{level-number} \metaelement{entry-name}
  \begin{0-1}
    \metaelement{blank-clause} \\
    \metaelement{column-clause} \\
    \key{GROUP} INDICATE \\
    \metaelement{justified-clause} \\
    \metaelement{line-clause} \\
    \metaelement{next-group-clause} \\
    \metaelement{picture-clause} \\
    \metaelement{present-when-clause} \\
    \metaelement{occurs-clause} \\
    \metaelement{sign-clause} \\
    \metaelement{source-clause} \\
    \metaelement{sum-clause} \\
    \metaelement{type-clause} \\
    \key{USAGE} IS \key{DISPLAY} \\
    \metaelement{value-clause} \\
    \metaelement{varying-clause}
  \end{0-1}\ldots\ {}.
\end{syntax}

\subsubsection{Syntax rules}

\subsubsection{General rules}

\section{Screen section}
\begin{syntax}
  \key{SCREEN} \key{SECTION}.\newline
  \begin{0-1}
    \metaelement{constant-definition} \\
    \metaelement{screen-description}
  \end{0-1}\ldots
\end{syntax}

\subsubsection{Syntax rules}

\subsubsection{General rules}

\subsection{Screen description}
\begin{syntax}
  \metaelement{level-number} \metaelement{entry-name}
  \miscext{
    \pending{
      \begin{0-1}
        \metaelement{control-definition}
        \begin{0-1}
          \metaelement{control-attributes}
        \end{0-1}\ldots
      \end{0-1}
    }
  }
  \begin{0-1}
    \metaelement{appearance-attribute-clauses} \\

    \begin{1=}
      \key{AUTO} \\
      \miscext{\key{AUTO-SKIP}} \\
      \miscext{\key{AUTOTERMINATE}} \\
    \end{1=} \\

    \metaelement{column-clause} \\
    \miscext{\pending{\metaelement{ccol-clause}}} \\
    \miscext{\pending{\metaelement{cline-clause}}} \\
    \miscext{\pending{\metaelement{csize-clause}}} \\

    \key{ERASE}
    \begin{1=}
      \key{EOL} \\
      \key{EOS} \\

      \begin{0-1}
        \miscext{TO} \key{END} OF
      \end{0-1}
      \begin{1=}
        \key{LINE} \\
        \key{SCREEN}
      \end{1=}
    \end{1=} \\

    \begin{1=}
      \key{FULL} \\
      \miscext{\key{LENGTH-CHECK}} \\
    \end{1=} \\

    IS \key{GLOBAL} \\
    \miscext{\pending{\key{GRID}}} \\

    \miscext{\key{INITIAL}} \\
    \miscext{\pending{\key{LEFTLINE}}} \\
    \metaelement{justified-clause} \\
    \metaelement{line-clause} \\

    \miscext{
      \begin{1=}
        \key{NO-ECHO} \\
        \key{NO} \key{ECHO} \\
        \key{OFF}
      \end{1=}
    } \\

    \metaelement{occurs-clause} \\
    \metaelement{picture-clause} \\

    \begin{1=}
      \key{REQUIRED} \\
      \miscext{\key{EMPTY-CHECK}}
    \end{1=} \\

    \metaelement{source-destination-clauses} \\

    \key{SECURE} \\
    \metaelement{sign-clause} \\
    \miscext{\pending{\metaelement{size-clause}}} \\
    \metaelement{usage-clause} \\
    \metaelement{value-clause}
  \end{0-1}\ldots\ {}.
\end{syntax}

% TO-DO: Move to own section and have separate control definition section.

% where \defnmetaelement{control-definition} is

% \begin{syntax}[\miscextcolour]
%   \begin{1=}
%     \key{ACTIVEX} \\
%     \key{BAR} \\
%     \key{BITMAP} \\
%     \key{CHECK-BOX} \\
%     \key{COMBO-BOX} \\
%     \key{DATE-ENTRY} \\
%     \key{ENTRY-FIELD} \\
%     \key{FRAME} \\
%     \key{LABEL} \\
%     \key{LIST-BOX} \\

%     \key{OBJECT}
%     \begin{1=}
%       \identifier \\
%       \integer
%     \end{1=} \\

%     \key{PUSH-BUTTON} \\
%     \key{RADIO-BUTTON} \\
%     \key{SCROLL-BAR} \\
%     \key{STATUS-BAR} \\
%     \key{TREE-VIEW} \\
%     \key{WEB-BROWSER}
%   \end{1=}
% \end{syntax}

% where \defnmetaelement{control-attributes} is

% \begin{syntax}[\miscextcolour]
%   \begin{1=}
%     \key{STYLE}
%     \begin{0-1}
%       IS \\
%       =
%     \end{0-1}
%     \begin{1=}
%       \identifier \\
%       \integer
%     \end{1=} \\

%     \begin{0-1}
%       \key{NOT}
%     \end{0-1}
%     \metaelement{control-style-name} \\

%     \begin{1=}
%       \key{PROPERTY} \integer \\
%       \metaelement{control-property-name}
%     \end{1=}
%     \begin{0-1}
%       IS \\
%       ARE \\
%       =
%     \end{0-1}
%     \begin{1=}
%       \identifier \\
%       \literal
%     \end{1=}\ldots
%   \end{1=} \ldots \\
% \end{syntax}
%
%where \defnmetaelement{control-style-name} is
% \begin{syntax}[\miscextcolour]
%   \begin{1=}
%     \key{PERMANENT} \\
%     \key{TEMPORARY} \\
%     \key{NOTAB} \\
%     \key{HEIGHT-IN-CELLS} \\
%     \key{WIDTH-IN-CELLS} \\
%     \key{THREEDIMENSIONAL} \\
%     \key{OVERLAP-LEFT} \\
%     \key{OVERLAP-TOP} \\
%     \key{SELF-ACT} \\
%     \key{NOTIFY} \\
%     \key{LEFT} \\
%     \key{RIGHT} \\
%     \key{CENTER} \\
%     \key{NO-KEY-LETTER} \\
%     \key{TRANSPARENT} \\
%     \key{NO-BOX} \\
%     \key{MULTILINE} \\
%     \key{VSCROLL} \\
%     \key{VSCROLL-BAR} \\
%     \key{USE-RETURN} \\
%     \key{USE-TAB} \\
%     \key{UPPER} \\
%     \key{LOWER} \\
%     \key{NO-AUTOSEL} \\
%     \key{READ-ONLY} \\
%     \key{NOTIFY-CHANGE} \\
%     \key{NUMERIC} \\
%     \key{SPINNER} \\
%     \key{AUTO-SPIN} \\
%     \key{DEFAULT-BUTTON} \\
%     \key{ESCAPE-BUTTON} \\
%     \key{OK-BUTTON} \\
%     \key{CANCEL-BUTTON} \\
%     \key{NO-AUTO-DEFAULT} \\
%     \key{BITMAP} \\
%     \key{SQUARE} \\
%     \key{FRAMED} \\
%     \key{UNFRAMED} \\
%     \key{FLAT} \\
%     \key{VTOP} \\
%     \key{LEFT-TEXT} \\
%     \key{UNSORTED} \\
%     \key{NOTIFY-DBLCLICK} \\
%     \key{NOTIFY-SELCHANGE} \\
%     \key{PAGED} \\
%     \key{NO-SEARCH} \\
%     \key{DROP-DOWN} \\
%     \key{STATIC-LIST} \\
%     \key{DROP-LIST} \\
%     \key{RAISED} \\
%     \key{LOWERED} \\
%     \key{ENGRAVED} \\
%     \key{RIMMED} \\
%     \key{HEAVY} \\
%     \key{VERY-HEAVY} \\
%     \key{ALTERNATE} \\
%     \key{FULL-HEIGHT} \\
%     \key{BUTTONS} \\
%     \key{FIXED-WIDTH} \\
%     \key{BOTTOM} \\
%     \key{VERTICAL} \\
%     \key{FLAT-BUTTONS} \\
%     \key{HOT-TRACK} \\
%     \key{NO-DIVIDERS} \\
%     \key{NO-FOCUS} \\
%     \key{DOTTED} \\
%     \key{DASH} \\
%     \key{DOTDASH} \\
%     \key{BOXED} \\
%     \key{HSCROLL} \\
%     \key{COLUMN-HEADINGS} \\
%     \key{ROW-HEADINGS} \\
%     \key{TITLED-HEADINGS} \\
%     \key{CENTERED-HEADINGS} \\
%     \key{ADJUSTABLE-COLUMNS} \\
%     \key{SHOW-LINES} \\
%     \key{LINES-AT-ROOT} \\
%     \key{SHOW-SEL-ALWAYS} \\
%     \key{USE-ALT} \\
%     \key{SHORT-DATE} \\
%     \key{CENTURY-DATE} \\
%     \key{LONG-DATE} \\
%     \key{TIME} \\
%     \key{NO-F4} \\
%     \key{NO-UPDOWN} \\
%     \key{RIGHT-ALIGN} \\
%     \key{SHOW-NONE}
%   \end{1=}
% \end{syntax}

where \defnmetaelement{appearance-attribute-clauses} is

\begin{syntax}
  \begin{0-1}
    \begin{1=}
      \key{BACKGROUND-COLOR} \\
      \miscext{\key{BACKGROUND-COLOUR}}
    \end{1=}
    IS
    \begin{1=}
      \identifier \\
      \integer
    \end{1=}
  \end{0-1}

  \begin{0-1}
    \key{BELL} \\
    \miscext{\key{BEEP}}
  \end{0-1}

  \key{BLANK}
  \begin{1=}
    \key{LINE} \\
    \key{SCREEN}
  \end{1=}

  \begin{1=}
    \key{BLINK} \\
    \miscext{\key{BLINKING}}
  \end{1=}

  \pending{
    WITH
    \begin{1=}
      \key{COLOR} \\
      \key{COLOUR}
    \end{1=}
    IS
    \begin{1=}
      \identifier \\
      \integer
    \end{1=}
  }

  \begin{0-1}
    \begin{1=}
      \key{FOREGROUND-COLOR} \\
      \miscext{\key{FOREGROUND-COLOUR}}
    \end{1=}
    IS
    \begin{1=}
      \identifier \\
      \integer
    \end{1=}
  \end{0-1}

  \begin{0-1}
    \key{HIGHLIGHT} \\
    \miscext{\key{HIGH}} \\
    \miscext{\key{BOLD}} \\
    \key{LOWLIGHT} \\
    \miscext{\key{LOW}}
  \end{0-1}

  \pending{
    \miscext{
      \begin{0-1}
        WITH \key{STANDARD}
      \end{0-1}
    }
  }

  \pending{
    \miscext{
      \begin{0-1}
        WITH \key{BACKGROUND-HIGH}
      \end{0-1}
    }
  }

  \pending{
    \miscext{
      \begin{0-1}
        WITH \key{BACKGROUND-STANDARD}
      \end{0-1}
    }
  }

  \pending{
    \miscext{
      \begin{0-1}
        WITH \key{BACKGROUND-LOW}
      \end{0-1}
    }
  }

  \pending{
    \miscext{
      \begin{0-1}
        \key{OVERLINE}
      \end{0-1}
    }
  }

  \miscext{
    \begin{0-1}
      \key{PROMPT}
      \begin{0-1}
        \key{CHARACTER} IS
        \begin{1=}
          \identifier \\
          \literal
        \end{1=}
      \end{0-1}
    \end{0-1}
  }

  \begin{0-1}
    \key{REVERSE-VIDEO} \\
    \miscext{\key{REVERSED}} \\
    \miscext{\key{REVERSE}}
  \end{0-1}

  \begin{0-1}
    \key{UNDERLINE} \\
    \miscext{\key{UNDERLINED}}
  \end{0-1}
\end{syntax}

where \defnmetaelement{source-destination-clauses} is

\begin{syntax}
  \begin{0-1}
    \key{FROM}
    \begin{1=}
      \identifier \\
      \literal
    \end{1=} \\
  \end{0-1}

  \begin{0-1}
    \key{TO} \identifier
  \end{0-1}

  \begin{0-1}
    \key{USING} \identifier
  \end{0-1}
\end{syntax}

\subsubsection{Syntax rules}

\subsubsection{General rules}

\section{Record description}
\format{data-description}
\begin{syntax}
  \metaelement{level-number} \metaelement{entry-name}
  \begin{0-1}
    \key{ANY}
    \begin{1=}
      \key{LENGTH} \\
      \gnucobol{\key{NUMERIC}}
    \end{1=} \\

    \metaelement{blank-when-zero-clause} \\

    IS \key{EXTERNAL}
    \begin{0-1}
      \key{AS} \literal
    \end{0-1} \\

    \miscext{IS \key{EXTERNAL-FORM}} \\

    \miscext{
      IS \key{IDENTIFIED} BY
      \begin{1=}
        \identifier \\
        \literal
      \end{1=}
    } \\

    IS \key{GLOBAL} \\

    \metaelement{justified-clause} \\

    \metaelement{occurs-clause} \\

    \begin{1=}
      \key{PICTURE} \\
      \key{PIC}
    \end{1=}
    IS \metaelement{picture-string-1} \\

    \key{REDEFINES} \identifier \\

    \metaelement{sign-clause} \\

    \begin{1=}
      \key{SYNCHRONIZED} \\
      \miscext{\key{SYNCHRONISED}} \\
      \key{SYNC}
    \end{1=}
    \begin{0-1}
      \key{LEFT} \\
      \pending{\key{RIGHT}}
    \end{0-1} \\

    \metaelement{usage-clause} \\

    \metaelement{value-clause}
  \end{0-1}\ldots\ {}.
\end{syntax}

\format{renames}
\begin{syntax}
  66 \identifier \key{RENAMES} \identifier
  \begin{0-1}
    \begin{1=}
      \key{THROUGH} \\
      \key{THRU}
    \end{1=}
    \identifier
  \end{0-1}.
\end{syntax}

\format{condition-name}
\begin{syntax}
  88 \identifier
  \begin{1=}
    \key{VALUE} \\
    \key{VALUES}
  \end{1=}
  \begin{0-1}
    IS \\
    ARE
  \end{0-1}
  \begin{1=}
    \literal
    \begin{0-1}
      \begin{1=}
        \key{THROUGH} \\
        \key{THRU}
      \end{1=}
      \literal
    \end{0-1}
  \end{1=} \ldots

  \begin{0-1}
    WHEN SET TO \key{FALSE} IS \literal
  \end{0-1}.
\end{syntax}

\subsubsection{Syntax rules}

\subsubsection{General rules}

\section{Constant definition}

\format{standard}
\begin{syntax}
  \begin{1=}
    1 \\
    01
  \end{1=}
  \identifier \key{CONSTANT}
  \begin{0-1}
    IS \key{GLOBAL}
  \end{0-1}
  \begin{1=}
    AS
    \begin{1=}
      \literal \\
      \begin{1=}
        \key{BYTE-LENGTH} \\
        \key{LENGTH}
      \end{1=}
      OF \identifier
    \end{1=} \\

    \key{FROM} \identifier
  \end{1=}.
\end{syntax}

\format{micro-focus}
\begin{syntax}[\miscextcolour]
  78 \identifier
  \gnucobol{
    \begin{0-1}
      IS \key{GLOBAL}
    \end{0-1}
  }
  \begin{1=}
    \key{VALUE} \\
    \key{VALUES}
  \end{1=}
  \begin{0-1}
    IS \\
    ARE
  \end{0-1}
  \begin{1=}
    \literal \\
    \key{START} OF \identifier \\
    \key{NEXT}
  \end{1=}
  .
\end{syntax}

\subsubsection{Syntax rules}

\subsubsection{General rules}

\section{Data division clauses}

\subsection{ANY LENGTH clause}

The ANY LENGTH clause specifies that the length of the data item will be determined at runtime.

\begin{syntax}
  \key{ANY}
  \begin{1=}
    \key{LENGTH} \\
    \gnucobol{\key{NUMERIC}} \\
  \end{1=}
\end{syntax}

\subsubsection{Syntax rules}

\subsubsection{General rules}

\subsection{AUTO clause}

The AUTO clause specifies that the screen cursor will immediate move to the next screen item when the current screen item is full.

\begin{syntax}
  \begin{1=}
    \key{AUTO} \\
    \miscext{\key{AUTO-SKIP}} \\
    \miscext{\key{AUTOTERMINATE}} \\
  \end{1=}
\end{syntax}

\subsubsection{Syntax rules}

\subsubsection{General rules}

\subsection{BACKGROUND-COLOR clause}

The BACKGROUND-COLOR clause specifies the background-color of the screen item.

\begin{syntax}
  \begin{1=}
    \key{BACKGROUND-COLOR} \\
    \miscext{\key{BACKGROUND-COLOUR}}
  \end{1=}
  IS
  \begin{1=}
    \identifier \\
    \literal
  \end{1=}
\end{syntax}

\subsubsection{Syntax rules}

\subsubsection{General rules}

\subsection{BACKGROUND-HIGH clause}

\begin{syntax}[\miscextcolour]
  \pending{\key{BACKGROUND-HIGH}}
\end{syntax}

\subsubsection{Syntax rules}

\subsubsection{General rules}

\subsection{BACKGROUND-LOW clause}

\begin{syntax}[\miscextcolour]
  \pending{\key{BACKGROUND-LOW}}
\end{syntax}

\subsubsection{Syntax rules}

\subsubsection{General rules}

\subsection{BACKGROUND-STANDARD clause}

\begin{syntax}[\miscextcolour]
  \pending{\key{BACKGROUND-STANDARD}}
\end{syntax}

\subsubsection{Syntax rules}

\subsubsection{General rules}

\subsection{BELL clause}

\begin{syntax}
  \begin{1=}
    \key{BELL} \\
    \miscext{\key{BEEP}}
  \end{1=}
\end{syntax}

\subsubsection{Syntax rules}

\subsubsection{General rules}

\subsection{BLANK clause}

\begin{syntax}
  \key{BLANK}
  \begin{1=}
    \key{LINE} \\
    \key{SCREEN}
  \end{1=}
\end{syntax}

\subsubsection{Syntax rules}

\subsubsection{General rules}

\subsection{BLANK WHEN ZERO clause}

The BLANK WHEN ZERO clause causes an item to be blanked when a value of zero is stored in it.

\begin{syntax}
  \key{BLANK} WHEN
  \begin{1=}
    \key{ZERO} \\
    \miscext{\key{ZEROES}} \\
    \miscext{\key{ZEROS}}
  \end{1=}
\end{syntax}

\subsubsection{Syntax rules}

\subsubsection{General rules}

\subsection{BLINK clause}

\begin{syntax}
  \begin{1=}
    \key{BLINK} \\
    \miscext{\key{BLINKING}}
  \end{1=}
\end{syntax}

\subsubsection{Syntax rules}

\subsubsection{General rules}

\subsection{BLOCK clause}

% TO-DO: CHECK!
The BLOCK clause specifies the size of a physical record, that is, how many logical records should be read in one physical I/O operation.

\begin{syntax}
  \key{BLOCK} CONTAINS \integer
  \begin{0-1}
    \key{TO} \integer
  \end{0-1}
  \begin{0-1}
    \key{CHARACTERS} \\
    \key{RECORDS}
  \end{0-1}\\
\end{syntax}

\subsubsection{Syntax rules}

\subsubsection{General rules}

\subsection{COLOR clause}

\begin{syntax}[\miscextcolour]
  \pending{
    WITH
    \begin{1=}
      \key{COLOR} \\
      \key{COLOUR}
    \end{1=}
    IS
    \begin{1=}
      \identifier \\
      \literal
    \end{1=}
  }
\end{syntax}

\subsubsection{Syntax rules}

\subsubsection{General rules}

\subsection{COLUMN clause}

The COLUMN clause specifies what column an item should be printed or displayed at.

\format{report-section}
\begin{syntax}
  \begin{1=}
    \key{COLUMN} \\
    \key{COLUMNS} \\
    \key{COL} \\
    \key{COLS}
  \end{1=}
  \begin{0-1}
    NUMBER \\
    NUMBERS
  \end{0-1}
  \begin{0-1}
    \key{LEFT} \\
    \key{RIGHT} \\
    \key{CENTER}
  \end{0-1}
  \begin{0-1}
    IS \\
    ARE
  \end{0-1}
  \begin{1=}
    \begin{0-1}
      \key{PLUS}
    \end{0-1}
    \integer \\

    \begin{1=}
      \integer
    \end{1=} \ldots
  \end{1=}
\end{syntax}

\format{screen-section}
\begin{syntax}
  \begin{1=}
    \key{COLUMN} \\
    \key{COL} \\
    \miscext{\key{POSITION}} \\
    \miscext{\key{POS}} \\
  \end{1=}
  NUMBER IS
  \begin{0-1}
    + \\
    - \\
    \key{PLUS} \\
    \key{MINUS}
  \end{0-1}
  \begin{1=}
    \identifier \\
    \integer
  \end{1=}
\end{syntax}

\subsubsection{Syntax rules}

\subsubsection{General rules}

\subsection{CCOL clause}

\begin{syntax}[\miscextcolour]
  \pending{
    \key{CCOL} NUMBER IS
    \begin{0-1}
      + \\
      - \\
      \key{PLUS} \\
      \key{MINUS}
    \end{0-1}
    \begin{1=}
      \identifier \\
      \integer
    \end{1=}
  }
\end{syntax}

\subsubsection{Syntax rules}

\subsubsection{General rules}

\subsection{CLINE clause}

\begin{syntax}[\miscextcolour]
  \pending{
    \key{CLINE} NUMBER IS
    \begin{0-1}
      + \\
      - \\
      \key{PLUS} \\
      \key{MINUS}
    \end{0-1}
    \begin{1=}
      \identifier \\
      \integer
    \end{1=}
  }
\end{syntax}

\subsubsection{Syntax rules}

\subsubsection{General rules}

\subsection{CSIZE clause}

\begin{syntax}[\miscextcolour]
  \pending{
    \key{CSIZE}
    \begin{0-1}
      IS \\
      =
    \end{0-1}
    \begin{1=}
      \identifier \\
      \integer
    \end{1=}
  }
\end{syntax}

\subsubsection{Syntax rules}

\subsubsection{General rules}

\subsection{DATA RECORDS clause}

\begin{syntax}[\deletedcolour]
  \key{DATA}
  \begin{1=}
    \key{RECORD} IS \\
    \key{RECORDS} ARE
  \end{1=}
  \begin{1=}
    \identifier
  \end{1=}\ldots
\end{syntax}

\subsubsection{Syntax rules}

\subsubsection{General rules}

\subsection{DESTINATION clause}

\begin{syntax}[\deletedcolour]
  \pending{\key{DESTINATION} IS \identifier}
\end{syntax}

\subsubsection{Syntax rules}

\subsubsection{General rules}

\subsection{DESTINATION COUNT clause}

\begin{syntax}[\deletedcolour]
  \pending{\key{DESTINATION} \key{COUNT} IS \identifier}
\end{syntax}

\subsubsection{Syntax rules}

\subsubsection{General rules}

\subsection{DESTINATION TABLE OCCURS clause}

\begin{syntax}[\deletedcolour]
  \pending{
    \key{DESTINATION} \key{TABLE} \key{OCCURS} \integer TIMES
    \begin{0-1}
      \key{INDEXED} BY
      \begin{1=}
        \cobolindexname
      \end{1=}\ldots
    \end{0-1}
  }
\end{syntax}

\subsubsection{Syntax rules}

\subsubsection{General rules}

\subsection{END KEY clause}

\begin{syntax}[\deletedcolour]
  \pending{\key{END} \key{KEY} IS \identifier}
\end{syntax}

\subsubsection{Syntax rules}

\subsubsection{General rules}

\subsection{Entry name}

The entry name specifies the name of the item being declared.

\begin{syntax}
  \begin{0-1}
    \key{FILLER} \\
    \identifier
  \end{0-1}
\end{syntax}

\subsubsection{Syntax rules}

\subsubsection{General rules}


\subsection{ERASE clause}

The ERASE clause indicates part of the screen to be blanked before displaying the item.

\begin{syntax}
  \key{ERASE}
  \begin{1=}
    \key{EOL} \\
    \key{EOS} \\

    \begin{0-1}
      \miscext{TO} \key{END} OF
    \end{0-1}
    \begin{1=}
      \key{LINE} \\
      \key{SCREEN}
    \end{1=}
  \end{1=}
\end{syntax}

\subsubsection{Syntax rules}

\subsubsection{General rules}

\subsection{ERROR KEY clause}

\begin{syntax}[\deletedcolour]
  \pending{\key{ERROR} \key{KEY} IS \identifier}
\end{syntax}

\subsubsection{Syntax rules}

\subsubsection{General rules}

\subsection{EXTERNAL clause}

\begin{syntax}
  IS \key{EXTERNAL}
  \begin{0-1}
    \key{AS} \literal
  \end{0-1}
\end{syntax}

\subsubsection{Syntax rules}

\subsubsection{General rules}

\subsection{EXTERNAL-FORM clause}

\begin{syntax}[\miscextcolour]
  \begin{1+}
    IS \key{EXTERNAL-FORM} \\

    IS \key{IDENTIFIED} BY
    \begin{1=}
      \identifier \\
      \literal
    \end{1=}
  \end{1+}
\end{syntax}

\subsubsection{Syntax rules}

\subsubsection{General rules}

\subsection{FOREGROUND-COLOR clause}

\begin{syntax}
  \begin{1=}
    \key{FOREGROUND-COLOR} \\
    \miscext{\key{FOREGROUND-COLOUR}}
  \end{1=}
  IS
  \begin{1=}
    \identifier \\
    \literal
  \end{1=}
\end{syntax}

\subsubsection{Syntax rules}

\subsubsection{General rules}

\subsection{FROM clause}

\begin{syntax}
  \key{FROM}
  \begin{1=}
    \identifier \\
    \literal
  \end{1=}
\end{syntax}

\subsubsection{Syntax rules}

\subsubsection{General rules}

\subsection{FULL clause}

% TO-DO: Check! Is it until the screen can be exited successfully?
The FULL clause specifies that the item must be filled entirely before the cursor can move to another item.

\begin{syntax}
  \begin{1=}
    \key{FULL} \\
    \miscext{\key{LENGTH-CHECK}} \\
  \end{1=}
\end{syntax}

\subsubsection{Syntax rules}

\subsubsection{General rules}

\subsection{GLOBAL clause}

The GLOBAL clause specifies that an item may be accessed from within nested programs.

\begin{syntax}
  IS \key{GLOBAL}
\end{syntax}

\subsubsection{Syntax rules}

\subsubsection{General rules}

\subsection{GRID clause}

\begin{syntax}[\miscextcolour]
  \pending{\key{GRID}}
\end{syntax}

\subsubsection{Syntax rules}

\subsubsection{General rules}

\subsection{HIGHLIGHT clause}

\begin{syntax}
  \begin{1=}
    \key{HIGHLIGHT} \\
    \miscext{\key{HIGH}} \\
    \miscext{\key{BOLD}}
  \end{1=}
\end{syntax}

\subsubsection{Syntax rules}

\subsubsection{General rules}

\subsection{INITIAL clause}

\begin{syntax}
  \key{INIITAL}
\end{syntax}

\subsubsection{Syntax rules}

\subsubsection{General rules}

\subsection{JUSTIFIED clause}

The JUSTIFIED clause causes data smaller than the data item to be padded by spaces on the left to fill the item.

\begin{syntax}
  \begin{1=}
    \key{JUSTIFIED} \\
    \key{JUST}
  \end{1=}
  RIGHT
\end{syntax}

\subsubsection{Syntax rules}

\subsubsection{General rules}

\subsection{LABEL RECORDS clause}

\begin{syntax}[\deletedcolour]
  \key{LABEL}
  \begin{1=}
    \key{RECORD} IS \\
    \key{RECORDS} ARE
  \end{1=}
  \begin{1=}
    \key{STANDARD} \\
    \key{OMITTED}
  \end{1=}
\end{syntax}

\subsubsection{Syntax rules}

\subsubsection{General rules}

\subsection{LEFTLINE clause}

\begin{syntax}[\miscextcolour]
  \pending{\key{LEFTLINE}}
\end{syntax}

\subsubsection{Syntax rules}

\subsubsection{General rules}

\subsection{Level-number}

A 1- or 2-digit integer having a value that is either between 1 and 49 or is 66, 77, \miscext{78} or 88.

\subsubsection{Syntax rules}

\subsubsection{General rules}

\subsection{LINAGE clause}

% TO-DO: Define logical page.
The LINAGE clause specifies the page limits of a logical page.

\begin{syntax}
  \key{LINAGE} IS
  \begin{1=}
    \identifier \\
    \literal
  \end{1=}
  LINES
  \begin{0-1}
    \begin{1=}
      \key{BOTTOM} \\
      \key{TOP} \\
      WITH \key{FOOTING} AT
    \end{1=}
    \begin{1=}
      \identifier \\
      \literal
    \end{1=} \\
  \end{0-1}
  \ldots \\
\end{syntax}

\subsubsection{Syntax rules}

\subsubsection{General rules}

\subsection{LINE clause}

The LINE clause specifies the line an item should be printer or displayed on.

\format{report section}
\begin{syntax}
  \key{LINE} NUMBER
  \begin{0-1}
    \key{IS} \\
    \gnucobol{\key{ARE}}
  \end{0-1} \\
  \begin{1=}
    \begin{0-1}
      + \\
      \key{PLUS}
    \end{0-1}
    \integer \\
    ON \key{NEXT} \key{PAGE}
  \end{1=}\ldots
\end{syntax}

\format{screen item}
\begin{syntax}
  \key{LINE} NUMBER IS
  \begin{0-1}
    + \\
    - \\
    \key{MINUS} \\
    \key{PLUS}
  \end{0-1}
  \begin{1=}
    \identifier \\
    \integer
  \end{1=}
\end{syntax}

\format{screen control}
\begin{syntax}[\miscextcolour]
  \pending{
    \key{LINES}
    \begin{0-1}
      IS \\
      =
    \end{0-1}
    \begin{1=}
      \identifier \\
      \integer
    \end{1=}
    \begin{0-1}
      \key{CELL} \\
      \key{PIXEL}
    \end{0-1}
  }
\end{syntax}

\subsubsection{Syntax rules}

\subsubsection{General rules}

\subsection{LOWLIGHT clause}

\begin{syntax}
  \begin{1=}
    \key{LOWLIGHT} \\
    \miscext{\key{LOW}}
  \end{1=}
\end{syntax}

\subsubsection{Syntax rules}

\subsubsection{General rules}

\subsection{MESSAGE COUNT clause}

\begin{syntax}[\deletedcolour]
  \pending{\key{MESSAGE} \key{COUNT} IS \identifier}
\end{syntax}

\subsubsection{Syntax rules}

\subsubsection{General rules}

\subsection{MESSAGE DATE clause}

\begin{syntax}[\deletedcolour]
  \pending{\key{MESSAGE} \key{DATE} IS \identifier}
\end{syntax}

\subsubsection{Syntax rules}

\subsubsection{General rules}

\subsection{MESSAGE TIME clause}

\begin{syntax}[\deletedcolour]
  \pending{\key{MESSAGE} \key{TIME} IS \identifier}
\end{syntax}

\subsubsection{Syntax rules}

\subsubsection{General rules}

\subsection{NEXT GROUP clause}

The NEXT GROUP clause specifies the number of blank lines that should follow the end of a report group.

\begin{syntax}
  \key{NEXT} \key{GROUP} IS
  \begin{1=}
    \begin{0-1}
      + \\
      \key{PLUS}
    \end{0-1}
    \integer \\
    ON \key{NEXT} \key{PAGE}
  \end{1=}
\end{syntax}

\subsubsection{Syntax rules}

\subsubsection{General rules}

\subsection{NO ECHO clause}

\begin{syntax}[\miscextcolour]
  \begin{1=}
    \key{NO} \key{ECHO} \\
    \key{NO-ECHO} \\
    \key{OFF}
  \end{1=}
\end{syntax}

\subsubsection{Syntax rules}

\subsubsection{General rules}

\subsection{OCCURS clause}

% TO-DO: Define tables and subscripts.
The OCCURS clause describes tables, repeated data items accessible by subscripts.

\format{usual}
\begin{syntax}
  \key{OCCURS}

  \begin{1=}
    \integer
    \begin{0-1}
      \key{TO} \integer
    \end{0-1}
    TIMES
    \begin{0-1}
      \key{DEPENDING} ON \identifier
    \end{0-1} \\

    \pending{
      \key{DYNAMIC}
      \begin{0-1}
        \key{CAPACITY} IN \identifier
      \end{0-1}
      \begin{0-1}
        \key{FROM} \integer
      \end{0-1}
      \begin{0-1}
        \key{TO} \integer
      \end{0-1}
      \begin{0-1}
        \key{INITIALIZED}
      \end{0-1}
    }
  \end{1=}

  \begin{0+} % TO-DO: This 0+ is a GnuCOBOL extension
    \begin{1=}
      \begin{1=}
        \key{ASCENDING} \\
        \key{DESCENDING}
      \end{1=}
      KEY IS
      \begin{1=}
        \identifier
      \end{1=}\ldots
    \end{1=}\ldots \\

    \key{INDEXED} BY
    \begin{1=}
      \cobolindexname
    \end{1=}\ldots
  \end{0+}
\end{syntax}

\format{report section}
\begin{syntax}
  \key{OCCURS} \integer
  \begin{0-1}
    \key{TO} \integer
  \end{0-1}
  TIMES
  \begin{0-1}
    \key{DEPENDING} ON \identifier
  \end{0-1}
  \begin{0-1}
    \key{STEP} \integer
  \end{0-1}
\end{syntax}

\format{screen section}
\begin{syntax}
  \key{OCCURS} \integer TIMES
\end{syntax}

\format{unbounded}
\begin{syntax}[\miscextcolour]
  \key{OCCURS}
  \begin{0-1}
    \integer \key{TO}
  \end{0-1}
  \key{UNBOUNDED} TIMES \key{DEPENDING} ON \identifier

  \begin{0+}
    \begin{1=}
      \begin{1=}
        \key{ASCENDING} \\
        \key{DESCENDING}
      \end{1=}
      KEY IS
      \begin{1=}
        \identifier
      \end{1=}\ldots
    \end{1=}\ldots \\

    \key{INDEXED} BY
    \begin{1=}
      \cobolindexname
    \end{1=}\ldots
  \end{0+}
\end{syntax}

\subsubsection{Syntax rules}

\subsubsection{General rules}

\subsection{OVERLINE clause}

\begin{syntax}[\miscextcolour]
  \pending{\key{OVERLINE}}
\end{syntax}

\subsubsection{Syntax rules}

\subsubsection{General rules}

\subsection{PICTURE clause}

The PICTURE clause describes the general characteristics and editing requirements of an elementary data item.

\begin{syntax}
  \begin{1=}
    \key{PICTURE} \\
    \key{PIC}
  \end{1=}
  IS \metaelement{picture-string-1}
\end{syntax}

\subsubsection{Syntax rules}

\subsubsection{General rules}

\subsection{PRESENT WHEN clause}

The PRESENT WHEN clause specifies a condition under which a report section entry will be processed.

\begin{syntax}
  \begin{1=}
    \key{PRESENT} \\
    \miscext{\key{ABSENT}}
  \end{1=}
  \begin{1=}
    \key{WHEN} \condition \\

    \miscext{
      \begin{1=}
        \key{BEFORE} \\
        \key{AFTER}
      \end{1=}
      NEW
      \begin{0-1}
        \key{PAGE} \\
        \identifier \\
        OR
      \end{0-1} \ldots
    } \\

    \miscext{
      \begin{1=}
        \key{JUSTIFIED} \\
        \key{JUST}
      \end{1=}
      \begin{1=}
        \key{BEFORE} \\
        \key{AFTER}
      \end{1=}
      NEW \key{PAGE}
    }
  \end{1=}
\end{syntax}

\subsubsection{Syntax rules}

\subsubsection{General rules}

\subsection{PROMPT clause}

\begin{syntax}[\miscextcolour]
  \key{PROMPT}
  \begin{0-1}
    \key{CHARACTER} IS
    \begin{1=}
      \identifier \\
      \literal
    \end{1=}
  \end{0-1}
\end{syntax}

\subsubsection{Syntax rules}

\subsubsection{General rules}

\subsection{RECORD clause}

The RECORD clause specifies the number of bytes of a logical record.

\begin{syntax}
  \key{RECORD}
  \begin{1=}
    CONTAINS \integer
    \begin{0-1}
      \key{TO} \integer
    \end{0-1}
    CHARACTERS \\
    IS \key{VARYING} in size
    \begin{0-1}
      FROM \integer
    \end{0-1}
    \begin{0-1}
      \key{TO} \integer
    \end{0-1}
    CHARACTERS \\\qquad
    \key{DEPENDING} ON \identifier
  \end{1=}
\end{syntax}

\subsubsection{Syntax rules}

\subsubsection{General rules}

\subsection{RECORDING MODE clause}

\begin{syntax}[\deletedcolour]
  \key{RECORDING} MODE IS
  \begin{1=}
    \begin{1=}
      \key{F} \\
      \key{FIXED} \\
    \end{1=} \\

    \begin{1=}
      \key{V} \\
      \key{VARIABLE} \\
    \end{1=} \\

    \key{U} \\
    \key{S}
  \end{1=}
\end{syntax}

\subsubsection{Syntax rules}

\subsubsection{General rules}

\subsection{REDEFINES clause}

The REDEFINES clause indicates the data shares the same memory as an item with a different description.

\begin{syntax}
  \key{REDEFINES} \identifier
\end{syntax}

\subsubsection{Syntax rules}

\subsubsection{General rules}

\subsection{REPORT clause}

\begin{syntax}
  \pending{
    \begin{1=}
      \key{REPORT} IS \\
      \key{REPORTS} ARE
    \end{1=}
    \begin{1=}
      \identifier
    \end{1=}\ldots
  }
\end{syntax}

\subsubsection{Syntax rules}

\subsubsection{General rules}

\subsection{REQUIRED clause}

\begin{syntax}
  \begin{1=}
    \key{REQUIRED} \\
    \miscext{\key{EMPTY-CHECK}}
  \end{1=}
\end{syntax}

\subsubsection{Syntax rules}

\subsubsection{General rules}

\subsection{REVERSE-VIDEO clause}

\begin{syntax}
  \begin{1=}
    \key{REVERSE-VIDEO} \\
    \miscext{\key{REVERSED}} \\
    \miscext{\key{REVERSE}}
  \end{1=}
\end{syntax}

\subsubsection{Syntax rules}

\subsubsection{General rules}

\subsection{SECURE clause}

\begin{syntax}
  \key{SECURE}
\end{syntax}

\subsubsection{Syntax rules}

\subsubsection{General rules}

\subsection{STATUS KEY clause}

\begin{syntax}[\deletedcolour]
  \pending{\key{STATUS} \key{KEY} IS \identifier}
\end{syntax}

\subsubsection{Syntax rules}

\subsubsection{General rules}

\subsection{SIGN clause}

The SIGN clause defines how to store item's sign.

\begin{syntax}
  SIGN IS
  \begin{1=}
    \key{LEADING} \\
    \key{TRAILING}
  \end{1=}
  \begin{0-1}
    \key{SEPARATE} CHARACTER
  \end{0-1}
\end{syntax}

\subsubsection{Syntax rules}

\subsubsection{General rules}

\subsection{SIZE clause}

\begin{syntax}[\miscextcolour]
  \pending{
    \key{SIZE}
    \begin{0-1}
      IS \\
      =
    \end{0-1}
    \begin{1=}
      \identifier \\
      \integer
    \end{1=}
  }
\end{syntax}

\subsubsection{Syntax rules}

\subsubsection{General rules}

\subsection{SOURCE clause}

The SOURCE clause identifies data to be used in processing a report section entry.

\begin{syntax}
  \key{SOURCE} IS \metaelement{number-1}
  \begin{0-1}
    \metaelement{rounded-phrase}
  \end{0-1}
\end{syntax}

\subsubsection{Syntax rules}

\subsubsection{General rules}

\subsection{STANDARD clause}

\begin{syntax}[\miscextcolour]
  WITH \key{STANDARD}
\end{syntax}

\subsubsection{Syntax rules}

\subsubsection{General rules}

\subsection{SUM clause}

The SUM clause provides a list of data items to be summed for use in an elementary report item.

\begin{syntax}
  \key{SUM} OF
  \begin{1=}
    \metaelement{number-1}
  \end{1=}\ldots
  \begin{0-1}
    \begin{1=}
      \key{RESET} ON
      \begin{1=}
        \identifier \\
        \key{FINAL}
      \end{1=} \\

      \key{UPON} \identifier
  \end{1=}
  \end{0-1}
\end{syntax}

\subsubsection{Syntax rules}

\subsubsection{General rules}

\subsection{SYMBOLIC DESTINATION clause}

\begin{syntax}[\deletedcolour]
  \pending{\key{SYMBOLIC} \key{DESTINATION} IS \identifier}
\end{syntax}

\subsubsection{Syntax rules}

\subsubsection{General rules}

\subsection{SYMBOLIC QUEUE clause}

\begin{syntax}[\deletedcolour]
  \pending{SYMBOLIC \key{QUEUE} IS \identifier}
\end{syntax}

\subsubsection{Syntax rules}

\subsubsection{General rules}

\subsection{SYMBOLIC SOURCE clause}

\begin{syntax}[\deletedcolour]
  \pending{SYMBOLIC \key{SOURCE} IS \identifier}
\end{syntax}

\subsubsection{Syntax rules}

\subsubsection{General rules}

\subsection{SYMBOLIC SUB-QUEUE-1 clause}

\begin{syntax}[\deletedcolour]
  \pending{SYMBOLIC \key{SUB-QUEUE-1} IS \identifier}
\end{syntax}

\subsubsection{Syntax rules}

\subsubsection{General rules}

\subsection{SYMBOLIC SUB-QUEUE-2 clause}

\begin{syntax}[\deletedcolour]
  \pending{SYMBOLIC \key{SUB-QUEUE-2} IS \identifier}
\end{syntax}

\subsubsection{Syntax rules}

\subsubsection{General rules}

\subsection{SYMBOLIC SUB-QUEUE-3 clause}

\begin{syntax}[\deletedcolour]
  \pending{SYMBOLIC \key{SUB-QUEUE-3} IS \identifier}
\end{syntax}

\subsubsection{Syntax rules}

\subsubsection{General rules}

\subsection{SYMBOLIC TERMINAL clause}

\begin{syntax}[\deletedcolour]
  \pending{\key{SYMBOLIC} \key{TERMINAL} IS \identifier}
\end{syntax}

\subsubsection{Syntax rules}

\subsubsection{General rules}

\subsection{SYNCHRONIZED clause}

% TO-DO: Check "should".
The SYNCHRONIZED clause specifies an item should be aligned in a byte boundary and in what way.

\begin{syntax}
  \begin{1=}
    \key{SYNCHRONIZED} \\
    \miscext{\key{SYNCHRONISED}} \\
    \key{SYNC}
  \end{1=}
  \begin{0-1}
    \key{LEFT} \\
    \pending{\key{RIGHT}}
  \end{0-1}
\end{syntax}

\subsubsection{Syntax rules}

\subsubsection{General rules}

\subsection{TEXT LENGTH clause}

\begin{syntax}[\deletedcolour]
  \pending{\key{TEXT} \key{LENGTH} IS \identifier}
\end{syntax}

\subsubsection{Syntax rules}

\subsubsection{General rules}

\subsection{TO clause}

\begin{syntax}
  \key{TO} \identifier
\end{syntax}

\subsubsection{Syntax rules}

\subsubsection{General rules}

\subsection{TYPE clause}

The TYPE clause specifies when to print a report group.

\begin{syntax}
  \key{TYPE} IS
  \begin{1=}
    \begin{1=}
      \key{CONTROL} \key{HEADING} \\
      \key{CH}
    \end{1=}
    \begin{0-1}
      ON \\
      FOR
    \end{0-1}
    \begin{1=}
      \identifier \\
      \key{FINAL}
    \end{1=}
    \begin{0-1}
      \key{OR} \key{PAGE}
    \end{0-1} \\

    \begin{1=}
      \key{CONTROL} \key{FOOTING} \\
      \key{CF}
    \end{1=}
    \begin{0-1}
      ON \\
      FOR
    \end{0-1}
    \begin{0-1}
      \begin{1=}
        \identifier \\
        \key{FINAL}
      \end{1=}
      \begin{0-1}
        \key{OR} \key{PAGE}
      \end{0-1} \\

      \key{ALL}
    \end{0-1} \\

    \begin{1=}
      \key{DETAIL} \\
      \key{DE}
    \end{1=} \\

    \begin{1=}
      \key{PAGE} \key{FOOTING} \\
      \key{PF}
    \end{1=} \\

    \begin{1=}
      \key{PAGE} \key{HEADING} \\
      \key{PH}
    \end{1=} \\

    \begin{1=}
      \key{REPORT} \key{FOOTING} \\
      \key{RF}
    \end{1=} \\

    \begin{1=}
      \key{REPORT} \key{HEADING} \\
      \key{RH}
    \end{1=}
  \end{1=}
\end{syntax}

\subsubsection{Syntax rules}

\subsubsection{General rules}

\subsection{UNDERLINE clause}

The UNDERLINE clause specifies that each character of a field is to be displayed with an underline.

\begin{syntax}
  \begin{1=}
    \key{UNDERLINE} \\
    \miscext{\key{UNDERLINED}}
  \end{1=}
\end{syntax}

\subsubsection{Syntax rules}

\subsubsection{General rules}

\subsection{USAGE clause}

The USAGE clause specifies the representation of a data item in memory.

\begin{syntax}
  \begin{0-1}
    \key{USAGE} IS
  \end{0-1}
  \begin{1=}
    \key{BINARY} \\
    \key{BIT} \\

    \metaelement{fixed-length-integers} \\
    \metaelement{computational-usages} \\

    \key{DISPLAY} \\

    \pending{\key{FLOAT-BINARY-32}} \\
    \pending{\key{FLOAT-BINARY-64}} \\
    \pending{\key{FLOAT-BINARY-128}} \\
    \key{FLOAT-DECIMAL-16} \\
    \key{FLOAT-DECIMAL-34} \\
    \key{FLOAT-LONG} \\
    \key{FLOAT-SHORT} \\

    \key{INDEX} \\

    \pending{\key{NATIONAL}} \\

    \key{PACKED} \key{DECIMAL} \\
    \miscext{\key{POINTER}} \\
    \key{PROGRAM-POINTER} \\

    \miscext{\metaelement{handle-usages}}
  \end{1=}
\end{syntax}

where \defnmetaelement{fixed-length-integers} is

\begin{syntax}
  \begin{1=}
    \begin{1=}
      \key{BINARY-CHAR} \\

      \begin{1=}
        \key{BINARY-LONG} \\
        \gnucobol{\key{BINARY-INT}}
      \end{1=} \\

      \gnucobol{\key{BINARY-C-LONG}} \\

      \begin{1=}
        \key{BINARY-DOUBLE} \\
        \gnucobol{\key{BINARY-LONG-LONG}} \\
      \end{1=} \\
    \end{1=}
    \begin{0-1}
      \key{SIGNED} \\
      \key{UNSIGNED}
    \end{0-1} \\

    \miscext{\key{SIGNED-SHORT}} \\
    \miscext{\key{SIGNED-INT}} \\
    \miscext{\key{SIGNED-LONG}} \\

    \miscext{\key{UNSIGNED-SHORT}} \\
    \miscext{\key{UNSIGNED-INT}} \\
    \miscext{\key{UNSIGNED-LONG}}
  \end{1=}
\end{syntax}

where \defnmetaelement{computation-usages} is

\begin{syntax}
  \begin{1=}
    \begin{1=}
      \key{COMP} \\
      \key{COMPUTATIONAL}
    \end{1=} \\

    \miscext{
      \begin{1=}
        \key{COMP-0} \\
        \key{COMPUTATIONAL-0}
      \end{1=}
    } \\
    
    \miscext{
      \begin{1=}
        \key{COMP-1} \\
        \key{COMPUTATIONAL-1}
      \end{1=}
    } \\

    \miscext{
      \begin{1=}
        \key{COMP-2} \\
        \key{COMPUTATIONAL-2}
      \end{1=}
    } \\

    \xopen{
      \begin{1=}
        \key{COMP-3} \\
        \key{COMPUTATIONAL-3}
      \end{1=}
    } \\

    \miscext{
      \begin{1=}
        \key{COMP-4} \\
        \key{COMPUTATIONAL-4}
      \end{1=}
    } \\

    \xopen{
      \begin{1=}
        \key{COMP-5} \\
        \key{COMPUTATIONAL-5}
      \end{1=}
    } \\

    \miscext{
      \begin{1=}
        \key{COMP-6} \\
        \key{COMPUTATIONAL-6}
      \end{1=}
    } \\

    \miscext{
      \begin{1=}
        \key{COMP-N} \\
        \key{COMPUTATIONAL-N}
      \end{1=}
    } \\

    \miscext{
      \begin{1=}
        \key{COMP-X} \\
        \key{COMPUTATIONAL-X}
      \end{1=}
    } \\
  \end{1=}
\end{syntax}

where \defnmetaelement{handle-usages} is

\begin{syntax}[\miscextcolour]
  \begin{1=}
    \key{HANDLE}
    \begin{0-1}
      OF
      \begin{1=}
        \pending{
          \key{FONT}
          \begin{0-1}
            \key{DEFAULT-FONT} \\
            \key{FIXED-FONT} \\
            \key{TRADITIONAL-FONT} \\
            \key{SMALL-FONT} \\
            \key{MEDIUM-FONT} \\
            \key{LARGE-FONT}
          \end{0-1}
        } \\

        \pending{
          \key{LAYOUT-MANAGER}
          \begin{0-1}
            \key{LM-RESIZE}
          \end{0-1}
        } \\

        \pending{\key{MENU}} \\
        \key{SUBWINDOW} \\
        \key{THREAD} \\
        \key{VARIANT} \\
        \key{WINDOW}
      \end{1=}
    \end{0-1}
  \end{1=}
\end{syntax}

\subsubsection{Syntax rules}

\subsubsection{General rules}

\subsection{USING clause}

\begin{syntax}
  \key{USING} \identifier
\end{syntax}

\subsubsection{Syntax rules}

\subsubsection{General rules}

\subsection{VALUE clause}

The VALUE clause specifies the initial value of the local-storage and working-storage section data items and the values to be used in INITALIZE statements.

The VALUE clause for condition-names specifies the values under which a condition-name is true (or false).

\format{initialization}
\begin{syntax}
  \begin{1=}
    \key{VALUE} \\
    \gnucobol{\key{VALUES}}
  \end{1=}
  \begin{0-1}
    IS \\
    \gnucobol{ARE}
  \end{0-1}
  \literal
\end{syntax}

\format{condition}
\begin{syntax}
  \begin{1=}
    \key{VALUE} \\
    \gnucobol{\key{VALUES}}
  \end{1=}
  \begin{0-1}
    IS \\
    \gnucobol{ARE}
  \end{0-1}
  \begin{1=}
    \literal
    \begin{0-1}
      \begin{1=}
        \key{THROUGH} \\
        \key{THRU}
      \end{1=}
      \literal
    \end{0-1}
  \end{1=} \ldots

  \begin{0-1}
    WHEN SET TO \key{FALSE} IS \literal
  \end{0-1}
\end{syntax}

\subsubsection{Syntax rules}

\subsubsection{General rules}

\subsection{VALUE OF clause}

\begin{syntax}[\deletedcolour]
  \begin{1=}
    \key{VALUE} \\
    \gnucobol{\key{VALUES}}
  \end{1=}
  \key{OF}
  \begin{1=}
    \key{FILE-ID} \\
    \key{ID} \\
    \identifier
  \end{1=}
  IS
  \begin{1=}
    \literal \\
    \identifier
  \end{1=}
\end{syntax}

\subsubsection{Syntax rules}

\subsubsection{General rules}

\subsection{VARYING clause}

The VARYING clause declares counters to be used in printing repeated items in the report writer.

\begin{syntax}
  \pending{
    \key{VARYING} \identifier \key{FROM} \metaelement{number-1} \key{BY} \metaelement{number-2}
  }
\end{syntax}

\subsubsection{Syntax rules}

\subsubsection{General rules}

\subsection{VOLATILE clause}

The VOLATILE clause indicates an item may change in ways outside the control of the program.

\begin{syntax}[\miscextcolour]
  \pending{\key{VOLATILE}}
\end{syntax}

\subsubsection{Syntax rules}

\subsubsection{General rules}

%%% Local Variables:
%%% mode: latex
%%% TeX-master: "grammar.tex"
%%% End:

\chapter{Procedure division}

\section{Concepts}

\subsection{Exceptions}

\begin{itemize}
\item EC-ALL

\item EC-ARGUMENT
  \begin{itemize}
  \item EC-ARGUMENT-FUNCTION
  \item EC-ARGUMENT-IMP
  \end{itemize}
\item EC-BOUND
  \begin{itemize}
  \item EC-BOUND-IMP
  \item EC-BOUND-ODO
  \item EC-BOUND-OVERFLOW
  \item EC-BOUND-PTR
  \item EC-BOUND-REF-MOD
  \item EC-BOUND-SET
  \item EC-BOUND-SUBSCRIPT
  \item EC-BOUND-TABLE-LIMIT
  \end{itemize}

\item EC-DATA
  \begin{itemize}
  \item EC-DATA-CONVERSION
  \item EC-DATA-IMP
  \item EC-DATA-INCOMPATIBLE
  \item EC-DATA-OVERFLOW
  \item EC-DATA-PTR-NULL
  \end{itemize}

\item EC-FLOW
  \begin{itemize}
  \item EC-FLOW-GLOBAL-EXIT
  \item EC-FLOW-GLOBAL-GOBACK
  \item EC-FLOW-IMP
  \item EC-FLOW-RELEASE
  \item EC-FLOW-REPORT
  \item EC-FLOW-RETURN
  \item EC-FLOW-SEARCH
  \item EC-FLOW-USE
  \end{itemize}

\item EC-FUNCTION
  \begin{itemize}
  \item EC-FUNCTION-PTR-INVALID
  \item EC-FUNCTION-PTR-NULL
  \end{itemize}

\item EC-I-O
  \begin{itemize}
  \item EC-I-O-AT-END
  \item EC-I-O-EOP
  \item EC-I-O-EOP-OVERFLOW
  \item EC-I-O-FILE-SHARING
  \item EC-I-O-IMP
  \item EC-I-O-INVALID-KEY
  \item EC-I-O-LINAGE
  \item EC-I-O-LOGIC-ERROR
  \item EC-I-O-PERMANENT-ERROR
  \item EC-I-O-RECORD-OPERATION
  \end{itemize}

\item EC-IMP
  \begin{itemize}
  \item \gnucobol{EC-IMP-ACCEPT}
  \item \gnucobol{EC-IMP-DISPLAY}
  \item \gnucobol{EC-IMP-UTC-UNKNOWN}
  \end{itemize}

\item EC-LOCALE
  \begin{itemize}
  \item EC-LOCALE-IMP
  \item EC-LOCALE-INCOMPATIBLE
  \item EC-LOCALE-INVALID
  \item EC-LOCALE-INVALID-PTR
  \item EC-LOCALE-MISSING
  \item EC-LOCALE-SIZE
  \end{itemize}

\item EC-OO
  \begin{itemize}
  \item EC-OO-CONFORMANCE
  \item EC-OO-EXCEPTION
  \item EC-OO-FINALIZABLE
  \item EC-OO-IMP
  \item EC-OO-METHOD
  \item EC-OO-NULL
  \item EC-OO-RESOURCE
  \item EC-OO-UNIVERSAL
  \end{itemize}

\item EC-ORDER
  \begin{itemize}
  \item EC-ORDER-IMP
  \item EC-ORDER-NOT-SUPPORTED
  \end{itemize}

\item EC-OVERFLOW
  \begin{itemize}
  \item EC-OVERFLOW-IMP
  \item EC-OVERFLOW-STRING
  \item EC-OVERFLOW-UNSTRING
  \end{itemize}

\item EC-PROGRAM
  \begin{itemize}
  \item EC-PROGRAM-ARG-MISMATCH
  \item EC-PROGRAM-ARG-OMITTED
  \item EC-PROGRAM-CANCEL-ACTIVE
  \item EC-PROGRAM-IMP
  \item EC-PROGRAM-NOT-FOUND
  \item EC-PROGRAM-PTR-NULL
  \item EC-PROGRAM-RECURSIVE-CALL
  \item EC-PROGRAM-RESOURCES
  \end{itemize}

\item EC-RAISING
  \begin{itemize}
  \item EC-RAISING-IMP
  \item EC-RAISING-NOT-SPECIFIED
  \end{itemize}

\item EC-RANGE
  \begin{itemize}
  \item EC-RANGE-IMP
  \item EC-RANGE-INDEX
  \item EC-RANGE-INSPECT-SIZE
  \item EC-RANGE-INVALID
  \item EC-RANGE-PERFORM-VARYING
  \item EC-RANGE-PTR
  \item EC-RANGE-SEARCH-INDEX
  \item EC-RANGE-SEARCH-NO-MATCH
  \end{itemize}

\item EC-REPORT
  \begin{itemize}
  \item EC-REPORT-ACTIVE
  \item EC-REPORT-COLUMN-OVERLAP
  \item EC-REPORT-FILE-MODE
  \item EC-REPORT-IMP
  \item EC-REPORT-INACTIVE
  \item EC-REPORT-LINE-OVERLAP
  \item EC-REPORT-NOT-TERMINATED
  \item EC-REPORT-PAGE-LIMIT
  \item EC-REPORT-PAGE-WIDTH
  \item EC-REPORT-SUM-SIZE
  \item EC-REPORT-VARYING
  \end{itemize}

\item EC-SCREEN
  \begin{itemize}
  \item EC-SCREEN-FIELD-OVERLAP
  \item EC-SCREEN-IMP
  \item EC-SCREEN-ITEM-TRUNCATED
  \item EC-SCREEN-LINE-NUMBER
  \item EC-SCREEN-STARTING-COLUMN
  \end{itemize}

\item EC-SIZE
  \begin{itemize}
  \item EC-SIZE-ADDRESS
  \item EC-SIZE-EXPONENTIATION
  \item EC-SIZE-IMP
  \item EC-SIZE-OVERFLOW
  \item EC-SIZE-TRUNCATION
  \item EC-SIZE-UNDERFLOW
  \item EC-SIZE-ZERO-DIVIDE
  \end{itemize}

\item EC-SORT-MERGE
  \begin{itemize}
  \item EC-SORT-MERGE-ACTIVE
  \item EC-SORT-MERGE-FILE-OPEN
  \item EC-SORT-MERGE-IMP
  \item EC-SORT-MERGE-RELEASE
  \item EC-SORT-MERGE-RETURN
  \item EC-SORT-MERGE-SEQUENCE
  \end{itemize}

\item EC-STORAGE
  \begin{itemize}
  \item EC-STORAGE-IMP
  \item EC-STORAGE-NOT-ALLOC
  \item EC-STORAGE-NOT-AVAIL
  \end{itemize}

\item EC-USER

\item EC-VALIDATE
  \begin{itemize}
  \item EC-VALIDATE-CONTENT
  \item EC-VALIDATE-FORMAT
  \item EC-VALIDATE-IMP
  \item EC-VALIDATE-RELATION
  \item EC-VALIDATE-VARYING
  \end{itemize}

\item EC-XML
  \begin{itemize}
  \item EC-XML-CODESET
  \item EC-XML-CODESET-CONVERSION
  \item EC-XML-COUNT
  \item EC-XML-DOCUMENT-TYPE
  \item EC-XML-IMPLICIT-CLOSE
  \item EC-XML-INVALID
  \item EC-XML-NAMESPACE
  \item EC-XML-STACKED-OPEN
  \item EC-XML-RANGE
  \end{itemize}
\end{itemize}

\section{Procedure division header}

\begin{syntax}
  \key{PROCEDURE} \key{DIVISION}
  \miscext{
    \begin{0-1}
      \mnemonicname
    \end{0-1}
  }
  \begin{0-1}
    \metaelement{using-chaining-clause}
  \end{0-1}
  \begin{0-1}
    \key{RETURNING}
    \begin{1=}
      \identifier \\
      \gnucobol{\key{OMITTED}}
    \end{1=}
  \end{0-1}.\newline
  \begin{0-1}
    \metaelement{declaratives}
  \end{0-1}\newline
  \begin{0-1}
    \metaelement{section-name-2} \key{SECTION}. \\
    \metaelement{paragraph-name-2}. \\
    \imperativestatement .
  \end{0-1} \ldots
\end{syntax}

where \metaelement{using-chaining-clause} is

\begin{syntax}
  \begin{1=}
    \key{USING} \\
    \miscext{\key{CHAINING}}
  \end{1=}

  \begin{1=}
    BY
    \begin{1=}
      \key{REFERENCE} \\
      \pending{\key{VALUE}}
    \end{1=}
    \miscext{
      \begin{0-1}
        \begin{0-1}
          \key{UNSIGNED}
        \end{0-1}
        \key{SIZE} IS
        \begin{1=}
          \key{AUTO} \\
          \integer
        \end{1=} \\

        \key{SIZE} IS \key{DEFAULT}
      \end{0-1}
    }

    \begin{0-1}
      \key{OPTIONAL}
    \end{0-1}
    \identifier
  \end{1=}\ldots
\end{syntax}

where \metaelement{declaratives} is

\begin{syntax}
  \key{DECLARATIVES}.\newline
  \begin{0-1}
    \metaelement{section-name-1} \key{SECTION}.
    \metaelement{use-statement}
    \begin{0-1}
      \metaelement{paragraph-name-2}. \\
      \imperativestatement .
    \end{0-1} \ldots
  \end{0-1}\ldots\newline
  \key{END} \key{DECLARATIVES}.
\end{syntax}

\subsubsection{Syntax rules}

\subsubsection{General rules}

\section{Common phrases}

\subsection{RETRY phrase}

\begin{syntax}
  \pending{
    \key{RETRY}
    \begin{1=}
      \gnucobol{FOR} \arithmeticexpression \key{TIMES} \\
      FOR \arithmeticexpression \key{SECONDS} \\
      \key{FOREVER}
    \end{1=}
  }
\end{syntax}

\subsubsection{Syntax rules}

\subsubsection{General rules}


\subsection{ROUNDED phrase}

\begin{syntax}
  \key{ROUNDED}
  \begin{0-1}
    \key{MODE} IS
    \begin{1=}
      \key{AWAY-FROM-ZERO} \\
      \key{NEAREST-AWAY-FROM-ZERO} \\
      \key{NEAREST-EVEN} \\
      \key{NEAREST-TOWARD-ZERO} \\
      \key{PROHIBITED} \\
      \key{TOWARD-GREATER} \\
      \key{TOWARD-LESSER} \\
      \key{TRUNCATION}
    \end{1=}
  \end{0-1}
\end{syntax}

\subsubsection{Syntax rules}

\subsubsection{General rules}


\subsection{SIZE phrase}

\begin{syntax}[\gnucobolcolour]
  \begin{1=}
    \key{SIZE} IS \key{AUTO} \\
    \key{SIZE} IS \key{DEFAULT} \\
    \key{SIZE} IS \integer \\
    \key{UNSIGNED} \key{SIZE} IS \key{AUTO} \\
    \key{UNSIGNED} \key{SIZE} IS \integer
  \end{1=}
\end{syntax}

\subsubsection{Syntax rules}

\subsubsection{General rules}

\section{ACCEPT statement}

The ACCEPT statement transfers data provided by the user or the operating system to the specified data item.

\format{device}
\begin{syntax}
  \key{ACCEPT}
  \begin{1=}
    \identifier \\
    \miscext{\key{OMITTED}}
  \end{1=}
  \begin{0-1}
    \key{FROM} \mnemonicname
  \end{0-1}
  \begin{0-1}
    \key{END-ACCEPT}
  \end{0-1}
\end{syntax}

\format{screen}
\begin{syntax}
  \key{ACCEPT}
  \begin{1=}
    \identifier \\
    \miscext{\key{OMITTED}}
  \end{1=}

  \begin{0+}
    \begin{1=}
      \begin{1+}
        AT \key{LINE} NUMBER
        \begin{1=}
          \identifier \\
          \integer
        \end{1=} \\

        AT
        \begin{1=}
          \key{COLUMN} \\
          \key{COL} \\
          \miscext{\key{POSITION}}
        \end{1=}
        NUMBER
        \begin{1=}
          \identifier \\
          \integer
        \end{1=}
      \end{1+} \\

      \miscext{
        \key{AT}
        \begin{1=}
          \identifier \\
          \integer
        \end{1=}
      }
    \end{1=} \\

    \miscext{\key{FROM} \key{CRT}} \\
    \miscext{\key{MODE} IS \key{BLOCK}} \\
    \miscext{\metaelement{appearance-attribute-clauses}} \\
    \miscext{\metaelement{accept-attribute-clauses}}
  \end{0+}
\end{syntax}

where \metaelement{appearance-attribute-clauses} is

\begin{syntax}[\miscextcolour]
  WITH
  \begin{1=}
    \key{BELL} \\
    \key{BEEP}
  \end{1=} \\
  
  WITH
  \begin{1=}
    \key{BLINK} \\
    \key{BLINKING}
  \end{1=} \\

  WITH
  \begin{1=}
    \key{HIGHLIGHT} \\
    \key{HIGH} \\
    \key{BOLD} \\
    \key{LOWLIGHT} \\
    \key{LOW}
  \end{1=} \\

  \pending{WITH \key{STANDARD}} \\
  
  \pending{WITH \key{BACKGROUND-HIGH}} \\
  
  \pending{WITH \key{BACKGROUND-STANDARD}} \\
  
  \pending{WITH \key{BACKGROUND-LOW}} \\
  
  WITH \key{LEFTLINE} \\

  WITH \key{OVERLINE} \\

  WITH \key{PROMPT}
  \begin{0-1}
    \key{CHARACTER} IS
    \begin{1=}
      \identifier \\
      \literal
    \end{1=}
  \end{0-1} \\

  WITH
  \begin{1=}
    \key{REVERSE-VIDEO} \\
    \key{REVERSED} \\
    \key{REVERSE}
  \end{1=} \\

  WITH PROTECTED \key{SIZE} IS
  \begin{1=}
    \identifier \\
    \integer
  \end{1=} \\

  WITH
  \begin{1=}
    \key{UNDERLINE} \\
    \key{UNDERLINED}
  \end{1=} \\

  WITH
  \begin{1=}
    \key{FOREGROUND-COLOR} \\
    \key{FOREGROUND-COLOUR}
  \end{1=}
  IS
  \begin{1=}
    \identifier \\
    \integer
  \end{1=} \\

  WITH
  \begin{1=}
    \key{BACKGROUND-COLOR} \\
    \key{BACKGROUND-COLOUR}
  \end{1=}
  IS
  \begin{1=}
    \identifier \\
    \integer
  \end{1=} \\

  \pending{
    WITH
    \begin{1=}
      \key{COLOR} \\
      \key{COLOUR}
    \end{1=}
    IS
    \begin{1=}
      \identifier \\
      \integer
    \end{1=}
  } \\

  WITH \key{SCROLL}
  \begin{1=}
    \key{UP} \\
    \key{DOWN}
  \end{1=}
  \begin{0-1}
    \begin{1=}
      \identifier \\
      \integer
    \end{1=}
    \begin{1=}
      \key{LINE} \\
      \key{LINES}
    \end{1=}
  \end{0-1}
\end{syntax}

where \metaelement{accept-attribute-clauses} is

\begin{syntax}[\miscextcolour]
  WITH
  \begin{1=}
    \key{AUTO} \\
    \key{TAB}
  \end{1=} \\

  WITH
  \begin{1=}
    \key{CONVERSION} \\
    \key{CONVERT}
  \end{1=} \\

  WITH
  \begin{1=}
    \key{FULL} \\
    \key{LENGTH-CHECK} \\
  \end{1=} \\

  WITH
  \begin{1=}
    \key{LOWER} \\
    \key{UPPER}
  \end{1=} \\

  WITH
  \begin{1=}
    \key{NO-ECHO} \\
    \key{NO} \key{ECHO} \\
    \key{OFF}
  \end{1=} \\

  WITH
  \begin{1=}
    \key{REQUIRED} \\
    \key{EMPTY-CHECK}
  \end{1=} \\

  WITH \key{SECURE} \\

  WITH
  \begin{0-1}
    \key{NO}
  \end{0-1}
  \begin{1=}
    \key{DEFAULT} \\
    \key{UPDATE}
  \end{1=} \\
    
  \begin{1=}
    \key{WITH}
    % TIME-OUT is context-sensitive, so WITH is required.
    \begin{1=}
      \key{TIMEOUT} \\
      \key{TIME-OUT} \\
    \end{1=}
    AFTER \\
    BEFORE \key{TIME}
  \end{1=}
  \begin{1=}
    \identifier \\
    \integer
  \end{1=}
\end{syntax}

% Shouldn't all the clauses except AT LINE/COLUMN be miscext?

\format{temporal}
\begin{syntax}
  \key{ACCEPT} \identifier \key{FROM}
  \begin{1=}
    \key{DATE}
    \begin{0-1}
      \key{YYYYMMDD}
    \end{0-1} \\

    \key{DAY}
    \begin{0-1}
      \key{YYYYDDD}
    \end{0-1} \\

    \key{DAY-OF-WEEK} \\
    \key{TIME} \\
  \end{1=}
\end{syntax}

\format{environment}
\begin{syntax}[\miscextcolour]
  \key{ACCEPT} \identifier \key{FROM}
  \begin{1=}
    \key{ARGUMENT-NUMBER} \\

    \begin{1=}
      \key{COLUMNS} \\
      \key{COLS}
    \end{1=} \\

    \key{COMMAND-LINE} \\
    \key{ESCAPE} \key{KEY} \\
    \key{EXCEPTION} \key{STATUS} \\

    \begin{1=}
      \key{LINES} \\
      \key{LINE} \key{NUMBER}
    \end{1=} \\

    \key{USER} \key{NAME} \\
    \key{WORD}
  \end{1=}
\end{syntax}

\format{environment-exception}
\begin{syntax}[\miscextcolour]
  \begin{minipage}[!h]{1.0\linewidth}
    \key{ACCEPT} \identifier \key{FROM}
    \begin{1=}
      \key{ARGUMENT-VALUE} \\
      \key{ENVIRONMENT}
      \begin{1=}
        \identifier \\
        \literal
      \end{1=} \\
      \key{ENVIRONMENT-VALUE} \\
    \end{1=}

    \begin{0+}
      ON
      \begin{1=}
        \key{EXCEPTION} \\
        \key{ESCAPE}
      \end{1=}
      \imperativestatement \\

      \key{NOT} ON
      \begin{1=}
        \key{EXCEPTION} \\
        \key{ESCAPE}
      \end{1=}
      \imperativestatement \\
    \end{0+}
  \end{minipage}
\end{syntax}

\format{message}
\begin{syntax}[\deletedcolour]
  \pending{\key{ACCEPT} \cdname MESSAGE \key{COUNT}}
\end{syntax}

\format{from screen}
\begin{syntax}[\miscextcolour]
  \key{ACCEPT} \identifier \key{FROM} \key{SCREEN}
  
  \begin{1=}
    \begin{1=}
      \begin{1+}
        AT \key{LINE} NUMBER
        \begin{1=}
          \identifier \\
          \integer
        \end{1=} \\

        AT
        \begin{1=}
          \key{COLUMN} \\
          \key{COL} \\
          \miscext{\key{POSITION}}
        \end{1=}
        NUMBER
        \begin{1=}
          \identifier \\
          \integer
        \end{1=}
      \end{1+} \\

      \miscext{
        \key{AT}
        \begin{1=}
          \identifier \\
          \integer
        \end{1=}
      }
    \end{1=} \\

    \key{SIZE} IS
    \begin{1=}
      \identifier \\
      \literal
    \end{1=}
  \end{1=}\ldots
\end{syntax}

\subsubsection{Syntax rules}

\subsubsection{General rules}

\section{ADD statement}

The ADD statement adds two or more numbers and stores the result.

\format{simple}
\begin{syntax}
  \key{ADD}
  \begin{1=}
    \identifier \\
    \literal \\
  \end{1=} \ldots
  \key{TO}
  \begin{1=}
    \identifier
  \end{1=} \ldots

  \begin{0+}
    ON \key{SIZE} \key{ERROR} \imperativestatement \\
    \key{NOT} ON \key{SIZE} \key{ERROR} \imperativestatement
  \end{0+}

  \begin{0-1}
    \key{END-ADD}
  \end{0-1}
\end{syntax}

\format{giving}
\begin{syntax}
  \key{ADD}
  \begin{1=}
    \identifier \\
    \literal \\
  \end{1=} \ldots
  \begin{0-1}
    \key{TO}
    \begin{0-1}
      \identifier
    \end{0-1} \ldots
  \end{0-1}

  \key{GIVING}
  \begin{1=}
    \identifier
    \begin{0-1}
      \metaelement{rounded-phrase}
    \end{0-1}
  \end{1=} \ldots

  \begin{0+}
    ON \key{SIZE} \key{ERROR} \imperativestatement \\
    \key{NOT} ON \key{SIZE} \key{ERROR} \imperativestatement
  \end{0+}

  \begin{0-1}
    \key{END-ADD}
  \end{0-1}
\end{syntax}

\format{corresponding}
\begin{syntax}
  \key{ADD}
  \begin{1=}
    \key{CORRESPONDING} \\
    \key{CORR}
  \end{1=}
  \identifier \key{TO} \identifier
  \begin{0-1}
    \metaelement{rounded-phrase}
  \end{0-1}

  \begin{0+}
    ON \key{SIZE} \key{ERROR} \imperativestatement \\
    \key{NOT} ON \key{SIZE} \key{ERROR} \imperativestatement
  \end{0+}

  \begin{0-1}
    \key{END-ADD}
  \end{0-1}
\end{syntax}

\format{table}
\begin{syntax}
  \pending{
    \key{ADD} \key{TABLE} \identifier \key{TO} \identifier
    \begin{0-1}
      \metaelement{rounded-phrase}
    \end{0-1}
  }

  \pending{
    \begin{0-1}
      \key{FROM} INDEX \integer \key{TO} \integer
    \end{0-1}
  }

  \pending{
    \begin{0-1}
      \key{DESTINATION} INDEX \integer
    \end{0-1}
  }

  \pending{
    \begin{0+}
      ON \key{SIZE} \key{ERROR} \imperativestatement \\
      \key{NOT} ON \key{SIZE} \key{ERROR} \imperativestatement
    \end{0+}
  }

  \pending{
    \begin{0-1}
      \key{END-ADD}
    \end{0-1}
  }
\end{syntax}

\subsubsection{Syntax rules}

\subsubsection{General rules}

\section{ALLOCATE statement}

The ALLOCATE statement requests memory from the operating system for a BASED data item or to be referenced by a data-pointer.

\begin{syntax}
  \key{ALLOCATE}
  \begin{1=}
    \identifier
    \begin{0-1}
      \key{INITIALIZED}
    \end{0-1} \\
    \arithmeticexpression
    \begin{0-1}
      \key{INITIALIZED}
      \gnucobol{
        \begin{0-1}
          \key{TO}
          \begin{1=}
            \identifier \\
            \literal
          \end{1=}
        \end{0-1}
      }
    \end{0-1}
  \end{1=}

  \begin{0-1}
    \key{RETURNING} \identifier
  \end{0-1}
\end{syntax}

\subsubsection{Syntax rules}

\subsubsection{General rules}

\section{ALTER statement}

The ALTER statement changes the target of a GO TO statement. Its use is strongly discouraged and commonly proscribed.

\begin{syntax}[\deletedcolour]
    \key{ALTER}
    \begin{1=}
      \procedurename TO PROCEED \key{TO} \procedurename
    \end{1=} \ldots
\end{syntax}

\subsubsection{Syntax rules}

\subsubsection{General rules}

\section{CALL statement}

The CALL statement transfers execution to another program, optionally with arguments and storing a return value.

\begin{syntax}
  \key{CALL}

  \miscext{
    \begin{0-1}
      \mnemonicname \\
    \end{0-1}
    \begin{0-1}
      \pending{IN \key{THREAD}}
    \end{0-1}
  }

  \begin{1=}
    \identifier \\
    \literal \\
    \functionname
    \begin{0-1}
      \key{AS}
      \begin{1=}
        \key{NESTED} \\
        \metaelement{program-prototype-name}
      \end{1=}
    \end{0-1}
  \end{1=}

  \miscext{
    \begin{0-1}
      \pending{\key{HANDLE} IN \identifier}
    \end{0-1}
  }

  \begin{0-1}
    \key{USING}
    \begin{1=}
      \begin{0-1}
        BY
        \begin{1=}
          \key{REFERENCE} \\
          \key{CONTENT} \\
          \key{VALUE}
        \end{1=}
      \end{0-1}
      \begin{1=}
        \key{OMITTED} \\

        \gnucobol{
          \begin{0-1}
            size-phrase
          \end{0-1}
        }
        \begin{1=}
          \identifier \\
          \literal
        \end{1=}
      \end{1=}
    \end{1=}\ldots
  \end{0-1}

  \begin{0-1}
    \begin{1=}
      \key{RETURNING} \\
      \miscext{\key{GIVING}}
    \end{1=}
    \begin{1=}
      INTO \identifier \\
      \key{ADDRESS} OF \identifier \\
      \gnucobol{\key{NOTHING}} \\
      \key{NULL} \\
      \key{OMITTED} \\
    \end{1=}
  \end{0-1}

  \begin{0+}
    ON
    \begin{1=}
      \key{EXCEPTION} \\
      \archaic{\key{OVERFLOW}}
    \end{1=}
    \imperativestatement \\
    \key{NOT} ON \key{EXCEPTION} \imperativestatement
  \end{0+}

  \begin{0-1}
    \key{END-CALL}
  \end{0-1}
\end{syntax}

\subsubsection{Syntax rules}

\subsubsection{General rules}

\section{CANCEL statement}

The CANCEL statement unloads a program from the operating system memory. This has the effect of freeing all the program's working-storage items and closing all files the program left open.

\begin{syntax}
  \key{CANCEL}
  \begin{1=}
    \identifier \\
    \literal
  \end{1=} \ldots
\end{syntax}

\subsubsection{Syntax rules}

\subsubsection{General rules}

\section{CLOSE statement}

The CLOSE statement prevents the program from accessing or modifying an open file. It first processes all pending writes for the file and releases all locks still held in the file.

\format{file}
\begin{syntax}
  \key{CLOSE}
  \begin{1=}
    \filename
    \begin{0-1}
      \begin{1=}
        \key{REEL} \\
        \key{UNIT}
      \end{1=}
      \begin{0-1}
        FOR \key{REMOVAL}
      \end{0-1} \\

      WITH \key{NO} \key{REWIND} \\
      WITH \key{LOCK}
    \end{0-1}
  \end{1=} \ldots
\end{syntax}

\format{window}
\begin{syntax}[\miscextcolour]
  \pending{
    \key{CLOSE} \key{WINDOW} \identifier
    \begin{0-1}
      WITH \key{NO} \key{DISPLAY}
    \end{0-1}
  }
\end{syntax}

\subsubsection{Syntax rules}

\subsubsection{General rules}

\section{COMMIT statement}

The COMMIT statement forces all pending file writes to be processed.

\begin{syntax}[\miscextcolour]
  \key{COMMIT}
\end{syntax}

\subsubsection{Syntax rules}

\subsubsection{General rules}

\section{COMPUTE statement}

The COMPUTE statement evaluates an arithmetic expression and stores the result.

\begin{syntax}
  \key{COMPUTE}
  \begin{1=}
    \identifier
    \begin{0-1}
      \metaelement{rounded-phrase}
    \end{0-1}
  \end{1=} \ldots
  \begin{1=}
    = \\
    \miscext{\key{EQUAL}} \\
    \miscext{\key{EQUALS}}
  \end{1=}
  \arithmeticexpression

  \begin{0+}
    ON \key{SIZE} \key{ERROR} \imperativestatement \\
    \key{NOT} ON \key{SIZE} \key{ERROR} \imperativestatement
  \end{0+}

  \begin{0-1}
    \key{END-COMPUTE}
  \end{0-1}
\end{syntax}

\subsubsection{Syntax rules}

\subsubsection{General rules}

\section{CONTINUE statement}

The CONTINUE statement has no effect on program execution.

\begin{syntax}
  \key{CONTINUE}
\end{syntax}

\subsubsection{Syntax rules}

\subsubsection{General rules}

\section{DELETE statement}

The DELETE statement removes a record from a file or removes an entire file.

\format{record}
\begin{syntax}
  \key{DELETE} \filename RECORD

  \begin{0-1}
    \metaelement{retry-phrase}
  \end{0-1}

  \begin{0+}
    \key{INVALID} KEY \imperativestatement \\
    \key{NOT} \key{INVALID} KEY \imperativestatement
  \end{0+}

  \begin{0-1}
    \key{END-DELETE}
  \end{0-1}
\end{syntax}

\format{file}
\begin{syntax}[\miscextcolour]
  \key{DELETE} \key{FILE}
  \begin{1=}
    \filename
  \end{1=} \ldots
  \gnucobol{
    \begin{0-1}
      \key{END-DELETE}
    \end{0-1}
  }
\end{syntax}

\subsubsection{Syntax rules}

\subsubsection{General rules}

\section{DESTROY statement}

\begin{syntax}[\miscextcolour]
  \pending{
    \key{DESTROY}
    \begin{1=}
      \key{ALL} CONTROLS \\
      \identifier\ldots
    \end{1=}
  }
\end{syntax}

\subsubsection{Syntax rules}

\subsubsection{General rules}

\section{DISABLE statement}

The DISABLE statement prevents the program from modifying or accessing an enabled communication descriptor.

\begin{syntax}[\deletedcolour]
  \pending{
    \key{DISABLE}
    \begin{0-1}
      \key{INPUT} TERMINAL \\
      \key{OUTPUT} \\
      \key{I-O} \key{TERMINAL} \\
      \miscext{\key{TERMINAL}}
    \end{0-1}
    \cdname
    \begin{0-1}
      WITH \key{KEY}
      \begin{1=}
        \identifier \\
        \literal
      \end{1=}
    \end{0-1}
  }
\end{syntax}

\subsubsection{Syntax rules}

\subsubsection{General rules}

\section{DISPLAY statement}

The DISPLAY statement displays data to a user or sends data to the operating system.

\format{device}
\begin{syntax}
  \key{DISPLAY}
  \begin{1=}
    \identifier \\
    \literal
  \end{1=} \ldots
  \begin{0+}
    \key{UPON} \mnemonicname \\
    WITH \key{NO} \key{ADVANCING}
  \end{0+}

  \begin{0+}
    ON \key{EXCEPTION} \imperativestatement \\
    \key{NOT} ON \key{EXCEPTION} \imperativestatement \\
  \end{0+}

  \begin{0-1}
    \key{END-DISPLAY}
  \end{0-1}
\end{syntax}

\format{environment}
\begin{syntax}[\miscextcolour]
  \key{DISPLAY}
  \begin{1=}
    \identifier \\
    \literal
  \end{1=}
  \key{UPON}
  \begin{1=}
    \key{ARGUMENT-NUMBER} \\
    \key{COMMAND-LINE} \\
    \key{ENVIRONMENT-NAME} \\
    \key{ENVIRONMENT-VALUE} \\
  \end{1=}

  \begin{0+}
    ON \key{EXCEPTION} \imperativestatement \\
    \key{NOT} ON \key{EXCEPTION} \imperativestatement \\
  \end{0+}

  \begin{0-1}
    \key{END-DISPLAY}
  \end{0-1}
\end{syntax}

\format{screen}
\begin{syntax}
  \key{DISPLAY}
  \begin{1=}
    \begin{1=}
      \identifier \\
      \miscext{\literal} \\
      \miscext{\key{OMITTED}}
    \end{1=}

    \begin{0+}
      \metaelement{position-clauses} \\

      \miscext{
        \key{UPON}
        \begin{1=}
          \key{CRT} \\
          \key{CRT-UNDER}
        \end{1=}
      } \\

      \miscext{\key{MODE} IS \key{BLOCK}} \\
      \miscext{\metaelement{appearance-attribute-clauses}}
    \end{0+}
  \end{1=} \miscext{\ldots}

  \begin{0+}
    ON \key{EXCEPTION} \imperativestatement \\
    \key{NOT} ON \key{EXCEPTION} \imperativestatement \\
  \end{0+}

  \begin{0-1}
    \key{END-DISPLAY}
  \end{0-1}
\end{syntax}

\format{ordinary window}
\begin{syntax}[\miscextcolour]
  \key{DISPLAY}
  \begin{1=}
    \key{WINDOW} \\
    \key{SUBWINDOW}
  \end{1=}
  \begin{0-1}
    \key{UPON} \identifier
  \end{0-1}

  \begin{1=}
    \key{POP-UP} AREA IS \identifier \\

    \key{HANDLE}
    \begin{0-1}
      IS \\
      IN
    \end{0-1}
    \identifier \\

    \key{LINES} \integer \\
    
    \key{TITLE} IS
    \begin{1=}
      \identifier \\
      \literal
    \end{1=} \\

    WITH \key{NO}
    \begin{0-1}
      \key{SCROLL} \\
      \key{WRAP}
    \end{0-1} \\

    \begin{0-1}
      \key{SHADOW} \\
      \key{BOXED}
    \end{0-1} \\
    
    \metaelement{position-clauses} \\
    \metaelement{appearance-attribute-clauses}
  \end{1=}\ldots

  \begin{0-1}
    \key{END-DISPLAY}
  \end{0-1}
\end{syntax}

\format{floating window}
\begin{syntax}[\miscextcolour]
  \key{DISPLAY} \key{FLOATING}
  \begin{0-1}
    \key{GRAPHICAL}
  \end{0-1}
  \key{WINDOW}
  \begin{0-1}
    \key{UPON} \identifier
  \end{0-1}

  \begin{1=}
    \key{POP-UP} AREA IS \identifier \\

    \key{HANDLE}
    \begin{0-1}
      IS \\
      IN
    \end{0-1}
    \identifier \\

    \key{LINES} \integer \\
    
    \key{TITLE} IS
    \begin{1=}
      \identifier \\
      \literal
    \end{1=} \\

    WITH \key{NO}
    \begin{0-1}
      \key{SCROLL} \\
      \key{WRAP}
    \end{0-1} \\

    \begin{0-1}
      \key{SHADOW} \\
      \key{BOXED}
    \end{0-1} \\

    \metaelement{position-clauses} \\
    \metaelement{appearance-attribute-clauses}
  \end{1=}\ldots

  \begin{0-1}
    \key{END-DISPLAY}
  \end{0-1}
\end{syntax}

\format{special window}
\begin{syntax}[\miscextcolour]
  \key{DISPLAY}
  \begin{1=}
    \key{INITIAL} \\
    \key{STANDARD} \\
    \key{INDEPENDENT}
  \end{1=}
  \begin{0-1}
    \key{GRAPHICAL}
  \end{0-1}
  \key{WINDOW}
  \begin{0-1}
    \key{UPON} \identifier
  \end{0-1}

  \begin{1=}
    \key{POP-UP} AREA IS \identifier \\

    \key{HANDLE}
    \begin{0-1}
      IS \\
      IN
    \end{0-1}
    \identifier \\

    \key{LINES} \integer \\
    
    \key{TITLE} IS
    \begin{1=}
      \identifier \\
      \literal
    \end{1=} \\

    WITH \key{NO}
    \begin{0-1}
      \key{SCROLL} \\
      \key{WRAP}
    \end{0-1} \\

    \begin{0-1}
      \key{SHADOW} \\
      \key{BOXED}
    \end{0-1} \\

    \metaelement{position-clauses} \\
    \metaelement{appearance-attribute-clauses}
  \end{1=}\ldots

  \begin{0-1}
    \key{END-DISPLAY}
  \end{0-1}
\end{syntax}

\format{message box}
\begin{syntax}[\miscextcolour]
  \key{DISPLAY} \key{MESSAGE} BOX
  \begin{1=}
    \identifier \\
    \literal
  \end{1=}\ldots
  
  \begin{0-1}
    \key{TITLE}
    \begin{0-1}
      IS \\
      =
    \end{0-1}
    \begin{1=}
      \identifier \\
      \literal
    \end{1=} \\

    \key{TYPE}
    \begin{0-1}
      IS \\
      =
    \end{0-1}
    \begin{1=}
      \identifier \\
      \literal
    \end{1=} \\

    \key{ICON}
    \begin{0-1}
      IS \\
      =
    \end{0-1}
    \begin{1=}
      \identifier \\
      \literal
    \end{1=} \\

    \key{DEFAULT}
    \begin{0-1}
      IS \\
      =
    \end{0-1}
    \begin{1=}
      \identifier \\
      \literal
    \end{1=} \\

    \begin{1=}
      \key{RETURNING} \\
      \key{GIVING}
    \end{1=}
    \begin{1=}
      \identifier \\
      \literal
    \end{1=} \\
  \end{0-1}\ldots

  \begin{0-1}
    \key{END-DISPLAY}
  \end{0-1}
\end{syntax}

where \metaelement{position-clauses} is

\begin{syntax}
  \begin{1=}
    \begin{1+}
      \key{LINE} NUMBER
      \begin{1=}
        \identifier \\
        \literal
      \end{1=} \\

      \key{AT}
      \begin{1=}
        \key{COLUMN} \\
        \key{COL} \\
        \miscext{\key{POSITION}}
      \end{1=}
      NUMBER
      \begin{1=}
        \identifier \\
        \literal
      \end{1=}
    \end{1+} \\


    \miscext{
      \key{AT}
      \begin{1=}
        \identifier \\
        \literal
      \end{1=}
    }
  \end{1=}
\end{syntax}

where \metaelement{appearance-attribute-clauses} is

\begin{syntax}[\miscextcolour]
  WITH
  \begin{1=}
    \key{BELL} \\
    \key{BEEP} \\
  \end{1=} \\

  WITH \key{BLANK}
  \begin{1=}
    \key{LINE} \\
    \key{SCREEN}
  \end{1=} \\

  WITH
  \begin{1=}
    \key{BLINK} \\
    \key{BLINKING}
  \end{1=} \\

  WITH
  \begin{1=}
    \key{CONVERSION} \\
    \key{CONVERT}
  \end{1=} \\

  WITH \key{ERASE}
  \begin{1=}
    \key{EOL} \\
    \key{EOS} \\

    \begin{0-1}
      \miscext{TO} \key{END} OF
    \end{0-1}
    \begin{1=}
      \key{LINE} \\
      \key{SCREEN}
    \end{1=}
  \end{1=} \\

  WITH
  \begin{1=}
    \key{HIGHLIGHT} \\
    \key{HIGH} \\
    \key{BOLD} \\
    \key{LOWLIGHT} \\
    \key{LOW}
  \end{1=} \\

  \pending{WITH \key{STANDARD}} \\
  \pending{WITH \key{BACKGROUND-HIGH}} \\
  \pending{WITH \key{BACKGROUND-STANDARD}} \\
  \pending{WITH \key{BACKGROUND-LOW}} \\
  
  WITH \key{OVERLINE} \\

  WITH
  \begin{1=}
    \key{REVERSE-VIDEO} \\
    \key{REVERSED} \\
    \key{REVERSE}
  \end{1=} \\

  WITH \key{SIZE} IS
  \begin{1=}
    \identifier \\
    \literal
  \end{1=} \\

  WITH
  \begin{1=}
    \key{UNDERLINE} \\
    \key{UNDERLINED}
  \end{1=} \\

  WITH
  \begin{1=}
    \key{FOREGROUND-COLOR} \\
    \key{FOREGROUND-COLOUR}
  \end{1=}
  IS
  \begin{1=}
    \identifier \\
    \integer
  \end{1=} \\

  WITH
  \begin{1=}
    \key{BACKGROUND-COLOR} \\
    \key{BACKGROUND-COLOUR}
  \end{1=}
  IS
  \begin{1=}
    \identifier \\
    \integer
  \end{1=} \\

  \pending{
    WITH
    \begin{1=}
      \key{COLOR} \\
      \key{COLOUR}
    \end{1=}
    IS
    \begin{1=}
      \identifier \\
      \integer
    \end{1=}
  } \\

  WITH \key{SCROLL}
  \begin{1=}
    \key{UP} \\
    \key{DOWN}
  \end{1=}
  \begin{0-1}
    \begin{1=}
      \identifier \\
      \integer
    \end{1=}
    \begin{1=}
      \key{LINE} \\
      \key{LINES}
    \end{1=}
  \end{0-1}
\end{syntax}

\subsubsection{Syntax rules}

\subsubsection{General rules}

\section{DIVIDE statement}

The DIVIDE statement divides one or more numbers by another and stores the results.

\format{into}
\begin{syntax}
  \key{DIVIDE}
  \begin{1=}
    \identifier \\
    \literal
  \end{1=}
  \key{INTO}
  \begin{1=}
    \begin{1=}
      \identifier \\
      \literal
    \end{1=}
    \begin{0-1}
      \metaelement{rounded-phrase}
    \end{0-1}
  \end{1=} \ldots

  \begin{0+}
    ON \key{SIZE} \key{ERROR} \imperativestatement \\
    \key{NOT} ON \key{SIZE} \key{ERROR} \imperativestatement
  \end{0+}

  \begin{0-1}
    \key{END-DIVIDE}
  \end{0-1}
\end{syntax}

\format{giving}
\begin{syntax}
  \key{DIVIDE}
  \begin{1=}
    \identifier \\
    \literal
  \end{1=}
  \begin{1=}
    \key{BY} \\
    \key{INTO}
  \end{1=}
  \begin{1=}
    \identifier \\
    \literal
  \end{1=}

  \key{GIVING}
  \begin{1=}
    \begin{1=}
      \identifier \\
      \literal
    \end{1=}
    \begin{0-1}
      \metaelement{rounded-phrase}
    \end{0-1}
  \end{1=}
  \ldots

  \begin{0-1}
    \key{REMAINDER}
    \begin{1=}
      \identifier \\
      \literal
    \end{1=}
  \end{0-1}

  \begin{0+}
    ON \key{SIZE} \key{ERROR} \imperativestatement \\
    \key{NOT} ON \key{SIZE} \key{ERROR} \imperativestatement
  \end{0+}

  \begin{0-1}
    \key{END-DIVIDE}
  \end{0-1}
\end{syntax}

\subsubsection{Syntax rules}

\subsubsection{General rules}

\section{ENABLE statement}

The ENABLE statement allows the program to access and modify a communication descriptor.

\begin{syntax}[\deletedcolour]
  \pending{
    \key{ENABLE}
    \begin{0-1}
      \key{INPUT} TERMINAL \\
      \key{OUTPUT} \\
      \key{I-O} \key{TERMINAL} \\
      \miscext{\key{TERMINAL}}
    \end{0-1}
    \cdname
    \begin{0-1}
      WITH \key{KEY}
      \begin{1=}
        \identifier \\
        \literal
      \end{1=}
    \end{0-1}
  }
\end{syntax}

\subsubsection{Syntax rules}

\subsubsection{General rules}

\section{ENTRY statement}

The ENTRY statement indicates an alternative point of entry into the program.

\begin{syntax}[\miscextcolour]
  \begin{minipage}[!h]{1.0\linewidth}
    \key{ENTRY}
    \begin{0-1}
      \mnemonicname
    \end{0-1}
    \literal

    \begin{0-1}
      \key{USING}

      \begin{1=}
        \begin{0-1}
          BY
          \begin{1=}
            \key{REFERENCE} \\
            \gnucobol{\key{CONTENT}} \\
            \key{VALUE}
          \end{1=}
        \end{0-1}

        \begin{1=}
          \gnucobol{\key{OMITTED}} \\

          \begin{1=}
            \gnucobol{
              \begin{0-1}
                size-phrase
              \end{0-1}
            }
            \begin{1=}
              \identifier \\
              \literal
            \end{1=}
          \end{1=}
        \end{1=}\ldots
      \end{1=}
    \end{0-1}
  \end{minipage}
\end{syntax}

\subsubsection{Syntax rules}

\subsubsection{General rules}

\section{EVALUATE statement}

The EVALUATE statement evaluates one or more conditions and execute the statements corresponding to the first true condition.

\begin{syntax}
  \key{EVALUATE}
  \begin{1=}
    \expression \\
    \key{TRUE} \\
    \key{FALSE}
  \end{1=}
  \begin{0-1}
    \key{ALSO}
    \begin{1=}
      \expression \\
      \key{TRUE} \\
      \key{FALSE}
    \end{1=}
  \end{0-1} \ldots

  \begin{1=}
    \key{WHEN}
    \metaelement{selection-object}
    \begin{0-1}
      \key{ALSO} \metaelement{selection-object}
    \end{0-1}\ldots\ {}
    \imperativestatement
  \end{1=} \ldots

  \begin{0-1}
    \key{WHEN} \key{OTHER} \imperativestatement
  \end{0-1}

  \begin{0-1}
    \key{END-EVALUATE}
  \end{0-1}
\end{syntax}

where \metaelement{selection-object} is

\begin{syntax}
  \begin{1=}
    \metaelement{partial-expression-1}
    \begin{0-1}
      \begin{1=}
        \key{THROUGH} \\
        \key{THRU}
      \end{1=}
      \expression
    \end{0-1} \\

    \key{ANY} \\
    \key{TRUE} \\
    \key{FALSE}
  \end{1=}
\end{syntax}

\subsubsection{Syntax rules}

\subsubsection{General rules}

\section{EXIT statement}

The EXIT statement indicates the end of a path of execution.

\begin{syntax}
  \key{EXIT}
  \begin{0-1}
    \key{FUNCTION} \\
    \key{PARAGRAPH} \\

    \key{PERFORM}
    \begin{0-1}
      \key{CYCLE}
    \end{0-1} \\

    \key{PROGRAM}
    \miscext{
      \begin{0-1}
        \begin{1=}
          \key{RETURNING} \\
          \key{GIVING}
        \end{1=}
      \end{0-1}
      \begin{1=}
        \identifier \\
        \literal
      \end{1=}
    } \\

    \key{SECTION} \\
  \end{0-1}
\end{syntax}

\subsubsection{Syntax rules}

\subsubsection{General rules}

\section{FREE statement}

The FREE statement returns memory to the operating system.

\begin{syntax}
  \key{FREE}
  \begin{1=}
    \identifier
  \end{1=} \ldots
\end{syntax}

\subsubsection{Syntax rules}

\subsubsection{General rules}

\section{GENERATE statement}

The GENERATE statement a specified report entry.

\begin{syntax}
  \pending{\key{GENERATE} \reportname}
\end{syntax}

\subsubsection{Syntax rules}

\subsubsection{General rules}

\section{GO TO statement}

The GO TO statement transfer execution to another part of the program.

\begin{syntax}
  \key{GO} TO
  \begin{1=}
    \procedurename
  \end{1=} \ldots
  \begin{0-1}
    \key{DEPENDING} ON \identifier
  \end{0-1}
\end{syntax}

\subsubsection{Syntax rules}

\subsubsection{General rules}

\section{GOBACK statement}

The GOBACK statement terminates execution in the program, returning control to the calling program or, if no such program exists, to the operating system.

\begin{syntax}
  \key{GOBACK}
  \begin{0-1}
    \begin{1=}
      \key{RETURNING} \\
      \miscext{\key{GIVING}}
    \end{1=}
    \begin{1=}
      \identifier \\
      \literal
    \end{1=}
  \end{0-1}
\end{syntax}

\subsubsection{Syntax rules}

\subsubsection{General rules}

\section{IF statement}

The IF statement evaluates a condition and executes statements depending on whether the condition was true or false.

\begin{syntax}
  \key{IF} \metaelement{condition} THEN
  \begin{1=}
    \imperativestatement \\
    \key{ELSE} \imperativestatement
  \end{1=} \ldots

  \begin{0-1}
    \key{END-IF}
  \end{0-1}
\end{syntax}

\subsubsection{Syntax rules}

\subsubsection{General rules}

\section{INITIALIZE statement}

The INITIALIZE statement sets data items to their default values.

\begin{syntax}
  \begin{1=}
    \key{INITIALIZE} \\
    \miscext{\key{INITIALISE}}
  \end{1=}
  \begin{1=}
    \identifier \\
    \metaelement{basic-literal-1}
  \end{1=} \ldots
  \begin{0-1}
    WITH \key{FILLER}
  \end{0-1}

  \begin{0-1}
    \begin{1=}
      \key{ALL} \\
      \key{ALPHABETIC} \\
      \key{ALPHANUMERIC} \\
      \key{ALPHANUMERIC-EDITED} \\
      \key{NATIONAL} \\
      \key{NATIONAL-EDITED} \\
      \key{NUMERIC} \\
      \key{NUMERIC-EDITED}
    \end{1=}
    TO \key{VALUE}
  \end{0-1}

  \begin{0-1}
    \key{REPLACING}
    \begin{1=}
      \begin{1=}
        \key{ALPHABETIC} \\
        \key{ALPHANUMERIC} \\
        \key{ALPHANUMERIC-EDITED} \\
        \key{NATIONAL} \\
        \key{NATIONAL-EDITED} \\
        \key{NUMERIC} \\
        \key{NUMERIC-EDITED}
      \end{1=}
      DATA \key{BY}
      \begin{1=}
        \identifier \\
        \literal
      \end{1=}
    \end{1=} \ldots
  \end{0-1}

  \begin{0-1}
    THEN TO \key{DEFAULT}
  \end{0-1}
\end{syntax}

\subsubsection{Syntax rules}

\subsubsection{General rules}

\section{INITIATE statement}

The INITIATE statement allows the program to begin generating the specified report.

\begin{syntax}
  \pending{
    \key{INITIATE}
    \begin{1=}
      \reportname
    \end{1=} \ldots
  }
\end{syntax}

\subsubsection{Syntax rules}

\subsubsection{General rules}

\section{INSPECT statement}

The INSPECT statement counts the number of occurrences of a character string, replaces occurrences or both.

\begin{syntax}
  \key{INSPECT}
  \begin{1=}
    \identifier \\
    \literal \\
    \functionname
  \end{1=}
  \begin{1=}
    \metaelement{tallying-phrase}
    \begin{0-1}
      \metaelement{replacing-phrase}
    \end{0-1} \\

    \metaelement{replacing-phrase} \\
    \metaelement{converting-phrase}
  \end{1=}
\end{syntax}

where \metaelement{tallying-phrase} is

\begin{syntax}
  \key{TALLYING}
  \begin{1=}
    \begin{1=}
      \begin{1=}
        \identifier \\
        \literal
      \end{1=}
      \key{FOR}
      \begin{1=}
        \key{CHARACTERS} \\

        \begin{1=}
          \key{ALL} \\
          \key{LEADING} \\
          \key{TRAILING}
        \end{1=}
        \begin{1=}
          \identifier \\
          \literal
        \end{1=}
      \end{1=}
    \end{1=}\ldots
    \begin{0-1}
      before-after-phrase
    \end{0-1}
  \end{1=} \ldots
\end{syntax}

where \metaelement{replacing-phrase} is

\begin{syntax}
  \key{REPLACING}
  \begin{1=}
    \begin{1=}
      \key{CHARACTERS} \\

      \begin{0-1}
        \key{ALL} \\
        \key{LEADING} \\
        \key{FIRST} \\
        \key{TRAILING}
      \end{0-1}
      \begin{1=}
        \identifier \\
        \literal
      \end{1=}
    \end{1=}
    \key{BY}
    \begin{1=}
      \identifier \\
      \literal \\
    \end{1=}
    \begin{0-1}
      \metaelement{before-after-phrase}
    \end{0-1} \\
  \end{1=} \ldots
\end{syntax}

where \metaelement{converting-phrase} is

\begin{syntax}
  \key{CONVERTING}
  \begin{1=}
    \identifier \\
    \literal
  \end{1=}
  \key{TO}
  \begin{1=}
    \identifier \\
    \literal
  \end{1=}
  \begin{0-1}
    \metaelement{before-after-phrase}
  \end{0-1}
\end{syntax}

where \metaelement{before-after-phrase} is

\begin{syntax}
  \begin{0+}
    \key{BEFORE} INITIAL
    \begin{1=}
      \identifier \\
      \literal
    \end{1=} \\

    \key{AFTER} INITIAL
    \begin{1=}
      \identifier \\
      \literal
    \end{1=}
  \end{0+}
\end{syntax}

\subsubsection{Syntax rules}

\subsubsection{General rules}

\section{MERGE statement}

The MERGE statements reads multiple files with the same record desciption, combines their records and sorts them.

\begin{syntax}
  \key{MERGE} \identifier
  \begin{0-1}
    ON
    \begin{1=}
      \key{ASCENDING} \\
      \key{DESCENDING}
    \end{1=}
    KEY
    \begin{0-1}
      \identifier
    \end{0-1}\ldots
  \end{0-1} \ldots

  \begin{0-1}
    WITH \key{DUPLICATES}
    \begin{0-1}
      IN \key{ORDER}
    \end{0-1}
  \end{0-1}

  \begin{0-1}
    COLLATING \key{SEQUENCE} IS \identifier
  \end{0-1}

  \begin{0-1}
    \key{USING}
    \begin{1=}
      \filename
    \end{1=}\ldots
  \end{0-1}

  \begin{0-1}
    \key{GIVING}
    \begin{1=}
      \filename
    \end{1=}\ldots \\

    \key{OUTPUT} \key{PROCEDURE} IS
    \procedurename
    \begin{0-1}
      \begin{1=}
        \key{THROUGH} \\
        \key{THRU}
      \end{1=}
      \procedurename
    \end{0-1}
  \end{0-1}
\end{syntax}

\subsubsection{Syntax rules}

\subsubsection{General rules}

\section{MOVE statement}

The MOVE statement sets the value of one or more data items.

\begin{syntax}
  \key{MOVE}
  \begin{0-1}
    \key{CORRESPONDING} \\
    \key{CORR}
  \end{0-1}
  \begin{1=}
    \identifier \\
    \literal
  \end{1=}
  \key{TO}
  \begin{1=}
    \identifier
  \end{1=} \ldots
\end{syntax}

\subsubsection{Syntax rules}

\subsubsection{General rules}

\section{MULTIPLY statement}

The MULTIPLY statement multiplies multiple numbers and stores the result.

\format{simple}
\begin{syntax}
  \key{MULTIPLY}
  \begin{1=}
    \identifier \\
    \literal
  \end{1=}
  \key{BY}
  \begin{1=}
    \begin{1=}
      \identifier \\
      \literal
    \end{1=}
    \begin{0-1}
      \metaelement{rounded-phrase}
    \end{0-1}
  \end{1=} \ldots

  \begin{0+}
    ON \key{SIZE} \key{ERROR} \imperativestatement \\
    \key{NOT} ON \key{SIZE} \key{ERROR} \imperativestatement
  \end{0+}

  \begin{0-1}
    \key{END-MULTIPLY}
  \end{0-1}
\end{syntax}

\format{giving}
\begin{syntax}
  \key{MULTIPLY}
  \begin{1=}
    \identifier \\
    \literal
  \end{1=}
  \key{BY}
  \begin{1=}
    \identifier \\
    \literal
  \end{1=}

  \key{GIVING}
  \begin{1=}
    \begin{1=}
      \identifier \\
      \literal
    \end{1=}
    \begin{0-1}
      \metaelement{rounded-phrase}
    \end{0-1}
  \end{1=} \ldots

  \begin{0+}
    ON \key{SIZE} \key{ERROR} \imperativestatement \\
    \key{NOT} ON \key{SIZE} \key{ERROR} \imperativestatement
  \end{0+}

  \begin{0-1}
    \key{END-MULTIPLY}
  \end{0-1}
\end{syntax}

\subsubsection{Syntax rules}

\subsubsection{General rules}

\section{NEXT SENTENCE statement}

The NEXT SENTENCE statement transfers execution to the first statement following the current sentence.

\begin{syntax}[\archaiccolour]
    \key{NEXT} \key{SENTENCE}
\end{syntax}

\subsubsection{Syntax rules}

\subsubsection{General rules}

\section{OPEN statement}

The OPEN statement allows the program to access or modify specified files.

\begin{syntax}
  \key{OPEN}
  \begin{1=}
    \begin{1=}
      \key{INPUT} \\
      \key{OUTPUT} \\
      \key{I-O} \\
      \key{EXTEND}
    \end{1=}
    \begin{0-1}
      \metaelement{sharing-mode}
    \end{0-1}
    \begin{0-1}
      \metaelement{retry-phrase}
    \end{0-1}
    \begin{1=}
      \filename
    \end{1=} \ldots
    \begin{0-1}
      WITH \key{NO} \key{REWIND} \\
      WITH \key{LOCK} \\
      \deleted{\key{REVERSED}}
    \end{0-1}
  \end{1=}\ldots
\end{syntax}

where \metaelement{sharing-mode} is

\begin{syntax}
  \key{SHARING} WITH
  \begin{1=}
    \key{ALL} OTHER \\
    \key{NO} OTHER \\
    \key{READ} \key{ONLY}
  \end{1=}
\end{syntax}

\subsubsection{Syntax rules}

\subsubsection{General rules}

\section{PERFORM statement}

The PERFORM statement executes the specified procedures or statements one or more times.

\format{procedure}
\begin{syntax}
  \key{PERFORM}
  \miscext{
    \begin{0-1}
      \pending{IN \key{THREAD}}
    \end{0-1}
  }
  \procedurename
  \begin{0-1}
    \begin{1=}
      \key{THROUGH} \\
      \key{THRU}
    \end{1=}
    \procedurename
  \end{0-1}

  \miscext{
    \begin{0-1}
      \pending{\key{HANDLE} IN \identifier}
    \end{0-1}
  }
  \begin{0-1}
    \gnucobol{\key{FOREVER}} \\
    \metaelement{times-phrase} \\
    \metaelement{until-phrase} \\
    \metaelement{varying-phrase}
  \end{0-1}
\end{syntax}

\format{inline}
\begin{syntax}
  \key{PERFORM}
  \miscext{
    \begin{0-1}
      \pending{IN \key{THREAD}}
    \end{0-1}
  }
  \begin{0-1}
    \gnucobol{\key{FOREVER}} \\
    \metaelement{times-phrase} \\
    \metaelement{until-phrase} \\
    \metaelement{varying-phrase}
  \end{0-1}
  \miscext{
    \begin{0-1}
      \pending{\key{HANDLE} IN \identifier}
    \end{0-1}
  }

  \imperativestatement

  \begin{0-1}
    \key{END-PERFORM}
  \end{0-1}
\end{syntax}

where \metaelement{times-phrase} is

\begin{syntax}
  \begin{1=}
    \identifier \\
    \literal \\
    \functionname
  \end{1=}
  \key{TIMES} \\
\end{syntax}

where \metaelement{until-phrase} is

\begin{syntax}
  \begin{0-1}
    WITH \key{TEST}
    \begin{1=}
      \key{BEFORE} \\
      \key{AFTER} \\
    \end{1=}
  \end{0-1}
  \key{UNTIL}
  \begin{1=}
    \condition \\
    \gnucobol{\key{EXIT}}
  \end{1=} \\
\end{syntax}

and where \metaelement{varying-phrase} is

\begin{syntax}
  \begin{0-1}
    WITH \key{TEST}
    \begin{1=}
      \key{BEFORE} \\
      \key{AFTER} \\
    \end{1=}
  \end{0-1}

  \key{VARYING} \identifier \key{FROM}
  \begin{1=}
    \identifier \\
    \literal
  \end{1=}
  \key{BY}
  \begin{1=}
    \identifier \\
    \literal
  \end{1=}
  \key{UNTIL}
  \condition

  \begin{0-1}
    \key{AFTER} \identifier \key{FROM}
    \begin{1=}
      \identifier \\
      \literal
    \end{1=}
    \key{BY}
    \begin{1=}
      \identifier \\
      \literal
    \end{1=}

    \key{UNTIL}
    \condition
  \end{0-1} \ldots
\end{syntax}

% TO-DO: Improve

\subsubsection{Syntax rules}

\subsubsection{General rules}

\section{PURGE statement}

The PURGE statement

\begin{syntax}[\deletedcolour]
  \pending{\key{PURGE} \cdname}
\end{syntax}

\subsubsection{Syntax rules}

\subsubsection{General rules}

\section{READ statement}

The READ statement transfer data from a file to the file's record or to a data item.

\begin{syntax}
  \key{READ} \filename
  \begin{0-1}
    \key{NEXT} \\
    \key{PREVIOUS}
  \end{0-1}
  RECORD
  \begin{0-1}
    \key{INTO} \identifier
  \end{0-1}

  \begin{0-1}
    \begin{1=}
      \key{IGNORING} \key{LOCK} \\
      \miscext{WITH \key{IGNORE} \key{LOCK}}
    \end{1=} \\

    \begin{0-1}
      \pending{\key{ADVANCING} ON \key{LOCK}} \\
      \pending{\metaelement{retry-phrase}}
    \end{0-1}
    \begin{0-1}
      WITH
      \begin{1=}
        \begin{0-1}
          \key{NO} \\
          \miscext{\key{KEPT}}
        \end{0-1}
        \key{LOCK} \\

        \miscext{\key{WAIT}}
      \end{1=}
    \end{0-1}
  \end{0-1}

  \begin{0-1}
    \key{KEY} IS \identifier
  \end{0-1}

  \begin{0-1}
    \begin{1+}
      \key{INVALID} \key{KEY} \imperativestatement \\
      \key{NOT} \key{INVALID} \key{KEY} \imperativestatement
    \end{1+} \\

    \begin{1+}
      AT \key{END} \imperativestatement \\
      \key{NOT} AT \key{END} \imperativestatement
    \end{1+}
  \end{0-1}

  \begin{0-1}
    \key{END-READ}
  \end{0-1}
\end{syntax}

\subsubsection{Syntax rules}

\subsubsection{General rules}

\section{READY statement}

The READY TRACE statement causes the name of procedures reached by execution to be displayed.

\begin{syntax}[\miscextcolour]
  \key{READY} \key{TRACE}
\end{syntax}

\subsubsection{Syntax rules}

\subsubsection{General rules}

\section{RECEIVE statement}

The RECEIVE statement transfers data from a communication descriptor to a data item.

\begin{syntax}[\deletedcolour]
  \pending{
    \key{RECEIVE} \cdname
    \begin{1=}
      \key{MESSAGE} \\
      \key{SEGMENT}
    \end{1=}
    \key{INTO} \identifier
  }

  \pending{
    \begin{0+}
      WITH \key{DATA} \imperativestatement \\
      \key{NO} \key{DATA} \imperativestatement
    \end{0+}
  }

  \pending{
    \begin{0-1}
      \key{END-RECEIVE}
    \end{0-1}
  }
\end{syntax}

\subsubsection{Syntax rules}

\subsubsection{General rules}

\section{RELEASE statement}

The RELEASE statement provides a record for sorting.

\begin{syntax}
  \key{RELEASE} \identifier
  \begin{0-1}
    \key{FROM}
    \begin{1=}
      \identifier \\
      \literal \\
      \metaelement{function-call-1}
    \end{1=}
  \end{0-1}
\end{syntax}

\subsubsection{Syntax rules}

\subsubsection{General rules}

\section{RESET statement}

The RESET TRACE stops the names of procedures reached by execution being displayed.

\begin{syntax}[\miscextcolour]
  \key{RESET} \key{TRACE}
\end{syntax}

\subsubsection{Syntax rules}

\subsubsection{General rules}

\section{RETURN statement}

The RETURN statement retrieves records from the sorting process in order.

\begin{syntax}
  \key{RETURN} \filename RECORD
  \begin{0-1}
    \key{INTO} \identifier
  \end{0-1}

  AT \key{END} \imperativestatement

  \begin{0-1}
    \key{NOT} AT \key{END} \imperativestatement
  \end{0-1}

  \begin{0-1}
    \key{END-RETURN}
  \end{0-1}
\end{syntax}

\subsubsection{Syntax rules}

\subsubsection{General rules}

\section{REWRITE statement}

The REWRITE statement replaces an existing record in the file with one provided by the program.

\begin{syntax}
  \key{REWRITE}
  \begin{1=}
    \recordname
    \begin{0-1}
      \key{FROM}
      \begin{1=}
        \identifier \\
        \literal \\
        \functionname
      \end{1=}
    \end{0-1} \\

    \key{FILE} \filename \key{FROM}
    \begin{1=}
      \identifier \\
      \literal \\
      \functionname
    \end{1=}
  \end{1=}

  \begin{0-1}
    \pending{\metaelement{retry-phrase}}
  \end{0-1}

  \begin{0-1}
    WITH
    \begin{0-1}
      \key{NO}
    \end{0-1}
    \key{LOCK}
  \end{0-1}

  \begin{0+}
    \key{INVALID} \key{KEY} \imperativestatement \\
    \key{NOT} \key{INVALID} \key{KEY} \imperativestatement
  \end{0+} \\

  \begin{0-1}
    \key{END-REWRITE}
  \end{0-1}
\end{syntax}

\subsubsection{Syntax rules}

\subsubsection{General rules}

\section{ROLLBACK statement}

The ROLLBACK statement deletes any pending file writes.

\begin{syntax}[\miscextcolour]
  \key{ROLLBACK}
\end{syntax}

\subsubsection{Syntax rules}

\subsubsection{General rules}

\section{SEARCH statement}

The SEARCH statement iterates through a table to find a record satisfying a condition.

\format{simple}
\begin{syntax}
  \key{SEARCH} \identifier
  \begin{0-1}
    \key{VARYING} \identifier
  \end{0-1}

  \begin{0-1}
    AT \key{END} \imperativestatement
  \end{0-1}

  \begin{1=}
    \key{WHEN} \condition \imperativestatement
  \end{1=} \ldots

  \begin{0-1}
    \key{END-SEARCH}
  \end{0-1}
\end{syntax}

\format{all}
\begin{syntax}
  \key{SEARCH} \key{ALL} \identifier

  \begin{0-1}
    AT \key{END} \imperativestatement
  \end{0-1}

  \key{WHEN} \expression \imperativestatement

  \begin{0-1}
    \key{END-SEARCH}
  \end{0-1}
\end{syntax}

\subsubsection{Syntax rules}

\subsubsection{General rules}

\section{SEND statement}

The SEND statement provides data to a communication descriptor.

\format{from}
\begin{syntax}[\deletedcolour]
  \pending{
    \key{SEND} \cdname \key{FROM} \identifier
  }
\end{syntax}

\format{with indicator}
\begin{syntax}[\deletedcolour]
  \pending{
    \key{SEND} \cdname
    \begin{0-1}
      \key{FROM} \identifier
    \end{0-1}
    WITH
    \begin{1=}
      \identifier \\
      \key{ESI} \\
      \key{EMI} \\
      \key{EGI}
    \end{1=}
  }

  \begin{0-1}
    \begin{1=}
      \key{BEFORE} \\
      \key{AFTER}
    \end{1=}
    ADVANCING
    \begin{1=}
      \begin{1=}
        \identifier \\
        \literal
      \end{1=}
      \begin{0-1}
        \key{LINE} \\
        \key{LINES}
      \end{0-1} \\

      \mnemonicname \\

      \key{PAGE}
    \end{1=}
  \end{0-1}

  \pending{
    \begin{0-1}
      \key{REPLACING} LINE
    \end{0-1}
  }
\end{syntax}

\subsubsection{Syntax rules}

\subsubsection{General rules}

\section{SET statement}

The SET statement sets the value or properties of a data item.

\format{simple}
\begin{syntax}
  \key{SET} \identifier \key{TO}
  \begin{1=}
    \identifier \\
    \literal \\
    \arithmeticexpression
  \end{1=}
\end{syntax}

\format{entry}
\begin{syntax}[\gnucobolcolour]
  \key{SET} \identifier \key{TO} \key{ENTRY}
  \begin{1=}
    \identifier \\
    \literal
  \end{1=}
\end{syntax}

\format{environment}
\begin{syntax}[\miscextcolour]
  \key{SET} \key{ENVIRONMENT}
  \begin{1=}
    \identifier \\
    \literal
  \end{1=}
  \key{TO}
  \begin{1=}
    \identifier \\
    \literal
  \end{1=}
\end{syntax}

\format{attribute}
\begin{syntax}
  \key{SET} \identifier \key{ATTRIBUTE}
  \begin{1=}
    \begin{1=}
      \begin{1=}
        \key{BELL} \\
        \miscext{\key{BEEP}}
      \end{1=} \\

      \key{BLINK} \\
      \key{HIGHLIGHT} \\
      \key{LOWLIGHT} \\
      \key{REVERSE-VIDEO} \\
      \key{UNDERLINE} \\
      \key{LEFTLINE} \\
      \key{OVERLINE}
    \end{1=}
    \begin{1=}
      \key{ON} \\
      \key{OFF}
    \end{1=}
  \end{1=}\ldots
\end{syntax}

\format{arithmetic}
\begin{syntax}
  \key{SET}
  \begin{1=}
    \cobolindexname
  \end{1=}\ldots
  \begin{1=}
    \key{UP} \\
    \key{DOWN}
  \end{1=}
  \key{BY}
  \arithmeticexpression
\end{syntax}

\format{on\slash{}off}
\begin{syntax}
  \key{SET}
  \begin{1=}
    \begin{1=}
      \mnemonicname
    \end{1=}\ldots
    \key{TO}
    \begin{1=}
      \key{ON} \\
      \key{OFF}
    \end{1=}
  \end{1=} \ldots
\end{syntax}

\format{true\slash{}false}
\begin{syntax}
  \key{SET}
  \begin{1=}
    \begin{1=}
      \conditionname
    \end{1=}\ldots
    \key{TO}
    \begin{1=}
      \key{TRUE} \\
      \key{FALSE}
    \end{1=}
  \end{1=} \ldots
\end{syntax}

\format{exception}
\begin{syntax}
  \key{SET} \key{LAST} \key{EXCEPTION} \key{TO} \key{OFF}
\end{syntax}

\format{thread}
\begin{syntax}[\miscextcolour]
  \pending{
    \key{SET} \key{THREAD}
    \begin{0-1}
      \identifier
    \end{0-1}
    \key{PRIORITY} \key{TO}
    \begin{1=}
      \identifier
      \integer
    \end{1=}
  }
\end{syntax}

\subsubsection{Syntax rules}

\subsubsection{General rules}

\section{SORT statement}

The SORT statement sorts the record of a file.

\begin{syntax}
  \key{SORT} \identifier
  \begin{0-1}
    ON
    \begin{1=}
      \key{ASCENDING} \\
      \key{DESCENDING}
    \end{1=}
    KEY
    \begin{0-1}
      \identifier
    \end{0-1}\ldots
  \end{0-1} \ldots

  \begin{0-1}
    WITH \key{DUPLICATES}
    \begin{0-1}
      IN \key{ORDER}
    \end{0-1}
  \end{0-1}

  \begin{0-1}
    COLLATING \key{SEQUENCE} IS \identifier
  \end{0-1}

  \begin{0-1}
    \key{USING}
    \begin{1=}
      \filename
    \end{1=}\ldots \\

    \key{INPUT} \key{PROCEDURE} IS
    \procedurename
    \begin{0-1}
      \begin{1=}
        \key{THROUGH} \\
        \key{THRU}
      \end{1=}
      \procedurename
    \end{0-1}
  \end{0-1}

  \begin{0-1}
    \key{GIVING}
    \begin{1=}
      \filename
    \end{1=}\ldots \\

    \key{OUTPUT} \key{PROCEDURE} IS
    \procedurename
    \begin{0-1}
      \begin{1=}
        \key{THROUGH} \\
        \key{THRU}
      \end{1=}
      \procedurename
    \end{0-1}
  \end{0-1}
\end{syntax}

\subsubsection{Syntax rules}

\subsubsection{General rules}

\section{START statement}

The START statement changes the record currently being considered. It may also change the order in which records are accessed.

\begin{syntax}
  \key{START} \filename
  \begin{0-1}
    \key{FIRST} \\

    \key{KEY} IS \metaelement{relational-operator} \identifier \\

    \key{LAST}
  \end{0-1}

  \begin{0-1}
    WITH
    \begin{1=}
      \key{SIZE} \\
      \gnucobol{\key{LENGTH}}
    \end{1=}
    \arithmeticexpression
  \end{0-1}

  \begin{0+}
    \key{INVALID} KEY \imperativestatement \\
    \key{NOT} \key{INVALID} KEY \imperativestatement
  \end{0+}

  \begin{0-1}
    \key{END-START}
  \end{0-1}
\end{syntax}

\subsubsection{Syntax rules}

\subsubsection{General rules}

\section{STOP statement}

The STOP statement terminates the run unit and returns control to the operating system.

\format{standard}

\begin{syntax}
  \key{STOP} \key{RUN}
  \begin{0-1}
    \begin{1=}
      \key{RETURNING} \\
      \miscext{\key{GIVING}}
    \end{1=}
    \begin{1=}
      \identifier \\
      \literal
    \end{1=} \\

    WITH
    \begin{1=}
      \key{ERROR} \\
      \key{NORMAL}
    \end{1=}
    STATUS
    \begin{0-1}
      \identifier \\
      \literal
    \end{0-1}
  \end{0-1}
\end{syntax}

\format{literal}

\begin{syntax}[\deletedcolour]
  \key{STOP} \literal
\end{syntax}

\format{identifier}

\begin{syntax}[\miscextcolour]
  \key{STOP} \identifier
\end{syntax}

\format{ACUCOBOL}

\begin{syntax}[\miscextcolour]
  \key{STOP} \key{RUN}
  \begin{1=}
    \identifier \\
    \literal
  \end{1=}
\end{syntax}

\format{thread}

\begin{syntax}[\miscextcolour]
  \pending{
    \key{STOP} \key{THREAD}
    \begin{0-1}
      \identifier
    \end{0-1}
  }
\end{syntax}

\subsubsection{Syntax rules}

\subsubsection{General rules}

\section{STRING statement}

The STRING statement appends multiples character strings and stores the result.

\begin{syntax}
  \key{STRING}
  \begin{1=}
    \begin{1=}
      \identifier \\
      \literal
    \end{1=}

    \begin{0-1}
      \key{DELIMITED} BY
      \begin{1=}
        \key{SIZE} \\
        \identifier \\
        \literal
      \end{1=}
    \end{0-1}
  \end{1=} \ldots\ {}
  \key{INTO} \identifier

  \begin{0-1}
    WITH \key{POINTER} IS \identifier
  \end{0-1}

  \begin{0+}
    ON \key{OVERFLOW} \imperativestatement \\
    \key{NOT} ON \key{OVERFLOW} \imperativestatement
  \end{0+}
\end{syntax}

\subsubsection{Syntax rules}

\subsubsection{General rules}

\section{SUBTRACT statement}

The SUBTRACT statement subtracts one set of numbers from another set of numbers and stores the results.

\format{simple}
\begin{syntax}
  \key{SUBTRACT}
  \begin{1=}
    \identifier \\
    \literal
  \end{1=} \ldots
  \key{FROM}
  \begin{1=}
    \begin{1=}
      \identifier \\
      \literal
    \end{1=}
    \begin{0-1}
      \metaelement{rounded-phrase}
    \end{0-1}
  \end{1=} \ldots

  \begin{0+}
    ON \key{SIZE} \key{ERROR} \imperativestatement \\
    \key{NOT} ON \key{SIZE} \key{ERROR} \imperativestatement
  \end{0+}

  \begin{0-1}
    \key{END-SUBTRACT}
  \end{0-1}
\end{syntax}

\format{giving}
\begin{syntax}
  \key{SUBTRACT}
  \begin{1=}
    \identifier \\
    \literal
  \end{1=} \ldots
  \key{FROM}
  \begin{1=}
    \identifier \\
    \literal
  \end{1=}

  \key{GIVING}
  \begin{1=}
    \begin{1=}
      \identifier \\
      \literal
    \end{1=}
    \begin{0-1}
      \metaelement{rounded-phrase}
    \end{0-1}
  \end{1=} \ldots

  \begin{0+}
    ON \key{SIZE} \key{ERROR} \imperativestatement \\
    \key{NOT} ON \key{SIZE} \key{ERROR} \imperativestatement
  \end{0+}

  \begin{0-1}
    \key{END-SUBTRACT}
  \end{0-1}
\end{syntax}

\format{corresponding}
\begin{syntax}
  \key{SUBTRACT}
  \begin{1=}
    \key{CORR} \\
    \key{CORRESPONDING}
  \end{1=}
  \identifier{} \key{FROM} \identifier
  \begin{0-1}
    \metaelement{rounded-phrase}
  \end{0-1}

  \begin{0+}
    ON \key{SIZE} \key{ERROR} \imperativestatement \\
    \key{NOT} ON \key{SIZE} \key{ERROR} \imperativestatement
  \end{0+}

  \begin{0-1}
    \key{END-SUBTRACT}
  \end{0-1}
\end{syntax}

\format{table}
\begin{syntax}
  \pending{
    \key{SUBTRACT} \key{TABLE} \identifier \key{TO} \identifier
    \begin{0-1}
      \metaelement{rounded-phrase}
    \end{0-1}
  }

  \pending{
    \begin{0-1}
      \key{FROM} INDEX \integer \key{TO} \integer
    \end{0-1}
  }

  \pending{
    \begin{0-1}
      \key{DESTINATION} INDEX \integer
    \end{0-1}
  }

  \pending{
    \begin{0+}
      ON \key{SIZE} \key{ERROR} \imperativestatement \\
      \key{NOT} ON \key{SIZE} \key{ERROR} \imperativestatement
    \end{0+}
  }

  \pending{
    \begin{0-1}
      \key{END-SUBTRACT}
    \end{0-1}
  }
\end{syntax}

\subsubsection{Syntax rules}

\subsubsection{General rules}

\section{SUPPRESS statement}

The SUPPRESS statement suppresses the writing of a report group.

\begin{syntax}
  \pending{
    \key{SUPPRESS} PRINTING
  }
\end{syntax}

\subsubsection{Syntax rules}

\subsubsection{General rules}

\section{TERMINATE statement}

The TERMINATE statement prevents further writing of a report.

\begin{syntax}
  \pending{
    \key{TERMINATE}
    \begin{1=}
      \reportname
    \end{1=} \ldots
  }
\end{syntax}

\subsubsection{Syntax rules}

\subsubsection{General rules}

\section{TRANSFORM statement}

The TRANSFORM statement replaces instances of character with another character.

\begin{syntax}[\deletedcolour]
  \key{TRANSFORM} \identifier \key{FROM}
  \begin{1=}
    \identifier \\
    \literal
  \end{1=}
  \key{TO}
  \begin{1=}
    \identifier \\
    \literal
  \end{1=}
\end{syntax}

\subsubsection{Syntax rules}

\subsubsection{General rules}

\section{UNLOCK statement}

The UNLOCK statement releases all currently held locks on a file.

\begin{syntax}
  \key{UNLOCK} \filename
  \begin{0-1}
    \key{RECORD} \\
    \key{RECORDS}
  \end{0-1}
\end{syntax}

\subsubsection{Syntax rules}

\subsubsection{General rules}

\section{UNSTRING statement}

The UNSTRING statement extracts substrings from a string and copies the substrings to specified data items.

\begin{syntax}
  \key{UNSTRING}
  \begin{1=}
    \identifier \\
    \miscext{\literal}
  \end{1=}

  \begin{0-1}
    \key{DELIMITED} BY
    \begin{0-1}
      \key{ALL}
    \end{0-1}
    \begin{1=}
      \identifier \\
      \literal
    \end{1=}
    \begin{1=}
      \key{OR}
      \begin{0-1}
        \key{ALL}
      \end{0-1}
      \begin{1=}
        \identifier \\
        \literal
      \end{1=}
    \end{1=} \ldots
  \end{0-1}

  \key{INTO}
  \begin{1=}
    \identifier
    \begin{0-1}
      \key{DELIMITER} IN \identifier
    \end{0-1}
    \begin{0-1}
      \key{COUNT} IN \identifier
    \end{0-1}
  \end{1=} \ldots

  \begin{0-1}
    WITH \key{POINTER} IS \identifier
  \end{0-1}

  \begin{0-1}
    \key{TALLYING} IN \identifier
  \end{0-1}

  \begin{0+}
    ON \key{OVERFLOW} \imperativestatement \\
    \key{NOT} ON \key{OVERFLOW} \imperativestatement
  \end{0+}

  \begin{0-1}
    \key{END-OVERFLOW}
  \end{0-1}
\end{syntax}

\subsubsection{Syntax rules}

\subsubsection{General rules}

\section{USE statement}

The USE statement indicates when a declarative should be executed.

\format{file exception}
\begin{syntax}
  \key{USE}
  \begin{0-1}
    \key{GLOBAL}
  \end{0-1}
  AFTER STANDARD
  \begin{1=}
    \key{EXCEPTION} \\
    \key{ERROR}
  \end{1=}
  PROCEDURE ON

  \begin{1=}
    \begin{1=}
      \filename
    \end{1=} \ldots
    \begin{0+}
      \key{INPUT} \\
      \key{OUTPUT} \\
      \key{I-O} \\
      \key{EXTEND}
    \end{0+} \ldots
  \end{1=}
\end{syntax}

\format{debugging}
\begin{syntax}[\deletedcolour]
  \key{USE} FOR \key{DEBUGGING} ON
  \begin{1=}
    \procedurename \\
    \key{ALL} \key{PROCEDURES} \\
    \key{ALL} REFERENCES OF \identifier
  \end{1=} \ldots
\end{syntax}

\format{start\slash{}end}
\begin{syntax}[\miscextcolour]
  \pending{
    \key{USE} AT \key{PROGRAM}
    \begin{1=}
      \key{START} \\
      \key{END}
    \end{1=}
  }
\end{syntax}

\format{reporting}
\begin{syntax}
  \key{USE}
  \begin{0-1}
    \key{GLOBAL}
  \end{0-1}
  \key{BEFORE} \key{REPORTING} \identifier
\end{syntax}

\format{exception}
\begin{syntax}
  \pending{
    \key{USE}
    \begin{1=}
      \key{EXCEPTION-CONDITION} \\
      \key{EC}
    \end{1=}
  }
\end{syntax}

\subsubsection{Syntax rules}

\subsubsection{General rules}

\section{WRITE statement}

The WRITE statement provides a new record to the file.

\format{sequential}
\begin{syntax}
  \key{WRITE}
  \begin{1=}
    \recordname
    \begin{0-1}
      \key{FROM}
      \begin{1=}
        \identifier \\
        \literal \\
        \functionname
      \end{1=}
    \end{0-1} \\

    \key{FILE} \filename \key{FROM}
    \begin{1=}
      \identifier \\
      \literal \\
      \functionname
    \end{1=}
  \end{1=}

  \begin{0-1}
    \begin{1=}
      \key{BEFORE} \\
      \key{AFTER}
    \end{1=}
    ADVANCING
    \begin{1=}
      \begin{1=}
        \identifier \\
        \literal
      \end{1=}
      \begin{0-1}
        \key{LINE} \\
        \key{LINES}
      \end{0-1} \\

      \mnemonicname \\

      \key{PAGE}
    \end{1=}
  \end{0-1}

  \begin{0-1}
    \pending{\metaelement{retry-phrase}}
  \end{0-1}
  \begin{0-1}
    WITH
    \begin{0-1}
      \key{NO}
    \end{0-1}
    \key{LOCK}
  \end{0-1}

  \begin{0+}
    AT
    \begin{1=}
      \key{END-OF-PAGE} \\
      \key{EOP}
    \end{1=}
    \imperativestatement \\

    \key{NOT} AT
    \begin{1=}
      \key{END-OF-PAGE} \\
      \key{EOP}
    \end{1=}
    \imperativestatement
  \end{0+}

  \begin{0-1}
    \key{END-WRITE}
  \end{0-1}
\end{syntax}

\format{random}
\begin{syntax}
  \key{WRITE}
  \begin{1=}
    \recordname
    \begin{0-1}
      \key{FROM}
      \begin{1=}
        \identifier \\
        \literal \\
        \functionname
      \end{1=}
    \end{0-1} \\

    \key{FILE} \filename \key{FROM}
    \begin{1=}
      \identifier \\
      \literal \\
      \functionname
    \end{1=}
  \end{1=}

  \begin{0-1}
    \begin{1=}
      \key{BEFORE} \\
      \key{AFTER}
    \end{1=}
    ADVANCING
    \begin{1=}
      \begin{1=}
        \identifier \\
        \literal
      \end{1=}
      \begin{0-1}
        \key{LINE} \\
        \key{LINES}
      \end{0-1} \\

      \mnemonicname \\

      \key{PAGE}
    \end{1=}
  \end{0-1}

  \begin{0-1}
    \pending{\metaelement{retry-phrase}}
  \end{0-1}
  \begin{0-1}
    WITH
    \begin{0-1}
      \key{NO}
    \end{0-1}
    \key{LOCK}
  \end{0-1}

  \begin{0+}
    \key{INVALID} \key{KEY} \imperativestatement \\
    \key{NOT} \key{INVALID} \key{KEY} \imperativestatement
  \end{0+} \\

  \begin{0-1}
    \key{END-WRITE}
  \end{0-1}
\end{syntax}

\subsubsection{Syntax rules}

\subsubsection{General rules}


%%% Local Variables:
%%% mode: latex
%%% TeX-master: "grammar.tex"
%%% End:

\chapter{Intrinsic functions}

\section{ABS function}

\begin{syntax}
  \key{FUNCTION}
  \begin{1=}
    \key{ABS} \\
    \miscext{\key{ABSOLUTE-VALUE}}
  \end{1=}
  ( \argument )
\end{syntax}

\subsubsection{Syntax rules}

\subsubsection{General rules}

\section{ACOS function}

\begin{syntax}
  \key{FUNCTION} \key{ACOS} ( \argument)
\end{syntax}

\subsubsection{Syntax rules}

\subsubsection{General rules}

\section{ANNUITY function}

\begin{syntax}
  \key{FUNCTION} \key{ANNUITY} ( \argument \argument )
\end{syntax}

\subsubsection{Syntax rules}

\subsubsection{General rules}

\section{ASIN function}

\begin{syntax}
  \key{FUNCTION} \key{ASIN} ( \argument )
\end{syntax}

\subsubsection{Syntax rules}

\subsubsection{General rules}

\section{ATAN function}

\begin{syntax}
  \key{FUNCTION} \key{ATAN} ( \argument )
\end{syntax}

\subsubsection{Syntax rules}

\subsubsection{General rules}

\section{BOOLEAN-OF-INTEGER function}

\begin{syntax}
  \pending{
    \key{FUNCTION} \key{BOOLEAN-OF-INTEGER} ( \argument \argument )
  }
\end{syntax}

\subsubsection{Syntax rules}

\subsubsection{General rules}

\section{BYTE-LENGTH function}

\begin{syntax}
  \key{FUNCTION} \key{BYTE-LENGTH} ( \argument )
\end{syntax}

\subsubsection{Syntax rules}

\subsubsection{General rules}

\section{CHAR function}

\begin{syntax}
  \key{FUNCTION} \key{CHAR} ( \argument )
\end{syntax}

\subsubsection{Syntax rules}

\subsubsection{General rules}

\section{CHAR-NATIONAL function}

\begin{syntax}
  \pending{
    \key{FUNCTION} \key{CHAR-NATIONAL} ( \argument )
  }
\end{syntax}

\subsubsection{Syntax rules}

\subsubsection{General rules}

\section{COMBINED-DATETIME function}

\begin{syntax}
  \key{FUNCTION} \key{COMBINED-DATETIME} ( \argument \argument )
\end{syntax}

\subsubsection{Syntax rules}

\subsubsection{General rules}

\section{CONCATENATE function}

\begin{syntax}
  \key{FUNCTION} \key{CONCATENATE} (
  \begin{1=}
    \argument
  \end{1=}
  \ldots
  \ {})
\end{syntax}

\subsubsection{Syntax rules}

\subsubsection{General rules}

\section{COS function}

\begin{syntax}
  \key{FUNCTION} \key{COS} ( \argument )
\end{syntax}

\subsubsection{Syntax rules}

\subsubsection{General rules}

\section{CURRENCY-SYMBOL function}

\begin{syntax}[\gnucobolcolour]
  \key{FUNCTION} \key{CURRENCY-SYMBOL}
\end{syntax}

\subsubsection{Syntax rules}

\subsubsection{General rules}

\section{CURRENT-DATE function}

\begin{syntax}
  \key{FUNCTION} \key{CURRENT-DATE}
\end{syntax}

\subsubsection{Syntax rules}

\subsubsection{General rules}

\section{DATE-OF-INTEGER function}

\begin{syntax}
  \key{FUNCTION} \key{DATE-OF-INTEGER} ( \argument )
\end{syntax}

\subsubsection{Syntax rules}

\subsubsection{General rules}

\section{DATE-TO-YYYYMMDD function}

\begin{syntax}
  \key{FUNCTION} \key{DATE-TO-YYYYMMDD} (
  \argument
  \begin{0-1}
    \argument
    \begin{0-1}
      \argument
    \end{0-1}
  \end{0-1}
  \ {})
\end{syntax}

\subsubsection{Syntax rules}

\subsubsection{General rules}

\section{DAY-OF-INTEGER function}

\begin{syntax}
  \key{FUNCTION} \key{DAY-OF-INTEGER} ( \argument )
\end{syntax}

\subsubsection{Syntax rules}

\subsubsection{General rules}

\section{DAY-TO-YYYYDDD function}

\begin{syntax}
  \key{FUNCTION} \key{DAY-TO-YYYYDDD} (
  \argument
  \begin{0-1}
    \argument
    \begin{0-1}
      \argument
    \end{0-1}
  \end{0-1}
  \ {})
\end{syntax}

\subsubsection{Syntax rules}

\subsubsection{General rules}

\section{DISPLAY-OF function}

\begin{syntax}
  \pending{
    \key{FUNCTION} \key{DISPLAY-OF} ( \argument )
  }
\end{syntax}

\subsubsection{Syntax rules}

\subsubsection{General rules}

\section{E function}

\begin{syntax}
  \key{FUNCTION} \key{E}
\end{syntax}

\subsubsection{Syntax rules}

\subsubsection{General rules}

\section{EXCEPTION-FILE function}

\begin{syntax}
  \key{FUNCTION} \key{EXCEPTION-FILE}
\end{syntax}

\subsubsection{Syntax rules}

\subsubsection{General rules}

\section{EXCEPTION-FILE-N function}

\begin{syntax}
  \pending{
    \key{FUNCTION} \key{EXCEPTION-FILE-N}
  }
\end{syntax}

\subsubsection{Syntax rules}

\subsubsection{General rules}

\section{EXCEPTION-LOCATION function}

\begin{syntax}
  \key{FUNCTION} \key{EXCEPTION-LOCATION}
\end{syntax}

\subsubsection{Syntax rules}

\subsubsection{General rules}

\section{EXCEPTION-LOCATION-N function}

\begin{syntax}
  \pending{
    \key{FUNCTION} \key{EXCEPTION-LOCATION-N}
  }
\end{syntax}

\subsubsection{Syntax rules}

\subsubsection{General rules}

\section{EXCEPTION-STATEMENT function}

\begin{syntax}
  \key{FUNCTION} \key{EXCEPTION-STATEMENT}
\end{syntax}

\subsubsection{Syntax rules}

\subsubsection{General rules}

\section{EXCEPTION-STATUS function}

\begin{syntax}
  \key{FUNCTION} \key{EXCEPTION-STATUS}
\end{syntax}

\subsubsection{Syntax rules}

\subsubsection{General rules}

\section{EXP function}

\begin{syntax}
  \key{FUNCTION} \key{EXP} ( \argument )
\end{syntax}

\subsubsection{Syntax rules}

\subsubsection{General rules}

\section{EXP10 function}

\begin{syntax}
  \key{FUNCTION} \key{EXP10} ( \argument )
\end{syntax}

\subsubsection{Syntax rules}

\subsubsection{General rules}

\section{FACTORIAL function}

\begin{syntax}
  \key{FUNCTION} \key{FACTORIAL} ( \argument )
\end{syntax}

\subsubsection{Syntax rules}

\subsubsection{General rules}

\section{FORMATTED-CURRENT-DATE function}

\begin{syntax}
  \key{FUNCTION} \key{FORMATTED-CURRENT-DATE} ( \argument )
\end{syntax}

\subsubsection{Syntax rules}

\subsubsection{General rules}

\section{FORMATTED-DATE function}

\begin{syntax}
  \key{FUNCTION} \key{FORMATTED-DATE} ( \argument \argument)
\end{syntax}

\subsubsection{Syntax rules}

\subsubsection{General rules}

\section{FORMATTED-DATETIME function}

\begin{syntax}
  \key{FUNCTION} \key{FORMATTED-DATETIME}

  ( \argument \argument \argument
  \begin{0-1}
    \argument \\
    \gnucobol{\key{SYSTEM-OFFSET}}
  \end{0-1}
  )
\end{syntax}

\subsubsection{Syntax rules}

\subsubsection{General rules}

\section{FORMATTED-TIME function}

\begin{syntax}
  \key{FUNCTION} \key{FORMATTED-TIME} ( \argument \argument
  \begin{0-1}
    \argument \\
    \gnucobol{\key{SYSTEM-OFFSET}}
  \end{0-1}
  )
\end{syntax}

\subsubsection{Syntax rules}

\subsubsection{General rules}

\section{FRACTION-PART function}

\begin{syntax}
  \key{FUNCTION} \key{FRACTION-PART} ( \argument )
\end{syntax}

\subsubsection{Syntax rules}

\subsubsection{General rules}

\section{HIGHEST-ALGEBRAIC function}

\begin{syntax}
  \key{FUNCTION} \key{HIGHEST-ALGEBRAIC} ( \argument )
\end{syntax}

\subsubsection{Syntax rules}

\subsubsection{General rules}

\section{INTEGER function}

\begin{syntax}
  \key{FUNCTION} \key{INTEGER} ( \argument )
\end{syntax}

\subsubsection{Syntax rules}

\subsubsection{General rules}

\section{INTEGER-OF-BOOLEAN function}

\begin{syntax}
  \pending{
    \key{FUNCTION} \key{INTEGER-OF-BOOLEAN} ( \argument )
  }
\end{syntax}

\subsubsection{Syntax rules}

\subsubsection{General rules}

\section{INTEGER-OF-DATE function}

\begin{syntax}
  \key{FUNCTION} \key{INTEGER-OF-DATE} ( \argument )
\end{syntax}

\subsubsection{Syntax rules}

\subsubsection{General rules}

\section{INTEGER-OF-DAY function}

\begin{syntax}
  \key{FUNCTION} \key{INTEGER-OF-DAY} ( \argument )
\end{syntax}

\subsubsection{Syntax rules}

\subsubsection{General rules}

\section{INTEGER-OF-FORMATTED-DATE function}

\begin{syntax}
  \key{FUNCTION} \key{INTEGER-OF-FORMATTED-DATE} ( \argument \argument )
\end{syntax}

\subsubsection{Syntax rules}

\subsubsection{General rules}

\section{INTEGER-PART function}

\begin{syntax}
  \key{FUNCTION} \key{INTEGER-PART} ( \argument )
\end{syntax}

\subsubsection{Syntax rules}

\subsubsection{General rules}

\section{LENGTH function}

\begin{syntax}
  \key{FUNCTION} \key{LENGTH} ( \argument )
\end{syntax}

\subsubsection{Syntax rules}

\subsubsection{General rules}

\section{LENGTH-AN function}

\begin{syntax}[\miscextcolour]
  \key{FUNCTION} \key{LENGTH-AN} ( \argument )
\end{syntax}

\subsubsection{Syntax rules}

\subsubsection{General rules}

\section{LOCALE-COMPARE function}

\begin{syntax}
  \key{FUNCTION} \key{LOCALE-COMPARE} ( \argument \argument
  \begin{0-1}
    \argument
  \end{0-1}
  )
\end{syntax}

\subsubsection{Syntax rules}

\subsubsection{General rules}

\section{LOCALE-DATE function}

\begin{syntax}
  \key{FUNCTION} \key{LOCALE-DATE} ( \argument
  \begin{0-1}
    \argument
  \end{0-1}
  )
\end{syntax}

\subsubsection{Syntax rules}

\subsubsection{General rules}

\section{LOCALE-TIME function}

\begin{syntax}
  \key{FUNCTION} \key{LOCALE-TIME} ( \argument
  \begin{0-1}
    \argument
  \end{0-1}
  )
\end{syntax}

\subsubsection{Syntax rules}

\subsubsection{General rules}

\section{LOCALE-TIME-FROM-SECONDS function}

\begin{syntax}
  \key{FUNCTION} \key{LOCALE-TIME-FROM-SECONDS} ( \argument
  \begin{0-1}
    \argument
  \end{0-1}
  )
\end{syntax}

\subsubsection{Syntax rules}

\subsubsection{General rules}

\section{LOG function}

\begin{syntax}
  \key{FUNCTION} \key{LOG} ( \argument )
\end{syntax}

\subsubsection{Syntax rules}

\subsubsection{General rules}

\section{LOG10 function}

\begin{syntax}
  \key{FUNCTION} \key{LOG10} ( \argument )
\end{syntax}

\subsubsection{Syntax rules}

\subsubsection{General rules}

\section{LOWER-CASE function}

\begin{syntax}
  \key{FUNCTION} \key{LOWER-CASE} ( \argument )
\end{syntax}

\subsubsection{Syntax rules}

\subsubsection{General rules}

\section{LOWEST-ALGEBRAIC function}

\begin{syntax}
  \key{FUNCTION} \key{LOWEST-ALGEBRAIC} ( \argument )
\end{syntax}

\subsubsection{Syntax rules}

\subsubsection{General rules}

\section{MAX function}

\begin{syntax}
  \key{FUNCTION} \key{MAX} (
  \begin{1=}
    \argument
  \end{1=}\ldots
  \ {})
\end{syntax}

\subsubsection{Syntax rules}

\subsubsection{General rules}

\section{MEAN function}

\begin{syntax}
  \key{FUNCTION} \key{MEAN} (
  \begin{1=}
    \argument
  \end{1=}\ldots
  \ {})
\end{syntax}

\subsubsection{Syntax rules}

\subsubsection{General rules}

\section{MEDIAN function}

\begin{syntax}
  \key{FUNCTION} \key{MEDIAN} (
  \begin{1=}
    \argument
  \end{1=}\ldots
  \ {})
\end{syntax}

\subsubsection{Syntax rules}

\subsubsection{General rules}

\section{MIDRANGE function}

\begin{syntax}
  \key{FUNCTION} \key{MIDRANGE} (
  \begin{1=}
    \argument
  \end{1=}\ldots
  \ {})
\end{syntax}

\subsubsection{Syntax rules}

\subsubsection{General rules}

\section{MIN function}

\begin{syntax}
  \key{FUNCTION} \key{MIN} (
  \begin{1=}
    \argument
  \end{1=}\ldots
  \ {})
\end{syntax}

\subsubsection{Syntax rules}

\subsubsection{General rules}

\section{MOD function}

\begin{syntax}
  \key{FUNCTION} \key{MOD} ( \argument \argument )
\end{syntax}

\subsubsection{Syntax rules}

\subsubsection{General rules}

\section{MODULE-CALLER-ID function}

\begin{syntax}[\gnucobolcolour]
  \key{FUNCTION} \key{MODULE-CALLER-ID}
\end{syntax}

\subsubsection{Syntax rules}

\subsubsection{General rules}

\section{MODULE-DATE function}

\begin{syntax}[\gnucobolcolour]
  \key{FUNCTION} \key{MODULE-DATE}
\end{syntax}

\subsubsection{Syntax rules}

\subsubsection{General rules}

\section{MODULE-FORMATTED-DATE function}

\begin{syntax}[\gnucobolcolour]
  \key{FUNCTION} \key{MODULE-FORMATTED-DATE}
\end{syntax}

\subsubsection{Syntax rules}

\subsubsection{General rules}

\section{MODULE-ID function}

\begin{syntax}[\gnucobolcolour]
  \key{FUNCTION} \key{MODULE-ID}
\end{syntax}

\subsubsection{Syntax rules}

\subsubsection{General rules}

\section{MODULE-PATH function}

\begin{syntax}[\gnucobolcolour]
  \key{FUNCTION} \key{MODULE-PATH}
\end{syntax}

\subsubsection{Syntax rules}

\subsubsection{General rules}

\section{MODULE-SOURCE function}

\begin{syntax}[\gnucobolcolour]
  \key{FUNCTION} \key{MODULE-SOURCE}
\end{syntax}

\subsubsection{Syntax rules}

\subsubsection{General rules}

\section{MODULE-TIME function}

\begin{syntax}[\gnucobolcolour]
  \key{FUNCTION} \key{MODULE-TIME}
\end{syntax}

\subsubsection{Syntax rules}

\subsubsection{General rules}

\section{MONETARY-DECIMAL-POINT function}

\begin{syntax}[\gnucobolcolour]
  \key{FUNCTION} \key{MONETARY-DECIMAL-POINT}
\end{syntax}

\subsubsection{Syntax rules}

\subsubsection{General rules}

\section{MONETARY-THOUSANDS-SEPARATOR function}

\begin{syntax}[\gnucobolcolour]
  \key{FUNCTION} \key{MONETARY-THOUSANDS-SEPARATOR}
\end{syntax}

\subsubsection{Syntax rules}

\subsubsection{General rules}

\section{NATIONAL-OF function}

\begin{syntax}
  \pending{
    \key{FUNCTION} \key{NATIONAL-OF} ( \argument
    \begin{0-1}
      \argument
    \end{0-1}
    )
  }
\end{syntax}

\subsubsection{Syntax rules}

\subsubsection{General rules}

\section{NUMERIC-DECIMAL-POINT function}

\begin{syntax}[\gnucobolcolour]
  \key{FUNCTION} \key{NUMERIC-DECIMAL-POINT}
\end{syntax}

\subsubsection{Syntax rules}

\subsubsection{General rules}

\section{NUMERIC-THOUSANDS-SEPARATOR function}

\begin{syntax}[\gnucobolcolour]
  \key{FUNCTION} \key{NUMERIC-THOUSANDS-SEPARATOR}
\end{syntax}

\subsubsection{Syntax rules}

\subsubsection{General rules}

\section{NUMVAL function}

\begin{syntax}
  \key{FUNCTION} \key{NUMVAL} ( \argument )
\end{syntax}

\subsubsection{Syntax rules}

\subsubsection{General rules}

\section{NUMVAL-C function}

\begin{syntax}
  \key{FUNCTION} \key{NUMVAL-C} ( \argument
  \begin{0-1}
    \argument
  \end{0-1}
  )
\end{syntax}

\subsubsection{Syntax rules}

\subsubsection{General rules}

\section{NUMVAL-F function}

\begin{syntax}
  \key{FUNCTION} \key{NUMVAL-F} ( \argument )
\end{syntax}

\subsubsection{Syntax rules}

\subsubsection{General rules}

\section{ORD function}

\begin{syntax}
  \key{FUNCTION} \key{ORD} ( \argument )
\end{syntax}

\subsubsection{Syntax rules}

\subsubsection{General rules}

\section{ORD-MAX function}

\begin{syntax}
  \key{FUNCTION} \key{ORD-MAX} (
  \begin{1=}
    \argument
  \end{1=} \ldots
  \ {})
\end{syntax}

\subsubsection{Syntax rules}

\subsubsection{General rules}

\section{ORD-MIN function}

\begin{syntax}
  \key{FUNCTION} \key{ORD-MIN} (
  \begin{1=}
    \argument
  \end{1=} \ldots
  \ {})
\end{syntax}

\subsubsection{Syntax rules}

\subsubsection{General rules}

\section{PI function}

\begin{syntax}
  \key{FUNCTION} \key{PI}
\end{syntax}

\subsubsection{Syntax rules}

\subsubsection{General rules}

\section{PRESENT-VALUE function}

\begin{syntax}
  \key{FUNCTION} \key{PRESENT-VALUE} (
  \begin{1=}
    \argument
  \end{1=} \ldots
  \ {})
\end{syntax}

\subsubsection{Syntax rules}

\subsubsection{General rules}

\section{RANDOM function}

\begin{syntax}
  \key{FUNCTION} \key{RANDOM}
  \begin{0-1}
    (
    \begin{0-1}
      \argument
    \end{0-1} \gnucobol{\ldots}\ {}
    )
  \end{0-1}
\end{syntax}

\subsubsection{Syntax rules}

\subsubsection{General rules}

\section{RANGE function}

\begin{syntax}
  \key{FUNCTION} \key{RANGE} (
  \begin{1=}
    \argument
  \end{1=}\ldots
  \ {})
\end{syntax}

\subsubsection{Syntax rules}

\subsubsection{General rules}

\section{REM function}

\begin{syntax}
  \key{FUNCTION} \key{REM} ( \argument \argument )
\end{syntax}

\subsubsection{Syntax rules}

\subsubsection{General rules}

\section{REVERSE function}

\begin{syntax}
  \key{FUNCTION} \key{REVERSE} ( \argument )
\end{syntax}

\subsubsection{Syntax rules}

\subsubsection{General rules}

\section{SECONDS-FROM-FORMATTED-TIME function}

\begin{syntax}
  \key{FUNCTION} \key{SECONDS-FROM-FORMATTED-TIME} ( \argument \argument )
\end{syntax}

\subsubsection{Syntax rules}

\subsubsection{General rules}

\section{SECONDS-PAST-MIDNIGHT function}

\begin{syntax}
  \key{FUNCTION} \key{SECONDS-PAST-MIDNIGHT} ( \argument )
\end{syntax}

\subsubsection{Syntax rules}

\subsubsection{General rules}

\section{SIGN function}

\begin{syntax}
  \key{FUNCTION} \key{SIGN} ( \argument )
\end{syntax}

\subsubsection{Syntax rules}

\subsubsection{General rules}

\section{SIN function}

\begin{syntax}
  \key{FUNCTION} \key{SIN} ( \argument )
\end{syntax}

\subsubsection{Syntax rules}

\subsubsection{General rules}

\section{SQRT function}

\begin{syntax}
  \key{FUNCTION} \key{SQRT} ( \argument )
\end{syntax}

\subsubsection{Syntax rules}

\subsubsection{General rules}

\section{STANDARD-COMPARE function}

\begin{syntax}
  \pending{
    \key{FUNCTION} \key{STANDARD-COMPARE}
  }

  \pending{
    ( \argument \argument
    \begin{0-1}
      \argument
    \end{0-1}
    \begin{0-1}
      \argument
    \end{0-1}
    )
  }
\end{syntax}

\subsubsection{Syntax rules}

\subsubsection{General rules}

\section{STANDARD-DEVIATION function}

\begin{syntax}
  \key{FUNCTION} \key{STANDARD-DEVIATION} (
  \begin{1=}
    \argument
  \end{1=}\ldots
  \ {})
\end{syntax}

\subsubsection{Syntax rules}

\subsubsection{General rules}

\section{STORED-CHAR-LENGTH function}

\begin{syntax}[\gnucobolcolour]
  \key{FUNCTION} \key{STORED-CHAR-LENGTH} ( \argument )
\end{syntax}

\subsubsection{Syntax rules}

\subsubsection{General rules}

\section{SUBSTITUTE function}

\begin{syntax}[\gnucobolcolour]
  \key{FUNCTION} \key{SUBSTITUTE} ( \argument
  \begin{1=}
    \argument \argument
  \end{1=}\ldots\ {}
  )
\end{syntax}

\subsubsection{Syntax rules}

\subsubsection{General rules}

\section{SUBSTITUTE-CASE function}

\begin{syntax}[\gnucobolcolour]
  \key{FUNCTION} \key{SUBSTITUTE-CASE} ( \argument
  \begin{1=}
    \argument \argument
  \end{1=}\ldots\ {}
  )
\end{syntax}

\subsubsection{Syntax rules}

\subsubsection{General rules}

\section{SUM function}

\begin{syntax}
  \key{FUNCTION} \key{SUM} (
  \begin{1=}
    \argument
  \end{1=}\ldots
  \ {})
\end{syntax}

\subsubsection{Syntax rules}

\subsubsection{General rules}

\section{TAN function}

\begin{syntax}
  \key{FUNCTION} \key{TAN} ( \argument )
\end{syntax}

\subsubsection{Syntax rules}

\subsubsection{General rules}

\section{TEST-DATE-YYYYMMDD function}

\begin{syntax}
  \key{FUNCTION} \key{TEST-DATE-YYYYMMDD} ( \argument )
\end{syntax}

\subsubsection{Syntax rules}

\subsubsection{General rules}

\section{TEST-DAY-YYYYDDD function}

\begin{syntax}
  \key{FUNCTION} \key{TEST-DAY-YYYYDDD} ( \argument )
\end{syntax}

\subsubsection{Syntax rules}

\subsubsection{General rules}

\section{TEST-FORMATTED-DATETIME function}

\begin{syntax}
  \key{FUNCTION} \key{TEST-FORMATTED-DATETIME} ( \argument \argument )
\end{syntax}

\subsubsection{Syntax rules}

\subsubsection{General rules}

\section{TEST-NUMVAL function}

\begin{syntax}
  \key{FUNCTION} \key{TEST-NUMVAL} ( \argument )
\end{syntax}

\subsubsection{Syntax rules}

\subsubsection{General rules}

\section{TEST-NUMVAL-C function}

\begin{syntax}
  \key{FUNCTION} \key{TEST-NUMVAL-C} ( \argument \argument )
\end{syntax}

\subsubsection{Syntax rules}

\subsubsection{General rules}

\section{TEST-NUMVAL-F function}

\begin{syntax}
  \key{FUNCTION} \key{TEST-NUMVAL-F} ( \argument )
\end{syntax}

\subsubsection{Syntax rules}

\subsubsection{General rules}

\section{TRIM function}

\begin{syntax}
  \key{FUNCTION} \key{TRIM} ( \argument
  \begin{0-1}
    \key{LEADING} \\
    \key{TRAILING}
  \end{0-1}
  )
\end{syntax}

\subsubsection{Syntax rules}

\subsubsection{General rules}

\section{UPPER-CASE function}

\begin{syntax}
  \key{FUNCTION} \key{UPPER-CASE} ( \argument )
\end{syntax}

\subsubsection{Syntax rules}

\subsubsection{General rules}

\section{VARIANCE function}

\begin{syntax}
  \key{FUNCTION} \key{VARIANCE} (
  \begin{1=}
    \argument
  \end{1=}\ldots
  \ {})
\end{syntax}

\subsubsection{Syntax rules}

\subsubsection{General rules}

\section{WHEN-COMPILED function}

\begin{syntax}
  \key{FUNCTION} \key{WHEN-COMPILED}
\end{syntax}

\subsubsection{Syntax rules}

\subsubsection{General rules}

\section{YEAR-TO-YYYY function}

\begin{syntax}
  \key{FUNCTION} \key{YEAR-TO-YYYY} ( \argument
  \begin{0-1}
    \argument
    \begin{0-1}
      \argument
    \end{0-1}
  \end{0-1}
  )
\end{syntax}

\subsubsection{Syntax rules}

\subsubsection{General rules}

%%% Local Variables:
%%% mode: latex
%%% TeX-master: "grammar.tex"
%%% End:


\begin{appendices}
  \chapter{GNU Free Documentation License}
\label{label_fdl}

 \begin{center}

       Version 1.3, 3 November 2008


 Copyright \textcopyright{} 2000, 2001, 2002, 2007, 2008  Free Software Foundation, Inc.
 
 \bigskip
 
     \texttt{<http://fsf.org/>}
  
 \bigskip
 
 Everyone is permitted to copy and distribute verbatim copies
 of this license document, but changing it is not allowed.
\end{center}


\section{Preamble}

The purpose of this License is to make a manual, textbook, or other
functional and useful document ``free'' in the sense of freedom: to
assure everyone the effective freedom to copy and redistribute it,
with or without modifying it, either commercially or noncommercially.
Secondarily, this License preserves for the author and publisher a way
to get credit for their work, while not being considered responsible
for modifications made by others.

This License is a kind of ``copyleft'', which means that derivative
works of the document must themselves be free in the same sense.  It
complements the GNU General Public License, which is a copyleft
license designed for free software.

We have designed this License in order to use it for manuals for free
software, because free software needs free documentation: a free
program should come with manuals providing the same freedoms that the
software does.  But this License is not limited to software manuals;
it can be used for any textual work, regardless of subject matter or
whether it is published as a printed book.  We recommend this License
principally for works whose purpose is instruction or reference.


\section{Applicability and definitions}

This License applies to any manual or other work, in any medium, that
contains a notice placed by the copyright holder saying it can be
distributed under the terms of this License.  Such a notice grants a
world-wide, royalty-free license, unlimited in duration, to use that
work under the conditions stated herein.  The ``\textbf{Document}'', below,
refers to any such manual or work.  Any member of the public is a
licensee, and is addressed as ``\textbf{you}''.  You accept the license if you
copy, modify or distribute the work in a way requiring permission
under copyright law.

A ``\textbf{Modified Version}'' of the Document means any work containing the
Document or a portion of it, either copied verbatim, or with
modifications and/or translated into another language.

A ``\textbf{Secondary Section}'' is a named appendix or a front-matter section of
the Document that deals exclusively with the relationship of the
publishers or authors of the Document to the Document's overall subject
(or to related matters) and contains nothing that could fall directly
within that overall subject.  (Thus, if the Document is in part a
textbook of mathematics, a Secondary Section may not explain any
mathematics.)  The relationship could be a matter of historical
connection with the subject or with related matters, or of legal,
commercial, philosophical, ethical or political position regarding
them.

The ``\textbf{Invariant Sections}'' are certain Secondary Sections whose titles
are designated, as being those of Invariant Sections, in the notice
that says that the Document is released under this License.  If a
section does not fit the above definition of Secondary then it is not
allowed to be designated as Invariant.  The Document may contain zero
Invariant Sections.  If the Document does not identify any Invariant
Sections then there are none.

The ``\textbf{Cover Texts}'' are certain short passages of text that are listed,
as Front-Cover Texts or Back-Cover Texts, in the notice that says that
the Document is released under this License.  A Front-Cover Text may
be at most 5 words, and a Back-Cover Text may be at most 25 words.

A ``\textbf{Transparent}'' copy of the Document means a machine-readable copy,
represented in a format whose specification is available to the
general public, that is suitable for revising the document
straightforwardly with generic text editors or (for images composed of
pixels) generic paint programs or (for drawings) some widely available
drawing editor, and that is suitable for input to text formatters or
for automatic translation to a variety of formats suitable for input
to text formatters.  A copy made in an otherwise Transparent file
format whose markup, or absence of markup, has been arranged to thwart
or discourage subsequent modification by readers is not Transparent.
An image format is not Transparent if used for any substantial amount
of text.  A copy that is not ``Transparent'' is called ``\textbf{Opaque}''.

Examples of suitable formats for Transparent copies include plain
ASCII without markup, Texinfo input format, LaTeX input format, SGML
or XML using a publicly available DTD, and standard-conforming simple
HTML, PostScript or PDF designed for human modification.  Examples of
transparent image formats include PNG, XCF and JPG.  Opaque formats
include proprietary formats that can be read and edited only by
proprietary word processors, SGML or XML for which the DTD and/or
processing tools are not generally available, and the
machine-generated HTML, PostScript or PDF produced by some word
processors for output purposes only.

The ``\textbf{Title Page}'' means, for a printed book, the title page itself,
plus such following pages as are needed to hold, legibly, the material
this License requires to appear in the title page.  For works in
formats which do not have any title page as such, ``Title Page'' means
the text near the most prominent appearance of the work's title,
preceding the beginning of the body of the text.

The ``\textbf{publisher}'' means any person or entity that distributes
copies of the Document to the public.

A section ``\textbf{Entitled XYZ}'' means a named subunit of the Document whose
title either is precisely XYZ or contains XYZ in parentheses following
text that translates XYZ in another language.  (Here XYZ stands for a
specific section name mentioned below, such as ``\textbf{Acknowledgements}'',
``\textbf{Dedications}'', ``\textbf{Endorsements}'', or ``\textbf{History}''.)  
To ``\textbf{Preserve the Title}''
of such a section when you modify the Document means that it remains a
section ``Entitled XYZ'' according to this definition.

The Document may include Warranty Disclaimers next to the notice which
states that this License applies to the Document.  These Warranty
Disclaimers are considered to be included by reference in this
License, but only as regards disclaiming warranties: any other
implication that these Warranty Disclaimers may have is void and has
no effect on the meaning of this License.

\section{Verbatim copying}

You may copy and distribute the Document in any medium, either
commercially or noncommercially, provided that this License, the
copyright notices, and the license notice saying this License applies
to the Document are reproduced in all copies, and that you add no other
conditions whatsoever to those of this License.  You may not use
technical measures to obstruct or control the reading or further
copying of the copies you make or distribute.  However, you may accept
compensation in exchange for copies.  If you distribute a large enough
number of copies you must also follow the conditions in section~3.

You may also lend copies, under the same conditions stated above, and
you may publicly display copies.

\section{Copying in quantity}

If you publish printed copies (or copies in media that commonly have
printed covers) of the Document, numbering more than 100, and the
Document's license notice requires Cover Texts, you must enclose the
copies in covers that carry, clearly and legibly, all these Cover
Texts: Front-Cover Texts on the front cover, and Back-Cover Texts on
the back cover.  Both covers must also clearly and legibly identify
you as the publisher of these copies.  The front cover must present
the full title with all words of the title equally prominent and
visible.  You may add other material on the covers in addition.
Copying with changes limited to the covers, as long as they preserve
the title of the Document and satisfy these conditions, can be treated
as verbatim copying in other respects.

If the required texts for either cover are too voluminous to fit
legibly, you should put the first ones listed (as many as fit
reasonably) on the actual cover, and continue the rest onto adjacent
pages.

If you publish or distribute Opaque copies of the Document numbering
more than 100, you must either include a machine-readable Transparent
copy along with each Opaque copy, or state in or with each Opaque copy
a computer-network location from which the general network-using
public has access to download using public-standard network protocols
a complete Transparent copy of the Document, free of added material.
If you use the latter option, you must take reasonably prudent steps,
when you begin distribution of Opaque copies in quantity, to ensure
that this Transparent copy will remain thus accessible at the stated
location until at least one year after the last time you distribute an
Opaque copy (directly or through your agents or retailers) of that
edition to the public.

It is requested, but not required, that you contact the authors of the
Document well before redistributing any large number of copies, to give
them a chance to provide you with an updated version of the Document.


\section{Modifications}

You may copy and distribute a Modified Version of the Document under
the conditions of sections 2 and 3 above, provided that you release
the Modified Version under precisely this License, with the Modified
Version filling the role of the Document, thus licensing distribution
and modification of the Modified Version to whoever possesses a copy
of it.  In addition, you must do these things in the Modified Version:

\begin{itemize}
\item[A.] 
   Use in the Title Page (and on the covers, if any) a title distinct
   from that of the Document, and from those of previous versions
   (which should, if there were any, be listed in the History section
   of the Document).  You may use the same title as a previous version
   if the original publisher of that version gives permission.
   
\item[B.]
   List on the Title Page, as authors, one or more persons or entities
   responsible for authorship of the modifications in the Modified
   Version, together with at least five of the principal authors of the
   Document (all of its principal authors, if it has fewer than five),
   unless they release you from this requirement.
   
\item[C.]
   State on the Title page the name of the publisher of the
   Modified Version, as the publisher.
   
\item[D.]
   Preserve all the copyright notices of the Document.
   
\item[E.]
   Add an appropriate copyright notice for your modifications
   adjacent to the other copyright notices.
   
\item[F.]
   Include, immediately after the copyright notices, a license notice
   giving the public permission to use the Modified Version under the
   terms of this License, in the form shown in the Addendum below.
   
\item[G.]
   Preserve in that license notice the full lists of Invariant Sections
   and required Cover Texts given in the Document's license notice.
   
\item[H.]
   Include an unaltered copy of this License.
   
\item[I.]
   Preserve the section Entitled ``History'', Preserve its Title, and add
   to it an item stating at least the title, year, new authors, and
   publisher of the Modified Version as given on the Title Page.  If
   there is no section Entitled ``History'' in the Document, create one
   stating the title, year, authors, and publisher of the Document as
   given on its Title Page, then add an item describing the Modified
   Version as stated in the previous sentence.
   
\item[J.]
   Preserve the network location, if any, given in the Document for
   public access to a Transparent copy of the Document, and likewise
   the network locations given in the Document for previous versions
   it was based on.  These may be placed in the ``History'' section.
   You may omit a network location for a work that was published at
   least four years before the Document itself, or if the original
   publisher of the version it refers to gives permission.
   
\item[K.]
   For any section Entitled ``Acknowledgements'' or ``Dedications'',
   Preserve the Title of the section, and preserve in the section all
   the substance and tone of each of the contributor acknowledgements
   and/or dedications given therein.
   
\item[L.]
   Preserve all the Invariant Sections of the Document,
   unaltered in their text and in their titles.  Section numbers
   or the equivalent are not considered part of the section titles.
   
\item[M.]
   Delete any section Entitled ``Endorsements''.  Such a section
   may not be included in the Modified Version.
   
\item[N.]
   Do not retitle any existing section to be Entitled ``Endorsements''
   or to conflict in title with any Invariant Section.
   
\item[O.]
   Preserve any Warranty Disclaimers.
\end{itemize}

If the Modified Version includes new front-matter sections or
appendices that qualify as Secondary Sections and contain no material
copied from the Document, you may at your option designate some or all
of these sections as invariant.  To do this, add their titles to the
list of Invariant Sections in the Modified Version's license notice.
These titles must be distinct from any other section titles.

You may add a section Entitled ``Endorsements'', provided it contains
nothing but endorsements of your Modified Version by various
parties---for example, statements of peer review or that the text has
been approved by an organization as the authoritative definition of a
standard.

You may add a passage of up to five words as a Front-Cover Text, and a
passage of up to 25 words as a Back-Cover Text, to the end of the list
of Cover Texts in the Modified Version.  Only one passage of
Front-Cover Text and one of Back-Cover Text may be added by (or
through arrangements made by) any one entity.  If the Document already
includes a cover text for the same cover, previously added by you or
by arrangement made by the same entity you are acting on behalf of,
you may not add another; but you may replace the old one, on explicit
permission from the previous publisher that added the old one.

The author(s) and publisher(s) of the Document do not by this License
give permission to use their names for publicity for or to assert or
imply endorsement of any Modified Version.

\section{Combining documents}

You may combine the Document with other documents released under this
License, under the terms defined in section~4 above for modified
versions, provided that you include in the combination all of the
Invariant Sections of all of the original documents, unmodified, and
list them all as Invariant Sections of your combined work in its
license notice, and that you preserve all their Warranty Disclaimers.

The combined work need only contain one copy of this License, and
multiple identical Invariant Sections may be replaced with a single
copy.  If there are multiple Invariant Sections with the same name but
different contents, make the title of each such section unique by
adding at the end of it, in parentheses, the name of the original
author or publisher of that section if known, or else a unique number.
Make the same adjustment to the section titles in the list of
Invariant Sections in the license notice of the combined work.

In the combination, you must combine any sections Entitled ``History''
in the various original documents, forming one section Entitled
``History''; likewise combine any sections Entitled ``Acknowledgements'',
and any sections Entitled ``Dedications''.  You must delete all sections
Entitled ``Endorsements''.

\section{Collections of documents}

You may make a collection consisting of the Document and other documents
released under this License, and replace the individual copies of this
License in the various documents with a single copy that is included in
the collection, provided that you follow the rules of this License for
verbatim copying of each of the documents in all other respects.

You may extract a single document from such a collection, and distribute
it individually under this License, provided you insert a copy of this
License into the extracted document, and follow this License in all
other respects regarding verbatim copying of that document.

\section{Aggregation with independent works}

A compilation of the Document or its derivatives with other separate
and independent documents or works, in or on a volume of a storage or
distribution medium, is called an ``aggregate'' if the copyright
resulting from the compilation is not used to limit the legal rights
of the compilation's users beyond what the individual works permit.
When the Document is included in an aggregate, this License does not
apply to the other works in the aggregate which are not themselves
derivative works of the Document.

If the Cover Text requirement of section~3 is applicable to these
copies of the Document, then if the Document is less than one half of
the entire aggregate, the Document's Cover Texts may be placed on
covers that bracket the Document within the aggregate, or the
electronic equivalent of covers if the Document is in electronic form.
Otherwise they must appear on printed covers that bracket the whole
aggregate.


\section{Translation}

Translation is considered a kind of modification, so you may
distribute translations of the Document under the terms of section~4.
Replacing Invariant Sections with translations requires special
permission from their copyright holders, but you may include
translations of some or all Invariant Sections in addition to the
original versions of these Invariant Sections.  You may include a
translation of this License, and all the license notices in the
Document, and any Warranty Disclaimers, provided that you also include
the original English version of this License and the original versions
of those notices and disclaimers.  In case of a disagreement between
the translation and the original version of this License or a notice
or disclaimer, the original version will prevail.

If a section in the Document is Entitled ``Acknowledgements'',
``Dedications'', or ``History'', the requirement (section~4) to Preserve
its Title (section~1) will typically require changing the actual
title.

\section{Termination}

You may not copy, modify, sublicense, or distribute the Document
except as expressly provided under this License.  Any attempt
otherwise to copy, modify, sublicense, or distribute it is void, and
will automatically terminate your rights under this License.

However, if you cease all violation of this License, then your license
from a particular copyright holder is reinstated (a) provisionally,
unless and until the copyright holder explicitly and finally
terminates your license, and (b) permanently, if the copyright holder
fails to notify you of the violation by some reasonable means prior to
60 days after the cessation.

Moreover, your license from a particular copyright holder is
reinstated permanently if the copyright holder notifies you of the
violation by some reasonable means, this is the first time you have
received notice of violation of this License (for any work) from that
copyright holder, and you cure the violation prior to 30 days after
your receipt of the notice.

Termination of your rights under this section does not terminate the
licenses of parties who have received copies or rights from you under
this License.  If your rights have been terminated and not permanently
reinstated, receipt of a copy of some or all of the same material does
not give you any rights to use it.

\section{Future revisions of this license}

The Free Software Foundation may publish new, revised versions
of the GNU Free Documentation License from time to time.  Such new
versions will be similar in spirit to the present version, but may
differ in detail to address new problems or concerns.  See
\texttt{http://www.gnu.org/copyleft/}.

Each version of the License is given a distinguishing version number.
If the Document specifies that a particular numbered version of this
License ``or any later version'' applies to it, you have the option of
following the terms and conditions either of that specified version or
of any later version that has been published (not as a draft) by the
Free Software Foundation.  If the Document does not specify a version
number of this License, you may choose any version ever published (not
as a draft) by the Free Software Foundation.  If the Document
specifies that a proxy can decide which future versions of this
License can be used, that proxy's public statement of acceptance of a
version permanently authorizes you to choose that version for the
Document.

\section{Relicensing}
  
``Massive Multiauthor Collaboration Site'' (or ``MMC Site'') means any
World Wide Web server that publishes copyrightable works and also
provides prominent facilities for anybody to edit those works.  A
public wiki that anybody can edit is an example of such a server.  A
``Massive Multiauthor Collaboration'' (or ``MMC'') contained in the
site means any set of copyrightable works thus published on the MMC
site.

``CC-BY-SA'' means the Creative Commons Attribution-Share Alike 3.0
license published by Creative Commons Corporation, a not-for-profit
corporation with a principal place of business in San Francisco,
California, as well as future copyleft versions of that license
published by that same organization.

``Incorporate'' means to publish or republish a Document, in whole or
in part, as part of another Document.

An MMC is ``eligible for relicensing'' if it is licensed under this
License, and if all works that were first published under this License
somewhere other than this MMC, and subsequently incorporated in whole
or in part into the MMC, (1) had no cover texts or invariant sections,
and (2) were thus incorporated prior to November 1, 2008.

The operator of an MMC Site may republish an MMC contained in the site
under CC-BY-SA on the same site at any time before August 1, 2009,
provided the MMC is eligible for relicensing.

\section{Addendum: How to use this License for your documents}

To use this License in a document you have written, include a copy of
the License in the document and put the following copyright and
license notices just after the title page:

\bigskip
\begin{quote}
    Copyright \textcopyright{} YEAR YOUR NAME.
    Permission is granted to copy, distribute and/or modify this document
    under the terms of the GNU Free Documentation License, Version 1.3
    or any later version published by the Free Software Foundation;
    with no Invariant Sections, no Front-Cover Texts, and no Back-Cover Texts.
    A copy of the license is included in the section entitled ``GNU
    Free Documentation License''.
\end{quote}
\bigskip
    
If you have Invariant Sections, Front-Cover Texts and Back-Cover Texts,
replace the ``with \dots\ Texts.''\ line with this:

\bigskip
\begin{quote}
    with the Invariant Sections being LIST THEIR TITLES, with the
    Front-Cover Texts being LIST, and with the Back-Cover Texts being LIST.
\end{quote}
\bigskip
    
If you have Invariant Sections without Cover Texts, or some other
combination of the three, merge those two alternatives to suit the
situation.

If your document contains nontrivial examples of program code, we
recommend releasing these examples in parallel under your choice of
free software license, such as the GNU General Public License,
to permit their use in free software.

%%% Local Variables:
%%% mode: latex
%%% TeX-master: "grammar.tex"
%%% End:

\end{appendices}

\end{document}
